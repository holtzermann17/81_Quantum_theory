\documentclass[12pt]{article}
\usepackage{pmmeta}
\pmcanonicalname{HamiltonianAlgebroids}
\pmcreated{2013-03-22 18:13:44}
\pmmodified{2013-03-22 18:13:44}
\pmowner{bci1}{20947}
\pmmodifier{bci1}{20947}
\pmtitle{Hamiltonian algebroids}
\pmrecord{42}{40815}
\pmprivacy{1}
\pmauthor{bci1}{20947}
\pmtype{Topic}
\pmcomment{trigger rebuild}
\pmclassification{msc}{81P05}
\pmclassification{msc}{81R15}
\pmclassification{msc}{81R10}
\pmclassification{msc}{81R05}
\pmclassification{msc}{81R50}
\pmsynonym{quantum algebroid}{HamiltonianAlgebroids}
%\pmkeywords{extended quantum algebroid symmetry}
%\pmkeywords{Hamiltonian algebroid of a quantum system}
%\pmkeywords{generalizations of Lie algebras of canonical transformations}
%\pmkeywords{Poisson sigma model phase space}
%\pmkeywords{Relativistic Quantum Gravity}
\pmrelated{HamiltonianOperatorOfAQuantumSystem}
\pmrelated{JordanBanachAndJordanLieAlgebras}
\pmrelated{LieBracket}
\pmrelated{LieAlgebroids}
\pmrelated{QuantumGravityTheories}
\pmrelated{Algebroids}
\pmrelated{RCategory}
\pmrelated{RAlgebroid}
\pmdefines{Hamiltonian algebroid}
\pmdefines{Jordan algebra}
\pmdefines{Poisson algebra}

% this is the default PlanetMath preamble. 

\usepackage{amssymb}
\usepackage{amsmath}
\usepackage{amsfonts}

% define commands here
\usepackage{amsmath, amssymb, amsfonts, amsthm, amscd, latexsym}
%%\usepackage{xypic}
\usepackage[mathscr]{eucal}

\setlength{\textwidth}{6.5in}
%\setlength{\textwidth}{16cm}
\setlength{\textheight}{9.0in}
%\setlength{\textheight}{24cm}

\hoffset=-.75in %%ps format
%\hoffset=-1.0in %%hp format
\voffset=-.4in

\theoremstyle{plain}
\newtheorem{lemma}{Lemma}[section]
\newtheorem{proposition}{Proposition}[section]
\newtheorem{theorem}{Theorem}[section]
\newtheorem{corollary}{Corollary}[section]

\theoremstyle{definition}
\newtheorem{definition}{Definition}[section]
\newtheorem{example}{Example}[section]
%\theoremstyle{remark}
\newtheorem{remark}{Remark}[section]
\newtheorem*{notation}{Notation}
\newtheorem*{claim}{Claim}

\renewcommand{\thefootnote}{\ensuremath{\fnsymbol{footnote%%@
}}}
\numberwithin{equation}{section}

\newcommand{\Ad}{{\rm Ad}}
\newcommand{\Aut}{{\rm Aut}}
\newcommand{\Cl}{{\rm Cl}}
\newcommand{\Co}{{\rm Co}}
\newcommand{\DES}{{\rm DES}}
\newcommand{\Diff}{{\rm Diff}}
\newcommand{\Dom}{{\rm Dom}}
\newcommand{\Hol}{{\rm Hol}}
\newcommand{\Mon}{{\rm Mon}}
\newcommand{\Hom}{{\rm Hom}}
\newcommand{\Ker}{{\rm Ker}}
\newcommand{\Ind}{{\rm Ind}}
\newcommand{\IM}{{\rm Im}}
\newcommand{\Is}{{\rm Is}}
\newcommand{\ID}{{\rm id}}
\newcommand{\GL}{{\rm GL}}
\newcommand{\Iso}{{\rm Iso}}
\newcommand{\Sem}{{\rm Sem}}
\newcommand{\St}{{\rm St}}
\newcommand{\Sym}{{\rm Sym}}
\newcommand{\SU}{{\rm SU}}
\newcommand{\Tor}{{\rm Tor}}
\newcommand{\U}{{\rm U}}

\newcommand{\A}{\mathcal A}
\newcommand{\Ce}{\mathcal C}
\newcommand{\D}{\mathcal D}
\newcommand{\E}{\mathcal E}
\newcommand{\F}{\mathcal F}
\newcommand{\G}{\mathcal G}
\newcommand{\Q}{\mathcal Q}
\newcommand{\R}{\mathcal R}
\newcommand{\cS}{\mathcal S}
\newcommand{\cU}{\mathcal U}
\newcommand{\W}{\mathcal W}

\newcommand{\bA}{\mathbb{A}}
\newcommand{\bB}{\mathbb{B}}
\newcommand{\bC}{\mathbb{C}}
\newcommand{\bD}{\mathbb{D}}
\newcommand{\bE}{\mathbb{E}}
\newcommand{\bF}{\mathbb{F}}
\newcommand{\bG}{\mathbb{G}}
\newcommand{\bK}{\mathbb{K}}
\newcommand{\bM}{\mathbb{M}}
\newcommand{\bN}{\mathbb{N}}
\newcommand{\bO}{\mathbb{O}}
\newcommand{\bP}{\mathbb{P}}
\newcommand{\bR}{\mathbb{R}}
\newcommand{\bV}{\mathbb{V}}
\newcommand{\bZ}{\mathbb{Z}}

\newcommand{\bfE}{\mathbf{E}}
\newcommand{\bfX}{\mathbf{X}}
\newcommand{\bfY}{\mathbf{Y}}
\newcommand{\bfZ}{\mathbf{Z}}

\renewcommand{\O}{\Omega}
\renewcommand{\o}{\omega}
\newcommand{\vp}{\varphi}
\newcommand{\vep}{\varepsilon}

\newcommand{\diag}{{\rm diag}}
\newcommand{\grp}{{\mathbb G}}
\newcommand{\dgrp}{{\mathbb D}}
\newcommand{\desp}{{\mathbb D^{\rm{es}}}}
\newcommand{\Geod}{{\rm Geod}}
\newcommand{\geod}{{\rm geod}}
\newcommand{\hgr}{{\mathbb H}}
\newcommand{\mgr}{{\mathbb M}}
\newcommand{\ob}{{\rm Ob}}
\newcommand{\obg}{{\rm Ob(\mathbb G)}}
\newcommand{\obgp}{{\rm Ob(\mathbb G')}}
\newcommand{\obh}{{\rm Ob(\mathbb H)}}
\newcommand{\Osmooth}{{\Omega^{\infty}(X,*)}}
\newcommand{\ghomotop}{{\rho_2^{\square}}}
\newcommand{\gcalp}{{\mathbb G(\mathcal P)}}

\newcommand{\rf}{{R_{\mathcal F}}}
\newcommand{\glob}{{\rm glob}}
\newcommand{\loc}{{\rm loc}}
\newcommand{\TOP}{{\rm TOP}}

\newcommand{\wti}{\widetilde}
\newcommand{\what}{\widehat}

\renewcommand{\a}{\alpha}
\newcommand{\be}{\beta}
\newcommand{\ga}{\gamma}
\newcommand{\Ga}{\Gamma}
\newcommand{\de}{\delta}
\newcommand{\del}{\partial}
\newcommand{\ka}{\kappa}
\newcommand{\si}{\sigma}
\newcommand{\ta}{\tau}
\newcommand{\med}{\medbreak}
\newcommand{\medn}{\medbreak \noindent}
\newcommand{\bign}{\bigbreak \noindent}
\newcommand{\lra}{{\longrightarrow}}
\newcommand{\ra}{{\rightarrow}}
\newcommand{\rat}{{\rightarrowtail}}
\newcommand{\oset}[1]{\overset {#1}{\ra}}
\newcommand{\osetl}[1]{\overset {#1}{\lra}}
\newcommand{\hr}{{\hookrightarrow}}  
\begin{document}
\subsection{Introduction}

\emph{Hamiltonian algebroids} are generalizations of the Lie algebras of canonical transformations, but cannot be considered just a special case of Lie algebroids. They are instead a special case of a \PMlinkexternal{quantum algebroid}{http://planetphysics.org/encyclopedia/QuantumAlgebroid.html}.

\begin{definition}
  Let $X$ and $Y$ be two vector fields on a smooth manifold $M$, represented here as operators acting on functions.
Their commutator, or Lie bracket, $L$, is :
\begin{align*}
[X,Y](f)=X(Y(f))-Y(X(f)).
\end {align*}

 
  Moreover, consider the classical configuration space $Q = \bR^3$ of a classical, mechanical system, or particle whose phase space is the cotangent bundle $T^* \bR^3 \cong \bR^6$, for which the space of (classical) 
observables is taken to be the real vector space of smooth functions on $M$, and with T being an element
of a \PMlinkname{Jordan-Lie (Poisson) algebra}{JordanBanachAndJordanLieAlgebras} whose definition is also recalled next. Thus, one defines as in classical dynamics the \emph{Poisson algebra} as a Jordan algebra in which $\circ$ is associative. We recall that one needs to consider first a specific algebra (defined as a vector space $E$ over a ground field (typically $\bR$ or $\bC$)) equipped with a bilinear and distributive multiplication $\circ$~. Then one defines a \emph{Jordan algebra} (over $\bR$), as a a specific algebra over $\bR$ for which:

\bigbreak

 $ \begin{aligned} S \circ T &= T \circ S~, \\ S \circ (T \circ S^2) &= (S \circ T) \circ S^2 , 
\end{aligned},$

for all elements $S, T$ of this algebra.

 Then, the usual algebraic types of morphisms automorphism, isomorphism, etc.) apply to a
\PMlinkname{Jordan-Lie (Poisson) algebra}{JordanBanachAndJordanLieAlgebras} defined as a real vector space $U_{\bR}$  together with a \emph{Jordan product} $\circ$ and \emph{Poisson bracket}
\bigbreak
$\{~,~\}$, satisfying~:

\begin{itemize}
\item[1.] for all $S, T \in U_{\bR},$

$\begin{aligned} S \circ T & = T \circ S \\ \{S, T \} &= - \{T,
S\} \end{aligned}$

\item[2.] the \emph{Leibniz rule} holds

$$ \{S, T \circ W \} = \{S, T\} \circ W + T \circ \{S, W\}$$
for all $S, T, W \in U_{\bR}$, along with

\item[3.] the \emph{Jacobi identity}~:

 $$ \{S, \{T, W \}\} = \{\{S,T \}, W\} + \{T, \{S, W \}\}$$

\item[4.] for some $\hslash^2 \in \bR$, there is the \emph{associator identity} ~:
$$(S \circ T) \circ W - S \circ (T \circ W) = \frac{1}{4} \hslash^2 \{\{S, W \}, T \}~.$$
\end{itemize}
 
 Thus, the canonical transformations of the Poisson sigma model phase space specified by the \PMlinkname{Jordan-Lie (Poisson) algebra}{JordanBanachAndJordanLieAlgebras} (also Poisson algebra), which is determined by both the Poisson bracket and the \emph{Jordan product} $\circ$,  define a \emph{Hamiltonian algebroid} with the Lie brackets $L$ related to such a Poisson structure on the target space. 
\end{definition}
%%%%%
%%%%%
\end{document}
