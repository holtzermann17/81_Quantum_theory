\documentclass[12pt]{article}
\usepackage{pmmeta}
\pmcanonicalname{JensensInequality}
\pmcreated{2013-03-22 11:46:30}
\pmmodified{2013-03-22 11:46:30}
\pmowner{Andrea Ambrosio}{7332}
\pmmodifier{Andrea Ambrosio}{7332}
\pmtitle{Jensen's inequality}
\pmrecord{13}{30234}
\pmprivacy{1}
\pmauthor{Andrea Ambrosio}{7332}
\pmtype{Theorem}
\pmcomment{trigger rebuild}
\pmclassification{msc}{81Q30}
\pmclassification{msc}{26D15}
\pmclassification{msc}{39B62}
\pmclassification{msc}{18-00}
%\pmkeywords{Convex}
%\pmkeywords{Concave}
%\pmkeywords{Inequality}
\pmrelated{ConvexFunction}
\pmrelated{ConcaveFunction}
\pmrelated{ArithmeticGeometricMeansInequality}
\pmrelated{ProofOfGeneralMeansInequality}

\endmetadata

\usepackage{graphicx}
%%%%%\usepackage{xypic} 
\usepackage{bbm}
\newcommand{\Z}{\mathbbmss{Z}}
\newcommand{\C}{\mathbbmss{C}}
\newcommand{\R}{\mathbbmss{R}}
\newcommand{\Q}{\mathbbmss{Q}}
\newcommand{\mathbb}[1]{\mathbbmss{#1}}
\newcommand{\figura}[1]{\begin{center}\includegraphics{#1}\end{center}}
\newcommand{\figuraex}[2]{\begin{center}\includegraphics[#2]{#1}\end{center}}
\newcommand{\Expect}{\operatorname{\mathbbmss{E}}}
\begin{document}
If $f$ is a convex function on the interval $[a,b]$, for each $\left\{x_k\right\}_{k=1}^n \in[a,b]$ and each $\left\{\mu_k\right\}_{k=1}^n$ with $\mu_{k}\geq0$ one has:
$$f\left(\frac{\sum_{k=1}^{n}\mu_{k}x_{k}}{\sum_{k}^{n}\mu_{k}}\right)\leq\frac{\sum_{k=1}^{n}\mu_{k}f\left(x_{k}\right)}{\sum_{k}^{n}\mu_{k}}.$$

A common situation occurs when $\mu_1+\mu_2+\cdots+\mu_n=1$; in this case, the inequality simplifies to:

$$f\left(\sum_{k=1}^n \mu_k x_k\right)\leq \sum_{k=1}^n \mu_k f(x_k)$$
where $0\le \mu_k\le 1$.

If $f$ is a concave function, the inequality is reversed.
\medskip

\textbf{Example:}\\
$f(x)=x^2$ is a convex function on $[0,10]$. 
Then 
$$(0.2\cdot4+ 0.5\cdot3+0.3\cdot7)^2 \leq 0.2(4^2) + 0.5(3^2)+0.3(7^2).$$
\bigskip

A very special case of this inequality is when $\mu_k=\frac{1}{n}$ because then
$$f\left(\frac{1}{n}\sum_{k=1}^n x_k\right)\le\frac{1}{n}\sum_{k=1}^n f(x_k)$$
that is, the value of the function at the mean of the $x_k$ is less or equal than the mean of the values of the function at each $x_k$.

There is another formulation of Jensen's inequality used in probability:\\
Let $X$ be some random variable, and let $f(x)$ be a convex function (defined at least on a segment containing the range of $X$).  Then the expected value of $f(X)$ is at least the value of $f$ at the mean of $X$:
\[
\mathrm{E}[f(X)] \ge f(\mathrm{E}[ X]).
\]
With this approach, the weights of the first form can be seen as probabilities.
%%%%%
%%%%%
%%%%%
%%%%%
\end{document}
