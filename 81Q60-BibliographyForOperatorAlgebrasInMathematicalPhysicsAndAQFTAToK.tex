\documentclass[12pt]{article}
\usepackage{pmmeta}
\pmcanonicalname{BibliographyForOperatorAlgebrasInMathematicalPhysicsAndAQFTAToK}
\pmcreated{2013-03-22 18:46:14}
\pmmodified{2013-03-22 18:46:14}
\pmowner{bci1}{20947}
\pmmodifier{bci1}{20947}
\pmtitle{bibliography for operator algebras in mathematical physics and AQFT-A to K}
\pmrecord{10}{41554}
\pmprivacy{1}
\pmauthor{bci1}{20947}
\pmtype{Bibliography}
\pmcomment{trigger rebuild}
\pmclassification{msc}{81Q60}
\pmclassification{msc}{03G12}
\pmclassification{msc}{81R50}
\pmclassification{msc}{81T70}
\pmclassification{msc}{47C15}
\pmclassification{msc}{46L35}
\pmclassification{msc}{46L10}
\pmclassification{msc}{46L05}
\pmclassification{msc}{81T60}
\pmclassification{msc}{81T05}
%\pmkeywords{operator algebras in mathematical physics and AQFT}

% this is the default PlanetMath preamble. as your knowledge
% of TeX increases, you will probably want to edit this, but
\usepackage{amsmath, amssymb, amsfonts, amsthm, amscd, latexsym}
%%\usepackage{xypic}
\usepackage[mathscr]{eucal}
% define commands here
\theoremstyle{plain}
\newtheorem{lemma}{Lemma}[section]
\newtheorem{proposition}{Proposition}[section]
\newtheorem{theorem}{Theorem}[section]
\newtheorem{corollary}{Corollary}[section]
\theoremstyle{definition}
\newtheorem{definition}{Definition}[section]
\newtheorem{example}{Example}[section]
%\theoremstyle{remark}
\newtheorem{remark}{Remark}[section]
\newtheorem*{notation}{Notation}
\newtheorem*{claim}{Claim}
\renewcommand{\thefootnote}{\ensuremath{\fnsymbol{footnote%%@
}}}
\numberwithin{equation}{section}
\newcommand{\Ad}{{\rm Ad}}
\newcommand{\Aut}{{\rm Aut}}
\newcommand{\Cl}{{\rm Cl}}
\newcommand{\Co}{{\rm Co}}
\newcommand{\DES}{{\rm DES}}
\newcommand{\Diff}{{\rm Diff}}
\newcommand{\Dom}{{\rm Dom}}
\newcommand{\Hol}{{\rm Hol}}
\newcommand{\Mon}{{\rm Mon}}
\newcommand{\Hom}{{\rm Hom}}
\newcommand{\Ker}{{\rm Ker}}
\newcommand{\Ind}{{\rm Ind}}
\newcommand{\IM}{{\rm Im}}
\newcommand{\Is}{{\rm Is}}
\newcommand{\ID}{{\rm id}}
\newcommand{\GL}{{\rm GL}}
\newcommand{\Iso}{{\rm Iso}}
\newcommand{\Sem}{{\rm Sem}}
\newcommand{\St}{{\rm St}}
\newcommand{\Sym}{{\rm Sym}}
\newcommand{\SU}{{\rm SU}}
\newcommand{\Tor}{{\rm Tor}}
\newcommand{\U}{{\rm U}}
\newcommand{\A}{\mathcal A}
\newcommand{\Ce}{\mathcal C}
\newcommand{\D}{\mathcal D}
\newcommand{\E}{\mathcal E}
\newcommand{\F}{\mathcal F}
\newcommand{\G}{\mathcal G}
\newcommand{\Q}{\mathcal Q}
\newcommand{\R}{\mathcal R}
\newcommand{\cS}{\mathcal S}
\newcommand{\cU}{\mathcal U}
\newcommand{\W}{\mathcal W}
\newcommand{\bA}{\mathbb{A}}
\newcommand{\bB}{\mathbb{B}}
\newcommand{\bC}{\mathbb{C}}
\newcommand{\bD}{\mathbb{D}}
\newcommand{\bE}{\mathbb{E}}
\newcommand{\bF}{\mathbb{F}}
\newcommand{\bG}{\mathbb{G}}
\newcommand{\bK}{\mathbb{K}}
\newcommand{\bM}{\mathbb{M}}
\newcommand{\bN}{\mathbb{N}}
\newcommand{\bO}{\mathbb{O}}
\newcommand{\bP}{\mathbb{P}}
\newcommand{\bR}{\mathbb{R}}
\newcommand{\bV}{\mathbb{V}}
\newcommand{\bZ}{\mathbb{Z}}
\newcommand{\bfE}{\mathbf{E}}
\newcommand{\bfX}{\mathbf{X}}
\newcommand{\bfY}{\mathbf{Y}}
\newcommand{\bfZ}{\mathbf{Z}}
\renewcommand{\O}{\Omega}
\renewcommand{\o}{\omega}
\newcommand{\vp}{\varphi}
\newcommand{\vep}{\varepsilon}
\newcommand{\diag}{{\rm diag}}
\newcommand{\grp}{{\mathbb G}}
\newcommand{\dgrp}{{\mathbb D}}
\newcommand{\desp}{{\mathbb D^{\rm{es}}}}
\newcommand{\Geod}{{\rm Geod}}
\newcommand{\geod}{{\rm geod}}
\newcommand{\hgr}{{\mathbb H}}
\newcommand{\mgr}{{\mathbb M}}
\newcommand{\ob}{{\rm Ob}}
\newcommand{\obg}{{\rm Ob(\mathbb G)}}
\newcommand{\obgp}{{\rm Ob(\mathbb G')}}
\newcommand{\obh}{{\rm Ob(\mathbb H)}}
\newcommand{\Osmooth}{{\Omega^{\infty}(X,*)}}
\newcommand{\ghomotop}{{\rho_2^{\square}}}
\newcommand{\gcalp}{{\mathbb G(\mathcal P)}}
\newcommand{\rf}{{R_{\mathcal F}}}
\newcommand{\glob}{{\rm glob}}
\newcommand{\loc}{{\rm loc}}
\newcommand{\TOP}{{\rm TOP}}
\newcommand{\wti}{\widetilde}
\newcommand{\what}{\widehat}
\renewcommand{\a}{\alpha}
\newcommand{\be}{\beta}
\newcommand{\ga}{\gamma}
\newcommand{\Ga}{\Gamma}
\newcommand{\de}{\delta}
\newcommand{\del}{\partial}
\newcommand{\ka}{\kappa}
\newcommand{\si}{\sigma}
\newcommand{\ta}{\tau}
\newcommand{\lra}{{\longrightarrow}}
\newcommand{\ra}{{\rightarrow}}
\newcommand{\rat}{{\rightarrowtail}}
\newcommand{\oset}[1]{\overset {#1}{\ra}}
\newcommand{\osetl}[1]{\overset {#1}{\lra}}
\newcommand{\hr}{{\hookrightarrow}}
\begin{document}
\subsection{Bibliography for Operator Algebras in Mathematical Physics and Algebraic Quantum Field Theories (AQFT):}
\emph{Alphabetical order: Letters from A to K}

\begin{thebibliography} {399}

\bibitem{AW}
Akutsu, Y. and Wadati, M. (1987).
Knot invariants and critical statistical systems.
{\em Journal of the Physics Society of Japan},
{\bf 56}, 839--842.

\bibitem{Al}
Alexander, J. W. (1930).
The combinatorial theory of complexes.
{\em Annals of Mathematics}, {\bf (2) 31}, 294--322.

\bibitem{ABF}
Andrews, G. E., Baxter, R. J. and Forrester, P.J. (1984).
Eight vertex SOS model and generalized
Rogers--Ramanujan type identities. {\em Journal of Statistical Physics},  {\bf 35}, 193--266.

\bibitem{AoY}
Aoi, H. and Yamanouchi, T. (in press).
Construction of a canonical subfactor for an inclusion of factors with a common Cartan subalgebra.
{\rm Hokkaido Mathematical Journal}.

\bibitem{AGO}
Arcuri, R. C., Gomes, J. F. and D. I. Olive (1987). Conformal subalgebras and symmetric spaces.
{\em Nuclear Physics B}, {\bf 285}, 327--339.

\bibitem{Art}
Artin, E. (1947). Theory of braids. {\em Annals of Mathematics},
{\bf 48} 101--126.

\bibitem{A2}
Asaeda, M. (2007). Galois groups and an obstruction to principal graphs of subfactors.
{\em International Journal of Mathematics}, {\bf 18}, 191--202.
math.OA/0605318.

\bibitem{AH}
Asaeda, M. and Haagerup, U. (1999). Exotic subfactors of finite depth with Jones indices
${(5+\sqrt{13})}/{2}$ and ${(5+\sqrt{17})}/{2}$. {\em Communications in Mathematical Physics},
{\bf 202}, 1--63.

\bibitem{AY}
Asaeda, M. and Yasuda, S. (preprint 2007). On Haagerup's list of potential principal graphs of subfactors.
arXiv:0711.4144.

\bibitem{At1}
Atiyah, M. (1967). $K$-theory. {\em W. A. Benjamin Inc., New York}.

\bibitem{At2}
Atiyah, M. (1989). Topological quantum field theory. {\em Publication Math\'ematiques IHES},
{\bf 68}, 175--186.

\bibitem{Au}
Aubert, P.-L. (1976). Th\'eorie de Galois pour une $W^*$-alg\`ebre. {\em Commentarii Mathematici Helvetici},
{\bf 39 (51)},  411--433.

\bibitem{BSZ}
Baez, J. C., Segal, I. E. and Zhou, Z. (1992). Introduction to algebraic and constructive quantum
field theory. {\em Princeton University Press}.

\bibitem{BK}
Bakalov, B. and Kirillov, A. Jr. (2001). Lectures on tensor categories and modular functors.
University Lecture Series {\bf 21}, Amer. Math. Soc.

\bibitem{Ban1}
Banica, T. (1997). Le groupe quantique compact libre $U(n)$, {\em Communications in Mathematical Physics}, {\bf 190},
143--172.

\bibitem{Ban2}
Banica, T. (1998). Hopf algebras and subfactors associated to vertex models.
{\em Journal of Functional Analysis}, {\bf 159},  243--266.

\bibitem{Ban3}
Banica, T. (1999). Representations of compact quantum groups and subfactors.
{\em Journal f\"ur die Reine und Angewandte Mathematik}, {\bf 509}, 167--198.

\bibitem{Ban4}
Banica, T. (1999). Fusion rules for representations of compact quantum groups.
{\em Expositiones Mathematicae}, {\bf 17}, 313--337.

\bibitem{Ban5}
Banica, T. (1999). Symmetries of a generic coaction. {\em Mathematische Annalen}, {\bf 314}, 763--780.

\bibitem{Ban6}
Banica, T. (2000). Compact Kac algebras and commuting squares.
{\em Journal of Functional Analysis}, {\bf 176},  80--99.

\bibitem{Ban7}
Banica, T. (2001). Subfactors associated to compact Kac algebras.
{\em Integral Equations Operator Theory}, {\bf 39}, 1--14. 

\bibitem{Ban8}
Banica, T. (2002). Quantum groups and Fuss-Catalan algebras. 
{\em Communications in Mathematical Physics}, {\bf 226}, 221--232

\bibitem{Ban9}
Banica, T. (2005). The planar algebra of a coaction. {\em Journal of Operator Theory} {\bf 53}, 119--158.

\bibitem{Ban10}
Banica, T. (2005). Quantum automorphism groups of homogeneous graphs. 
{\em Journal of Functional Analysis}, {\bf 224}, 243--280.

\bibitem{Ban11}
Banica, T. (2005). Quantum automorphism groups of small metric spaces. 
{\em Pacific Journal of Mathematics}, {\bf 219}, 27--51. 

\bibitem{Ba1}
Baxter, R. J. (1981).Rogers--Ramanujan identities in the Hard Hexagon model. {\em  Journal of Statistical Physics}, {\bf 26},
427--452.

\bibitem{Ba2}
Baxter, R. J. (1982). {\em Exactly solved models in statistical mechanics}.
Academic Press, New York.

\bibitem{Ba4}
Baxter, R. J. (1988). The superintegrable chiral Potts model. {\em Physics Letters A}, {\bf 133}, 185--189.

\bibitem{Ba3}
Baxter, R. J. (1989). A simple solvable  $Z_4(N)$ Hamiltonian.
{\em Physics Letters A}, {\bf 140}, 155--157.

\bibitem{Ba5}
Baxter, R. J. (1989). Superintegrable Chiral Potts model: thermodynamic 
properties,  an ``inverse'' model, and a simple associated Hamiltonian. {\em Journal of Statistical
Physics}, {\bf 57}, 1--39.

\bibitem{BKW}
Baxter, R. J., Kelland, S. B. and Wu, F. Y. (1976). Potts model or Whitney Polynomial.
{\em Journal of Physics. A. Mathematical and General},
{\bf 9}, 397--406.

\bibitem{BPA}
Baxter, R. J., Perk, J. H. H. and  Au-Yang, H.  (1988). New solutions of the star-triangle relations for the chiral Potts model. {\em Physics Letters A} {\bf 128},  138--142.

\bibitem{BTA}
Baxter, R. J., Temperley, H. N. V. and Ashley, S. E. (1978).
Triangular Potts model and its transition temperature and related models.
{\em Proceedings of the Royal Society of London A},
{\bf 358}, 535--559.

\bibitem{BeE}
Behrend, R. E., Evans, D. E. (preprint 2003). Integrable Lattice Models for Conjugate $A^{(1)}_n$.
hep-th/0309068.

\bibitem{BPPZ}
Behrend, R. E., Pearce, P. A., Petkova, V. B. and Zuber, J-B.  (2000).
Boundary conditions in rational conformal field theories.
{\em Nuclear Physics B}, {\bf 579}, 707--773.

\bibitem{BPZ}
Belavin, A. A., Polyakov, A. M. and Zamolodchikov, A. B. (1980). 
Infinite conformal symmetry in two-dimensional quantum field theory.
{\em Nuclear Physics B}, {\bf 241}, 333--380.

\bibitem{Ber}
Berezin, F. A. (1966). A method of second quantization. {\em Academic Press}, London/New York.

\bibitem{BCL}
Bertozzini, P., Conti, R. and Longo, R. (1998) Covariant sectors with infinite dimension and positivity of the energy.
{\em Communications in Mathematical Physics}, {\bf 193}, 471--492.

\bibitem{BiN}
Bion-Nadal, J. (1992).
Subfactor of the hyperfinite $II_1$ factor with 
Coxeter graph $E_6$ as invariant.
{\em Journal of Operator Theory}, {\bf 28}, 27--50.

\bibitem{Bi}
Birman, J. (1974). Braids, links and mapping class groups.
{\em Annals of Mathematical Studies}, {\bf 82}.

\bibitem{BW}
Birman, J. S. and Wenzl, H. (1989).
Braids, link polynomials and a new algebra.
{\em Transactions of the American Mathematical Society}, 
{\bf 313}, 249--273.

\bibitem{Bs1}
Bisch, D. (1990). On the existence of central sequences in subfactors.
{\em Transactions of the American Mathematical Society}, 
{\bf 321}, 117--128.

\bibitem{Bs2}
Bisch, D. (1992). Entropy of groups and subfactors.
{\em Journal of Functional Analysis}, {\bf 103}, 
190--208.

\bibitem{Bs3}
Bisch, D. (1994). A note on intermediate subfactors.
{\em Pacific Journal of Mathematics}, {\bf 163}, 
201--216.

\bibitem{Bs4}
Bisch, D. (1994).
On the structure of finite depth subfactors.
in {\em Algebraic methods in operator theory},
(ed. R. Curto and P. E. T. J\"orgensen), 
Birkh\"auser, 175--194.

\bibitem{Bs5}
Bisch, D. (1994).
Central sequences in subfactors II.
{\em Proceedings of the American Mathematical Society},
{\bf 121}, 725--731.

\bibitem{Bs6}
Bisch, D. (1994).
An example of an irreducible subfactor of the hyperfinite
II$_1$ factor with rational, non-integer index.
{\em Journal f\"ur die Reine und Angewandte
Mathematik}, {\bf 455}, 21--34.

\bibitem{Bs7}
Bisch, D. (1997).
Bimodules, higher relative commutants and the fusion algebra
associated to  a subfactor.
In {\em Operator algebras and their applications}.
Fields Institute Communications,
Vol. 13, American Math. Soc., 13--63.

\bibitem{Bs8}
Bisch, D. (1998).
Principal graphs of subfactors with small Jones index.
{\em Mathematische Annalen}, {\bf 311}, 223--231.

\bibitem{Bs9}
Bisch, D. (2002).
Subfactors and planar algebras.
{\em Proc. ICM-2002, Beijing}, {\bf 2},  775--786.

\bibitem{BH}
Bisch, D. and Haagerup, U. (1996).
Composition of subfactors: New examples of infinite 
depth subfactors.
{\em Annales Scientifiques de l'\'Ecole Normale 
Superieur}, {\bf 29}, 329--383.

\bibitem{BJ}
Bisch, D. and Jones, V. F. R. (1997).
Algebras associated to intermediate subfactors.
{\em Inventiones Mathematicae},
{\bf 128}, 89--157.

\bibitem{BJ2}
Bisch, D. and Jones, V. F. R. (1997).
A note on free composition of subfactors.
In {\em Geometry and Physics, (Aarhus 1995)},
Marcel Dekker, Lecture Notes in Pure
and Applied Mathematics, Vol. 184, 339--361.

\bibitem{BJ3}
Bisch, D. and Jones, V. F. R. (2000).
Singly generated planar algebras of small dimension.
{\em Duke Mathematical Journal}, {\bf 101}, 41--75.

\bibitem{BJ4}
Bisch, D. and Jones, V. F. R. (2003).
Singly generated planar algebras of small dimension. II   
{\em Advances in Mathematics}, {\bf 175}, 297--318.

\bibitem{BNP}
Bisch, D., Nicoara, R. and Popa, S. (2007).
Continuous families of hyperfinite subfactors with 
the same standard invariant.
{\em International Journal of Mathematics}, {\bf 18}, 255--267.
math.OA/0604460.

\bibitem{BP}
Bisch, D. and Popa, S. (1999).
Examples of subfactors with property T standard invariant.
{\em Geometric and Functional Analysis}, {\bf 9}, 215--225.

\bibitem{Bk}
B\"ockenhauer, J. (1996).
An algebraic formulation of level one Wess-Zumino-Witten models.
{\em Reviews in Mathematical Physics}, {\bf 8}, 925--947.

\bibitem{BE}
B\"ockenhauer, J. and Evans, D. E. (1998).
Modular invariants, graphs and $\alpha$-induction for
nets of subfactors I.
{\em Communications in Mathematical Physics}, {\bf 197}, 361--386.

\bibitem{BE2}
B\"ockenhauer, J. and Evans, D. E. (1999).
Modular invariants, graphs and $\alpha$-induction for
nets of subfactors II.
{\em Communications in Mathematical Physics}, {\bf 200}, 57--103.

\bibitem{BE3}
B\"ockenhauer, J. and Evans, D. E. (1999).
Modular invariants, graphs and $\alpha$-induction for
nets of subfactors III.
{\em Communications in Mathematical Physics}, {\bf 205}, 183--228.

\bibitem{BE4}
B\"ockenhauer, J. and Evans, D. E. (2000).
Modular invariants from subfactors: Type I coupling matrices and
intermediate subfactors.
{\em Communications in Mathematical Physics}, {\bf 213}, 267--289.

\bibitem{BE5}
B\"ockenhauer, J. and Evans, D. E. (2002).
Modular invariants from subfactors.
in {\em Quantum Symmetries in Theoretical Physics and Mathematics}
(ed. R. Coquereaux et al.),
Comtemp. Math. {\bf 294}, Amer. Math. Soc., 95--131.
math.OA/0006114.

\bibitem{BE6}
B\"ockenhauer, J. and Evans, D. E. (2001).
Modular invariants and subfactors.
in {\em Mathematical Physics in Mathematics and Physics} (ed. R. Longo),
The Fields Institute Communications {\bf 30}, Providence, Rhode Island:
AMS Publications, 11--37.
math.OA/0008056.

\bibitem{BEK}
B\"ockenhauer, J., Evans, D. E. and Kawahigashi, Y. (1999).
On $\alpha$-induction, chiral generators
and modular invariants for subfactors.
{\em Communications in Mathematical Physics}, {\bf 208}, 429--487.
math.OA/9904109.

\bibitem{BEK2}
B\"ockenhauer, J., Evans, D. E. and Kawahigashi, Y. (2000).
Chiral structure of modular invariants for subfactors.
{\em Communications in Mathematical Physics}, {\bf 210}, 733--784.
math.OA/9907149.

\bibitem{BEK3}
B\"ockenhauer, J., Evans, D. E. and Kawahigashi, Y. (2001).
Longo-Rehren subfactors arising from $\alpha$-induction.
{\em Publications of the RIMS, Kyoto University}, {\bf 37}, 1--35.
math.OA/0002154.

\bibitem{BG}
de Boer, J. and Goeree, J. (1991).
Markov traces and II$_1$ factors in
conformal field theory.
{\em Communications in Mathematical Physics},
{\bf 139}, 267--304.

\bibitem{Bon}
Bongaarts, P. J. M. (1970).
The electron-positron field, coupled to external 
electromagnetic potentials as an elementary
$C^*$-algebra theory. {\em Annals of Physics},
{\bf 56}, 108--138.

\bibitem{Bra}
Bratteli, O. (1972).
Inductive limits of finite dimensional $C^*$-algebras.
{\em Transactions of the American Mathematical Society},
{\bf 171}, 195--234.

\bibitem{BGL1}
Brunetti, R., Guido, D. and Longo, R. (1993).
Modular structure and duality in conformal
quantum field theory. 
{\em Communications in Mathematical Physics},  {\bf 156}, 201--219.

\bibitem{BGL2}
Brunetti, R., Guido, D. and Longo, R. (1995).
Group cohomology, modular theory and space-time symmetries. 
{\em Reviews in Mathematical Physics}, {\bf 7} 57--71.

\bibitem{BDLR}
Buchholz, D., Doplicher, S., Longo, R. and Roberts, J. E. (1993).
Extensions of automorphisms and gauge symmetries.
{\em Communications in Mathematical Physics},
{\bf 155}, 123--134.

\bibitem{BMT}
Buchholz, D., Mack, G. and Todorov, I. (1988).
The current algebra on the circle as a germ of local field theories.
{\em Nuclear  Physics B (Proc. Suppl.)}, {\bf B5}, 20--56.

\bibitem{BS}
Buchholz, D. and Schulz-Mirbach, H. (1990).
Haag duality in conformal quantum field theoery, 
{\em Reviews in Mathematical Physics}, {\bf 2} 105--125.

\bibitem{CN}
Camp, W., and Nicoara, R. (preprint 2007).
Subfactors and Hadamard matrices.
arXiv:0704.1128.

\bibitem{CIZ}
Cappelli, A., Itzykson, C. and Zuber, J.-B. (1987).
The $A$-$D$-$E$ classification of minimal and
$A^{(1)}_1$ conformal invariant theories.
{\em Communications in Mathematical Physics},  {\bf 113}, 1--26.

\bibitem{Ca}
Carpi, S. (1998).
Absence of subsystems for the Haag-Kastler net generated by 
the energy-momentum tensor in two-dimensional conformal field theory.
{\em Letters in Mathematical Physics}, {\bf 45}, 259--267.

\bibitem{Ca2}
Carpi, S. (2003).
The Virasoro algebra and sectors with infinite statistical dimension.
{\em Annales Henri Poincar\'e}, {\bf 4}, 601--611.
math.OA/0203027.

\bibitem{Ca3}
Carpi, S. (2004).
On the representation theory of Virasoro nets.
{\em Communications in Mathematical Physics}, {\bf 244}, 261--284.
math.OA/0306425.

\bibitem{Ca4}
Carpi, S. (2005).
Intersecting Jones projections.
{\em International Journal of Mathematics}, {\bf 16}, 687--691.
math.OA/0412457.

\bibitem{CC}
Carpi, S. and Conti, R. (2001).
Classification of subsystems for local nets with trivial
superselection structure.
{\em Communications in Mathematical Physics},  {\bf 217}, 89--106.

\bibitem{CC2}
Carpi, S. and Conti, R. (2005).
Classification of subsystems for graded-local nets with trivial 
superselection structure.
{\em Communications in Mathematical Physics}.
{\bf 253}, 423--449.
math.OA/0312033.

\bibitem{CKL}
Carpi, S., Kawahigashi, Y. and Longo, R. (in press).
Structure and classification of superconformal nets.
{\em Annales Henri Poincar\'e}.
arXiv:0705.3609.

\bibitem{CW}
Carpi, S. and Weiner, M. (2005).
On the uniqueness of diffeomorphism symmetry in Conformal Field Theory.
{\em Communications in Mathematical Physics},
{\bf 258}, 203--221.
math.OA/0407190.

\bibitem{Ce}
Ceccherini, T. (1996).
Approximately inner and centrally free commuting squares
of type $II_1$ factors and their classification.
{\em Journal of Functioanl Analysis}, {\bf 142}, 296--336.

\bibitem{Chen}
Chen, J. (1993).
The Connes invariant $\chi(M)$ and cohomology of groups. 
Ph. D. dissertation at University of California, Berkeley.

\bibitem{Ch1}
Choda, M. (1989).
Index for factors generated by Jones' two sided 
sequence of projections. 
{\em Pacific Journal of Mathematics}, {\bf 139}, 1--16.

\bibitem{Ch2}
Choda, M. (1991).
Entropy for $*$-endomorphisms and relative entropy 
for subalgebras. {\em Journal of Operator Theory},
{\bf 25}, 125--140.

\bibitem{Ch3}
Choda, M. (1992).
Entropy for canonical shift.
{\em Transactions of the American Mathematical Society}, 
{\bf 334}, 827--849.

\bibitem{Ch4}
Choda, M. (1993).
Duality for finite bipartite graphs
(with applications to II$_1$ factors).
{\em Pacific Journal of Mathematics}, {\bf 158}, 49--65.

\bibitem{Ch5}
Choda, M. (1994).
Square roots of the canonical shifts.
{\em Journal of Operator Theory}, {\bf 31}, 145--163.

\bibitem{Ch6}
Choda, M. (1994).
Extension algebras via $*$-endomorphisms.
in {\em Subfactors ---
Proceedings of the Taniguchi Symposium, Katata ---},
(ed. H. Araki, et al.),
World Scientific, 105--128.

\bibitem{CH}
Choda, M. and Hiai, F. (1991).
Entropy for canonical shifts. II.
{\em Publications of the RIMS, Kyoto University},
{\bf 27}, 461--489.							

\bibitem{CK}
Choda, M. and Kosaki, H. (1994).
Strongly outer actions for an inclusion of factors.
{\em Journal of Functional Analysis}, {\bf 122}, 
315--332.

\bibitem{Chr}
Christensen, E. (1979).
Subalgebras of a finite algebra.
{\em Mathematische Annalen}, {\bf 243}, 17--29.

\bibitem{Com}
Combes, F. (1968).
Poids sur une $C^*$-alg\`ebre.
{\em Journal de Math\'ematiques Pures et 
Appliqu\'ees}, {\bf 47}, 57--100.

\bibitem{C1}
Connes, A. (1973).
Une classification des facteurs de type III.
{\em Annales Scientifiques de l'\'Ecole Normale Sup\'erieure},
{\bf 6}, 133--252.

\bibitem{C2}
Connes, A. (1975).
Outer conjugacy classes of automorphisms of factors.
{\em Annales Scientifiques de l'\'Ecole Normale Sup\'erieure},
{\bf 8}, 383--419.

\bibitem{C3}
Connes, A. (1975).
Hyperfinite factors of type III-0 and Krieger's factors. 
{\em Journal of Functional Analysis},
{\bf 18}, 318--327.

\bibitem{C4}
Connes, A. (1975).
Sur la classification des facteurs de type II.
{\em Comptes Rendus de l'Academie des Sciences, 
S\'erie I, Math\'ematiques}, {\bf 281},  13--15.

\bibitem{C5}
Connes, A. (1975).
A factor not antiisomorphic to itself.
{\em Annals of Mathematics}, {\bf 101},  536--554.

\bibitem{C6}
Connes, A. (1976).
Classification of injective factors.
{\em Annals of Mathematics},
{\bf 104}, 73--115.

\bibitem{C7}
Connes, A. (1976).
Outer conjugacy of automorphisms of factors.
{\em Symposia Mathematica}, {\bf XX}, 149--160.

\bibitem{C8}
Connes, A. (1976).
On the classification of von Neumann
algebras and their automorphisms.
{\em Symposia Mathematica}, {\bf XX}, 435--478.

\bibitem{C9}
Connes, A. (1977).
Periodic automorphisms of the hyperfinite factor of type II$_1$. 
{\em Acta Scientiarum Mathematicarum},
{\bf 39},  39--66.

\bibitem{C10}
Connes, A. (1978).
On the cohomology of operator algebras.
{\em Journal of Functional Analysis},
{\bf 28}, 248--253.

\bibitem{C11}
Connes, A. (1979).
Sur la th\'eorie non commutative de l'integration.
{\em Springer Lecture Notes in Math.},
{\bf 725}, 19--143.

\bibitem{C12}
Connes, A. (1980).
$C^*$-algebres et geom\`etrie diff\'erentielle.
{\em Comptes Rendus de l'Academie des Sciences,
S\'erie I, Math\'ematiques},
559--604.

\bibitem{C13}
Connes, A. (1980).
Spatial theory of von Neumann algebras.
{\em Journal of Functional Analysis}, {\bf 35} 
(1980), 153--164.

\bibitem{C14}
Connes, A. (1981).
An analogue of the Thom isomorphism for crossed 
products of a $C^*$-algebra by an action of
${\bf R}$. {\em Advances in Mathematics},
{\bf 39}, 311--355.

\bibitem{C15}
Connes, A. (1982).
Foliations and Operator Algebras.
{\em Proceedings of Symposia in Pure Mathematics. 
ed.  R. V. Kadison},
{\bf 38}, 521--628.

\bibitem{C16}
Connes, A. (1982).
Classification des facteurs.
{\em Proceedings of the Symposia in Pure Mathematics (II)},
{\bf 38}, 43--109.

\bibitem{C17}
Connes, A. (1985).
Non-commutative differential geometry I--II.
{\em Publication Math\'ematiques IHES},
{\bf 62}, 41--144.

\bibitem{C18}
Connes, A. (1985).
Factors of type III-1, property $L'_\lambda$ and
closure of inner automorphisms.
{\em Journal of Operator Theory}, {\bf 14}, 189--211.

\bibitem{C19}
Connes, A. (1985).
Non Commutative Differential Geometry,
Chapter II: De Rham homology and non commutative 
algebra. {\em Publication Math\'ematiques IHES},
{\bf 62}, 257--360.

\bibitem{C20}
Connes, A. (1994).
Noncommutative geometry.
{\em Academic Press}.

\bibitem{CE}
Connes, A. and Evans, D. E. (1989).
Embeddings of $U(1)$-current algebras in 
non-commutative algebras of classical statistical
mechanics. {\em Communications in Mathematical
Physics}, {\bf 121}, 507--525.

\bibitem{CoH}
Connes, A. and Higson, N. (1990).
D\'eformations, morphismes asymptotiques et
$K$-th\'eorie bivariante.
{\em Comptes Rendus de l' Academie des Sciences, 
S\'erie I, Math\'ematiques},
{\bf 311}, 101--106.

\bibitem{CKa}
Connes, A. and Karoubi, M.  (1988).
Caractere multiplicatif d'un module de Fredholm.
{\em $K$-theory}, {\bf 2} 431--463.

\bibitem{CKr}
Connes, A. and Krieger, W. (1977).
Measure space automorphism groups, the normalizer of their 
full groups, and approximate finiteness. 
{\em Journal of Functional Analysis}, {\bf 24}, 336--352.

\bibitem{CoR}
Connes, A. and Rieffel, M. (1985).
Yang-Mills for non-commutative tori.
{\em Contemporary Mathematics},
{\bf 62}, 237--265.

\bibitem{CoS}
Connes, A. and Skandalis, G. (1984).
The longitudinal index theorem for foliations.
{\em Publications of the  RIMS, Kyoto University},
{\bf 20}, 1139--1183.

\bibitem{CS}
Connes, A. and St\"ormer, E. (1975).
Entropy for automorphisms of $II_1$ von Neumann algebras.
{\em Acta Mathematica}, {\bf 134}, 289--306.

\bibitem{CT}
Connes, A. and Takesaki, M. (1977).
The flow of weights on factors of type III.
{\em Tohoku Mathematical Journal}, {\bf 29}, 73--555.

\bibitem{CDR}
Conti, R., Doplicher, S., and Roberts, J. E. (2001).
Superselection theory for subsystems.
{\em Communications in Mathematical Physics},
{\bf 218}, 263--281.

\bibitem{CP}
Conti, R. and Pinzari, C. (1996).
Remarks on the index of endomorphisms of Cuntz algebras. 
{\em Journal of Functional Analysis}, {\bf 142},  369--405. 

\bibitem{Cq}
Coquereaux, R. (2005)
The $A_2$ Ocneanu quantum groupoid.
in {\em Algebraic structures and their representations}, 
{\em Contemporary Mathematics}, {\bf 376}, 227--247.
hep-th/0311151.

\bibitem{CqS1}
Coquereaux, R. and Schieber, G. (2002).
Twisted partition functions for ADE boundary conformal
field theories and Ocneanu algebras of quantum symmetries.
{\em Journal of Geometry and Physics}, {\bf 42}, 216--258.

\bibitem{CqS2}
Coquereaux, R. and Schieber, G. (2003).
Determination of quantum symmetries for higher ADE systems
from the modular T matrix.
{\em Journal of Mathematical Physics}, {\bf 44}, 3809--3837.
hep-th/0203242.

\ bibitem{Cn}
Cuntz, J. (1977).
Simple $C^*$-algebras generated by isometries.
{\em Communications in Mathematical Physics},
{\bf 57}, 173--185.

\bibitem{Cun2}
Cuntz, J. (1981).
$K$-theory for certain $C^*$-algebras.
{\em Annals of Mathematics},
{\bf 113}, 181--197.

\bibitem{Cun4}
Cuntz, J. (1984).
$K$-theory and $C^*$-algebras.
{\em Lecture Notes in Mathematics, Springer-Verlag},
{\bf 1046}.

\bibitem{Cun5}
Cuntz, J. (1981).
A class of $C^*$-algebras and topological Markov chains II.
Reducible Markov chains and the $Ext$ functor for
$C^*$-algebras.
{\em Inventiones Mathematica},
{\bf 63}, 25--40.

\bibitem{CuntzK}
Cuntz, J. and Krieger, W. (1980).
A class of $C^*$-algebras and topological Markov chains.
{\em Inventiones Mathematicae},
{\bf 56}, 251--268.

\bibitem{CDG}
Cvetkovi\'c, D., Doob, M. and Gutman, I. (1982).
On graphs whose spectral radius does not exceed $(2+\sqrt5)^{1/2}$.
{\em Ars Combinatoria}, {\bf 14}, 225--239.

\bibitem{DFK}
D'Antoni, C., Fredenhagen, K. and K\"oster, S. (preprint 2003).
Implementation of conformal covariance by diffeomorphism symmetry.
math-ph/0312017.

\bibitem{DLR}
D'Antoni, C., Longo, R. and Radulescu, F. (2001).
Conformal nets, maximal temperature and models from free probability.
{\em Journal of Operator Theory}, {\bf 45}, 195--208.

\bibitem{DJKMO}
Date, E., Jimbo, M., Kuniba, A., Miwa, T. and Okado, M. (1988). 
Exactly solvable SOS models II:
Proof of the star-triangle relation and combinatorial identities.
{\em Advanced Studies in Pure Mathematics}, {\bf 16}, 17--122.

\bibitem{DJMO}
Date, E., Jimbo, M., Miwa, T. and Okado, M. (1987).
Solvable lattice models.
{\em Theta functions --- Bowdoin 1987, Part 1},
Proceedings of Symposia in Pure
Mathematics Vol. 49, American Mathematical Society,
Providence, R.I., pp. 295--332.

\bibitem{David}
David, M. C.  (1996).
Paragroupe d'Adrian Ocneanu et alg\`ebre de Kac.
{\em Pacific Journal of Mathematics}, {\bf 172}, 331--363.

\bibitem{Degi}
Degiovanni, P. (1990).
${\bf Z}/N{\bf Z}$ conformal field theories.
{\em Communications in Mathematical Physics},
{\bf 127}, 71--99.

\bibitem{Degi2}
Degiovanni, P. (1992).
Moore and Seiberg's equations and 3D toplogical field theory.
{\bf 145}, 459--505.

\bibitem{DiF}
Di Francesco, P. (1992).
Integrable lattice models, graphs, and modular 
invariant conformal field theories.
{\em International Journal of Modern Physics A},
{\bf 7}, 407--500.

\bibitem{DMS}
Di Francesco, P., Mathieu, P. and S\'en\'echal, D. (1996).
Conformal Field Theory.
Springer-Verlag, New York.

\bibitem{DSZ}
Di Francesco, P., Saleur, H. and Zuber, J.-B. (1987).
Modular invariance in non-minimal two-dimensional conformal field
theories.
{\em Nuclear Physics}, {\bf B285}, 454--480.

\bibitem{DZ1}
Di Francesco, P. and Zuber, J.-B. (1990).
$SU(N)$ lattice integrable models associated with graphs.
{\em Nuclear Physics B}, {\bf 338}, 602--646.

\bibitem{DZ2}
Di Francesco, P. and Zuber, J.-B. (1990).
$SU(N)$ lattice integrable models and modular invariance.
in {\em Recent Developments in Conformal Field Theories, Trieste, 1989}, 
World Scientific, 179--215.

\bibitem{DiPaRo1}
Dijkgraaf, R., Pasquier, V. and Roche, Ph. (1990).
Quasi Hopf algebras, group cohomology and orbifold models. 
{\em Nuclear Physics B(Proc. Suppl.)}, {\bf 18}, 60--72.

\bibitem{DiPaRo2}
Dijkgraaf, R., Pasquier, V. and Roche, Ph. (1991).
Quasi-quantum groups related to orbifold models.
{\em Proceedings of the International Colloquium on
Modern Quantum Field Theory}, World Scientific, 
Singapore, 375--383.

\bibitem{DVVV}
Dijkgraaf, R., Vafa, C., Verlinde, E. and Verlinde, 
H. (1989). The operator algebra of orbifold models.
{\em Communications in Mathematical Physics},  {\bf 123}, 485--526.

\bibitem{DW}
Dijkgraaf, R. and Witten, E. (1990).
Topological gauge theories and group cohomology.
{\em Communications in Mathematical Physics},  {\bf 129}, 393--429.

\bibitem{Di1}
Dixmier, J. (1964).
Les $C^*$-algebras et leurs repr\'esentations.
{\em Gauthier-Villars}.

\bibitem{Di2}
Dixmier, J. (1967).
On some $C^*$-algebras considered by Glimm.
{\em Journal of Functional Analysis}, {\bf 1}, 182--203.

\bibitem{Di3}
Dixmier, J.  (1969).
{\em Les alg\`ebres d'op\'erateurs dans l'espace 
Hilbertien.   (Alg\`ebres de von Neumann.)}  2nd ed.
Gauthier Villars,  Paris.

\bibitem{Dix5}
Dixmier, J. (1981).
Von Neumann Algebras.
{\em North-Holland}.

\bibitem{DiL}
Dixmier, J. and C. Lance (1969).
Deux nouveaux facteurs de type II.
{\em Inventiones Mathematicae},
{\bf 7}, 226--234.

\bibitem{DHVW1}
Dixon, L., Harvey, J. A., Vafa, C. and Witten, E. 
(1985). Strings on orbifolds.
{\em Nuclear Physics B}, {\bf 261},
678--686.

\bibitem{DHVW2}
Dixon, L., Harvey, J. A., Vafa, C. and Witten, E. (1986).
Strings on orbifolds.
{\em Nuclear Physics B}, {\bf 274}, 285--314.

\bibitem{DX}
Dong, C. and Xu, F. (2006).
Conformal nets associated with lattices and their orbifolds.
{\em Advances in Mathematics}, {\bf 206}, 279--306.
math.OA/0411499.

\bibitem{DHR1}
Doplicher, S., Haag, R. and Roberts, J. E. (1969).
Fields, observables and gauge transformations II.
{\em Communications in Mathematical Physics},
{\bf 15}, 173--200.

\bibitem{DHR2}
Doplicher, S., Haag, R. and Roberts, J. E. (1971, 74).
Local obsevables and particle statistics, I,II.
{\em Communications in Mathematical Physics},
{\bf 23}, 199--230 and {\bf 35}, 49--85.

\bibitem{DL}
Doplicher, S. and Longo, R. (1984).
Standard and split inclusions of von Neumann algebras.
{\em Inventiones Mathematicae}, {\bf 75}, 493--536.

bibitem{DP}
Doplicher, S. and , Piacitelli, G. (preprint 2002).
Any compact group is a gauge group.
hep-th/0204230.

\bibitem{DPR}
Doplicher, S., Pinzari, C. and Roberts, J. E. (2001).
An algebraic duality theory for multiplicative unitaries. 
{\em International Journal of Mathematics}, {\bf 12}, 415--459.

\bibitem{DR1}
Doplicher, S. and Roberts, J. E. (1989).
Endomorphisms of $C\sp *$-algebras, cross products
and duality for compact groups. 
{\em Annals of Mathematics}, {\bf 130}, 75--119.

\bibitem{DR2}
Doplicher, S. and Roberts, J. E. (1989).
A new duality theory for compact groups.
{\em Inventiones Mathematica}, {\bf 98}, 157--218.

\bibitem{D}
Drinfel$'$d, V. G. (1986).  Quantum groups.
{\em Proc. ICM-86, Berkeley}, 798--820.

\bibitem{DuS}
Dunford, N. and Schwartz, J. T. (1958).
Linear Operators Volume I. Interscience.

\bibitem{DJN}
Durhuus, B., Jakobsen, H. P. and Nest, R. (1993).
Topological quantum field theories from generalized $6j$-symbols. 
{\em Reviews in Mathematical Physics}, {\bf 5}, 1--67.

\bibitem{El1}
Elliott, G. A. (1976).
On the classification of inductive limits of 
sequences of semisimple finite-dimensional algebras.
{\em Journal of Algebra}, {\bf 38}, 29--44.

\bibitem{En}
Enock, M. (1998).
Inclusions irr\'educibles de facteurs et unitaires multiplicatifs, II.
{\em Journal of Functional Analysis}, {\bf 154}, 67--109.

\bibitem{En2}
Enock, M. (1999).
Sous-facteurs interm\'ediaires et groupes quantiques mesur\'es.
{\em Journal of Operator Theory}, {\bf 42}, 305--330.

\bibitem{En3}
Enock, M. (2000).
Inclusions of von Neumann algebras and quantum groupoids, II.
{\em Journal of Functional Analysis}, {\bf 178}, 156--225.

\bibitem{EN}
Enock, M. and Nest, R. (1996).
Irreducible inclusions of factors  multiplicative unitaries,
and Kac algebras.
{\em Journal of Functional Analysis}, {\bf 137}, 466--543.

\bibitem{EV}
Enock, M. and Vallin, J.-M. (2000).
Inclusions of von Neumann algebras and quantum groupoids.
{\em Journal of Functional Analysis}, {\bf 172}, 249--300.

\bibitem{Erl}
Erlijman, J. (1998).
New braided subfactors from braid group representations.
{\em Transactions of the American Mathematical Society}, 
{\bf 350}, 185--211.

\bibitem{Erl2}
Erlijman, J. (2000).
Two-sided braid subfactors and asymptotic inclusions.
{\em Pacific Journal of Mathematics}, {\bf 193}, 57--78.

\bibitem{Erl3}
Erlijman, J. (2001).
Multi-sided braid subfactors.
{\em Canadian Journal of Mathematics}, {\bf 53}, 546--564.

\bibitem{E1}
Evans, D.E. (1984).
The $C^*$-algebras of topological Markov chains.
{\em Tokyo Metropolitan University Lecture Notes}.

\bibitem{E2}
Evans, D. E. (1985).
The $C^*$-algebras of the two-dimensional Ising model.
Springer Lecture Notes in Mathematics, {\bf 1136}, 162--176.

\bibitem{E3}
Evans, D. E. (1985).
Quasi-product states on $C^*$-algebras.
{\em Operator algebras and their connections with 
topology and ergodic theory}, 
Springer Lecture Notes in Mathematics, {\bf 1132}, 129--151.

\bibitem{E4}
Evans, D. E. (1990).
$C^*$-algebraic methods in
statistical mechanics and field theory.
{\em International Journal of Modern Physics B},
{\bf 4}, 1069--1118.

\bibitem{E5}
Evans, D. E. (2002).
Fusion rules of modular invariants.
{\em Reviews in Mathematical Physics}, {\bf 14}, 709--731.
math.OA/0204278

\bibitem{E6}
Evans, D. E. (preprint 2002).
Critical phenomena, modular invariants and operator algebras.
math.OA/0204281.

\bibitem{EG0}
Evans, D. E. and Gould, J. D. (1989).
Dimension groups, embeddings and presentations of  
AF algebras associated to solvable lattice models.
{\em Modern Physics Letters A}, {\bf 20}, 1883--1890.

\bibitem{EG1}
Evans, D. E. and Gould, J. D. (1994).
Dimension groups and embeddings of graph algebras.
{\em International Journal of Mathematics},
{\bf 5}, 291--327.

\bibitem{EG2}
Evans, D. E. and Gould, J. D. (1994).
Presentations of AF algebras associated to $T$-graphs.
{\em Publications of the RIMS, Kyoto University}, {\bf 30}, 767--798.

\bibitem{EK1}
Evans, D. E. and Kawahigashi, Y. (1993).
Subfactors and conformal field theory.
in ``Quantum and non-commutative analysis'', 
341--369, Kluwer Academic.

\bibitem{EK2}
Evans, D. E. and Kawahigashi, Y. (1994).
Orbifold subfactors from Hecke algebras.
{\em Communications in Mathematical Physics},
{\bf 165}, 445--484.

\bibitem{EK3}
Evans, D. E. and Kawahigashi, Y. (1994).
The $E_7$ commuting squares produce
$D_{10}$ as principal graph.
{\em Publications of the RIMS, Kyoto University},
{\bf 30}, 151--166.

\bibitem{EK4}
Evans, D. E. and Kawahigashi, Y. (1995).
On Ocneanu's theory of asymptotic inclusions for 
subfactors, topological quantum field theories
and quantum doubles.
{\em International Journal of Mathematics},
{\bf 6}, 205--228.

\bibitem{EK5}
Evans, D. E. and Kawahigashi, Y. (1995).
From subfactors to $3$-dimensional topological
quantum field theories and back --- a detailed 
account of Ocneanu's theory ---.
{\em International Journal of Mathematics},
{\bf 6}, 537--558.

\bibitem{EK6}
Evans, D. E. and Kawahigashi, Y. (1998).
Orbifold subfactors from Hecke algebras II
---Quantum doubles and braiding---.
{\em Communications in Mathematical Physics},
{\bf 196}, 331--361.

\bibitem{EK7}
Evans, D. E. and Kawahigashi, Y. (1998).
Quantum symmetries on operator algebras.
{\em Oxford University Press}.

\bibitem{EP}
Evans, D. E. and Pinto, P. R. (2003).
Subfactor realisation of modular invariants.
{\em Communications in Mathematical Physics},
{\bf 237}, 309--363.
math.OA/0309174.

\bibitem{EP2}
Evans, D. E. and Pinto, P. R. (preprint 2003).
Modular invariants and their fusion rules.
math.OA/0309175

\bibitem{EP3}
Evans, D. E. and Pinto, P. R. (preprint 2004).
Modular invariants and the double of the Haagerup subfactor.

\bibitem{FackM1}
Fack, T. and Mar\`echal, O (1979).
Sur la classification des sym\`etries des 
$C^*$-alg\'ebres UHF.
{\em Canadian Journal of Mathematics},
{\bf 31}, 496--523.

\bibitem{FackM2}
Fack, T. and Mar\`echal, O. (1981).
Sur la classification des automorphisms 
p\`eriodiques des
$C^*$-alg\'ebres UHF.
{\em Journal of Functional Analysis},
{\bf 40}, 267--301.

\bibitem{FFK}
Felder, F., Fr\"ohlich, J. and Keller, G. (1990).
On the structure of unitary conformal field theory. 
II. Representation-theoretic approach.
{\em Communications in  Mathematical Physics}, 
{\bf 130},  1--49.

\bibitem{Fn}
Fendley, P. (1989).
New exactly solvable orbifold models.
{\em Journal of Physics A}, {\bf 22}, 4633--4642.

\bibitem{FendG}
Fendley, P. and Ginsparg, P. (1989).
Non-critical orbifolds.
{\em Nuclear Physics B},
{\bf 324}, 549--580.

\bibitem{FR}
Fenn, R. and Rourke, C. (1979).
On Kirby's calculus of links.
{\em Topology}, {\bf 18}, 1--15.

\bibitem{FI}
Fidaleo, F. and Isola, T. (preprint 1996).
The canonical endomorphism for infinite index inclusions.

\bibitem{Fre}
Fredenhagen, K. (1994).
Superselection sectors with infinite statistical dimension.
in {\em Subfactors ---
Proceedings of the Taniguchi Symposium, Katata ---},
(ed. H. Araki, et al.),
World Scientific, 242--258.

\bibitem{FJ}
Fredenhagen, K. and J\"or\ss, M. (1996).
Conformal Haag-Kastler nets, pointlike localized fields and
the existence of operator product expansion.
{\em Communications in Mathematical Physics},
{\bf 176} (1996) 541--554.

\bibitem{FRS}
Fredenhagen, K., Rehren, K.-H. and Schroer, B. (1989).
Superselection sectors with braid group statistics 
and exchange algebras.
{\em Communications in Mathematical Physics},
{\bf 125}, 201--226.

\bibitem{FRS2}
Fredenhagen, K., Rehren, K.-H. and Schroer, B. (1992).
Superselection sectors with braid group statistics
and exchange algebras II.  
{\em Reviews in Mathematical Physics} (special issue), 113--157.

\bibitem{HOMFLY}
Freyd, P., Yetter, D.; Hoste, J.; Lickorish, W., 
Millet, K.; and Ocneanu, A. (1985).
A new polynomial invariant of knots and links.
{\em Bulletin of the American Mathematical Society},
{\bf 12}, 239--246.

\bibitem{FQS}
Friedan, D., Qui, Z. and Shenker, S. (1984).
Conformal invariance, unitarity and critical 
exponents in two dimensions.
{\em Physical Review Letters},
{\bf 52}, 1575--1578.

\bibitem{Fro1}
Fr\"ohlich, J. (1987).
Statistics of fields, the Yang--Baxter equation and 
the theroy of knots and links.
{\em Proceedings of Carg\`ese, ed. G.'t Hooft et al}.

\bibitem{Fr}
Fr\"ohlich, J. (1988).
The Statistics of Fields, the Yang--Baxter Equation,
and the Theory of Knots and Links.
In ``Non-Pert. Quantum Field
Theory'', G. 'tHooft et al. (ed.), Plenum. 
(Cargese Lectures, 1987).

\bibitem{FG}
Fr\"ohlich, J. and Gabbiani, F. (1990).
Braid statistics in local quantum theory.
{\em Reviews in  Mathematical Physics}, {\bf 2},  
251--353.

\bibitem{FK}
Fr\"ohlich, J. and Kerler, T. (1993).
Quantum groups, quantum categories and quantum 
field theory. Lecture Notes in Mathematics, 
{\bf 1542}, Springer.

\bibitem{FKn1}
Fr\"ohlich, J. and King, C. (1989).
The Chern--Simons theory and knot polynomials.
{\em Communications in Mathematical Physics},  {\bf 126}, 167--199.

\bibitem{FKn2}
Fr\"ohlich, J. and King, C. (1989).
Two-dimensional conformal field theory and 
three-dimensional topology. 
{\em International Journal of Modern Physics A}, 
{\bf 4}, 5321--5399.

\bibitem{FM1}
Fr\"ohlich, J. and Marchetti, P.-A. (1988).
Quantum field theory of anyons.
{\em Letters in Mathematical Physics}, {\bf 16}, 
347--358.

\bibitem{FM2}
Fr\"ohlich, J. and Marchetti, P.-A. (1989).
Quantum field theories of vortices and anyons.
{\em Communications in Mathematical Physics}, {\bf 121}, 177--223.

\bibitem{Fuch}
Fuchs, J. (1992).
Affine Lie algebras and quantum groups.
{\em Cambridge University Press}.

\bibitem{FuRS}
Fuchs, J., Runkel, I. and Schweigert, C. (2002).
TFT construction of RCFT correlators I: Partition functions.
{\em Nuclear Physics B}, {\bf 646}, 353--497.
hep-th/0204148.

\bibitem{FS1}
Fuchs, J. and Schweigert, C. (2000).
Solitonic sectors, alpha-induction and symmetry breaking boundaries.
{\em Physics Letters}, {\bf B490}, 163--172.
hep-th/0006181.

\bibitem{FS2}
Fuchs, J. and Schweigert, C. (preprint 2001).
Category theory for conformal boundary conditions. 
math.CT/0106050.

\bibitem{GF}
Gabbiani, F. and Fr\"ohlich, J. (1993).
Operator algebras and conformal field theory.
{\em Communications in Mathematical Physics}, {\bf 155},  569--640.

\bibitem{Ga1}
Gannon, T. (1993).
WZW commutants, lattices and level -- one partition functions.
{\em Nuclear Physics B}, {\bf 396}, 708--736.

\bibitem{Ga2}
Gannon, T. (1994).
The classification of affine $SU(3)$ modular invariant
partition functions.
{\em Communications in Mathematical Physics},
{\bf 161}, 233--264.

\bibitem{GannH}
Gannon, T. and Ho-Kim, Q. (1994).
The rank-four heterotic modular invariant partition functions.
{\em Nuclear Physics B}, {\bf 425}, 319--342.

\bibitem{Gant}
Gantmacher, F. R. (1960).
The theory of matrices. Vol. 2.
Chelsea.

\bibitem{GP}
Ge, L. and Popa, S. (1998).
On some decomposition properties for factors of type II$_1$.
{\em Duke Mathematical Journal}, {\bf 94}, 79--101.

\bibitem{GW}
Gepner, D. and Witten, E. (1986).
String theory on group manifolds.
{\em Nuclear Physics B}, {\bf 278}, 493--549.

\bibitem{Gn}
Gnerre, S. (2000).
Free compositions of paragroups.
{\em Journal of Functional Analysis}, {\bf 175}, 251--278.

\bibitem{GKO}
Goddard, P., Kent, A. and Olive, D. (1986).
Unitary representations of the Virasoro and 
super-Virasoro algebras.
{\em Communications in Mathematical Physics},
{\bf 103}, 105--119.

\bibitem{GNO}
Goddard, P., Nahm, W. and Olive, D. (1985).
Symmetric spaces, Sugawara's energy momentum tensor in two
dimensions and free Fermions.
{\em Physics Letters}, {\bf 160B}, 111--116.

\bibitem{GO}
Goddard, P. and Olive, D. (1988).
Kac--Moody and Virasoro algebras.
{\em Advanced Series in Mathematical Physics},
{\bf 3}, World Scientific.

\bibitem{G}
Goldman, M. (1960).
On subfactors of type II$_1$.
{\em The Michigan Mathematical Journal}, {\bf 7}, 167--172.

\bibitem{GJ}
Goldschmidt, D. M. and Jones, V. F. R. (1989)
Metaplectic link invariants.
{\em Geometriae Dedicata}, {\bf 31}, 165--191.

\bibitem{GHJ}
Goodman, F., de la Harpe, P. and Jones, V. F. R. (1989).
Coxeter graphs and towers of algebras.
{\em MSRI Publications (Springer)}, {\bf 14}.

\bibitem{GWe}
Goodman, F. and Wenzl, H. (1990).
Littlewood Richardson coefficients for Hecke 
algebras at roots of unity.
{\em Advances in Mathematics}, {\bf 82}, 244--265.

\bibitem{Go1}
Goto, S. (1994).
Orbifold construction for non-AFD subfactors.
{\em International Journal of Mathematics}, {\bf 5}, 725--746.

\bibitem{Go2}
Goto, S. (1995).
Symmetric flat connections, triviality of Loi's invariant, and
orbifold subfactors.
{\em Publications of the RIMS, Kyoto University},
{\bf 31}, 609--624.

\bibitem{Go3}
Goto, S. (1996).
Commutativity of automorphisms of subfactors
modulo inner automorphisms.
{\em Proceedings of the American Mathematical Society},
{\bf 124}, 3391--3398.

\bibitem{Go4}
Goto, S. (2000).
Quantum double construction for subfactors arising from
periodic commuting squares.
{\em Journal of the Mathematical Society of Japan},
{\bf 52}, 187--198.

\bibitem{Gr}
Greenleaf, F. P. (1969).
{\em Invariant means on topological groups and their applications},
van Nostrand, New York.

\bibitem{Gro}
Grossman, P. (preprint 2006).
Forked Temperley-Lieb algebras and intermediate subfactors. 
math.OA/0607335.

\bibitem{GI}
Grossman, P., and Izumi, M. (2008).
Classification of noncommuting quadrilaterals of factors.
{\em International Journal of Mathematics},
{\bf 19}, 557--643.
arXiv:0704.1121.

\bibitem{GrJ}
Grossman, P. and Jones, V. F. R. (2007).
Intermediate subfactors with no extra structure.
{\em Journal of the American Mathematical Society},
{\bf 20}, 219--265.
math.OA/0412423.

\bibitem{GL1}
Guido, D. and  Longo, R. (1992).
Relativistic invariance and
charge conjugation in quantum field theory.
{\em Communications in Mathematical Physics},
{\bf 148}, 521---551.

\bibitem{GL2}
Guido, D. and  Longo, R. (1995).
An algebraic spin and statistics theorem.
{\em Communications in Mathematical Physics},
{\bf 172}, 517--533.

\bibitem{GL3}
Guido, D. and  Longo, R. (1996).
The conformal spin and statistics theorem.
{\em Communications in Mathematical Physics},
{\bf 181}, 11--35.

\bibitem{GL4}
Guido, D. and  Longo, R. (in press).
A converse Hawking-Unruh effect and dS$^2$/CFT correspondance.
{\em Annales Henri Poincar\'e}.
gr-qc/0212025.

\bibitem{GLW}
Guido, D., Longo, R. and  Wiesbrock, H.-W. (1998).
Extensions of conformal nets and superselection structures.
{\em Communications in Mathematical Physics},
{\bf 192}, 217--244.

\bibitem{GLRV}
Guido, D., Longo, R., Roberts, J. E. and Verch, R. (2001).
Charged sectors, spin and statistics in quantum field theory
on curved spacetimes. 
{\em Reviews in Mathematical Physics}, {\bf 13}, 125--198.

\bibitem{GJS}
Guionnet, A., Jones, V. F. R.,  Shlyakhtenko, D. (preprint 2007).
Random matrices, free probability, planar algebras and subfactors.
arXiv:0712.2904.

\bibitem{H}
Haag, R. (1996).
Local Quantum Physics.
Springer-Verlag, Berlin-Heidelberg-New York.

\bibitem{HaK}
Haag, R and Kastler, T. (1964).
An algebraic approach to quantum field theory.
{\em Journal of Mathematical Physics},
{\bf 5}, 848--861.

\bibitem{H1}
Haagerup, U. (1975).
Normal weights on $W^*$-algebras.
{\em Journal of Functional Analysis},
{\bf 19}, 302--317.

\bibitem{H2}
Haagerup, U. (1983).
All nuclear $C^*$-algebras are amenable.
{\em Inventiones Mathematica}, {\bf 74}, 305--319.

\bibitem{H3}
Haagerup, U. (1987).
Connes' bicentralizer problem and the uniqueness of 
the injective factor of type $III_1$.
{\em Acta Mathematica}, {\bf 158}, 95--148.

\bibitem{H4}
Haagerup, U. (preprint 1991).
Quasi-traces on exact $C^*$--algebras are traces.

\bibitem{H5}
Haagerup, U. (1994). Principal graphs of subfactors in the index range 
$4< 3+\sqrt2$. in {\em Subfactors -Proceedings of the Taniguchi Symposium, Katata -},
(ed. H. Araki, et al.), World Scientific, 1--38.

\bibitem{HS1}
Haagerup, U. and St\"ormer, E. (1990). Equivalences of normal states on von Neumann algebras 
and the flow of weights. {\em Advances in Mathematics} {\bf 83}, 180--262.

\bibitem{HS2}
Haagerup, U. and St\"ormer, E. (1990).
Pointwise inner automorphisms of von Neumann
algebras (with an appendix by Sutherland, C. E.).
{\em Journal of Functional Analysis}, {\bf 92}, 177--201.

\bibitem{HaW}
Haagerup, U. and Winsl\"ow, C. (1998).
The Effros--Mar\'echal topology in the space of
von Neumann algebras.
{\em American Journal of Mathematics}, {\bf 120}, 567--617.

\bibitem{HaW2}
Haagerup, U. and Winsl\"ow, C. (2000).
The Effros--Mar\'echal topology in the space of
von Neumann algebras, II.
{\em Journal of Functional Analysis}, {\bf 171}, 401--431.

\bibitem{HK1}
Hamachi, T. and Kosaki, H. (1988).
Index and flow of weights of factors of type III.
{\em Proceedings of the Japan Academy}, {\bf 64A}, 11--13.

\bibitem{HK2}
Hamachi, T. and Kosaki, H. (1988).
Inclusions of type III factors constructed from ergodic flows.
{\em Proceedings of the Japan Academy}, {\bf 64A}, 195--197.

\bibitem{HK3}
Hamachi, T. and Kosaki, H. (1993).
Orbital factor map.
{\em Ergodic Theory and Dynamical Systems}, {\bf 13}, 515--532.

\bibitem{dlH}
de la Harpe, P. (1979).
Moyennabilit\'e du groupe unitaire et propri\'et\'e
P de Schwartz des alg\`ebres de von Neumann.
{\em Alg\`ebres d'op\'erateurs (S\'eminaire, Les 
Plans-sur-Bex, Suisse 1978) (editor, P. de la
Harpe), Lecture Notes in Mathematics,
Springer-Verlag}, {\bf 725}, 220--227.

\bibitem{HW}
de la Harpe, P. and Wenzl, H. (1987).
Operations sur les rayons spectraux de matrices
symetriques entieres  positives.
{\em Comptes Rendus de l'Academie des Sciences, S\'erie I,
Math\'ematiques},
{\bf 305}, 733--736.

\bibitem{Hv1}
Havet, J.-F. (1976).
Esp\'erance conditionnelles permutables \`a un group
d'automorphismes sur une alg\`ebre de von Neumann.
{\em Comptes Rendus de l'Academie des Sciences, S\'erie I, Math\'ematiques},
{\bf 282}  (1976),  1095--1098..

\bibitem{Hv2}
Havet, J.-F. (1990).
Esp\'erance conditionnelle minimale.
{\em Journal of Operator Theory}, {\bf 24}, 33--35.

\bibitem{Hy1}
Hayashi, T. (1993).
Quantum group symmetry of partition functions of
IRF models and its application to Jones' index theory.
{\em Communications in Mathematical Physics}, {\bf 157}, 331--345.

\bibitem{Hy2}
Hayashi, T. (1996).
Compact quantum groups of face type.
{\em Publications of the RIMS, Kyoto University},
{\bf 32}, 351--369.

\bibitem{Hy3}
Hayashi, T. (1998).
Faes algebras I. A generalization of quantum group theory.
{\em Journal of the Mathematical Society of Japan}, 
{\bf 50}, 293--315.

\bibitem{Hy4}
Hayashi, T. (1999).
Face algebras and unitarity of SU$(N)_L$-TQFT.
{\em Communications in Mathematical Physics},
{\bf 203}, 211-247.

\bibitem{Hy5}
Hayashi, T. (1999).
Galois quantum groups of II$_1$-subfactors.
{\em T\^ohoku Mathematical Journal},
{\bf 51}, 365--389.

\bibitem{Hy-to}
Hayashi, T. (2000).
Harmonic function spaces of probability measures on fusion
algebras.
{\em Publications of the RIMS, Kyoto University},
{\bf 36}, 231--252.

\bibitem{HY}
Hayashi, T. and Yamagami, S. (2000).
Amenable tensor categories and their realizations
as AFD bimodules.
{\em Journal of Functional Analysis}, {\bf 172}, 19--75.

\bibitem{HO}
Herman, R. H. and Ocneanu, A. (1984).
Stability for integer actions on UHF $C^*$-algebras.
{\em Journal of Functional Analysis},
{\bf 59}, 132--144.

\bibitem{Hiai1}
Hiai, F. (1988).
Minimizing indices of conditional expectations onto a subfactor. 
{\em Publications of the RIMS, Kyoto University}, {\bf 24}, 673--678.

\bibitem{Hiai2}
Hiai, F. (1990).
Minimum index for subfactors and entropy I.
{\em Journal of Operator Theory}, {\bf 24}, 301--336.

\bibitem{Hi3}
Hiai, F. (1991).
Minimum index for subfactors and entropy, II.
{\em Journal of the Mathematical Society of Japan}, 
{\bf 43}, 347--380.

\bibitem{Hi4}
Hiai, F. (1994).
Entropy and growth for derived towers of subfactors.
in {\em Subfactors ---
Proceedings of the Taniguchi Symposium, Katata ---},
(ed. H. Araki, et al.),
World Scientific, 206--232.

\bibitem{Hi5}
Hiai, F.  (1995).
Entropy for canonical shifts and strong amenability. 
{\em International Journal of Mathematics},
{\bf 6}, 381--396.

\bibitem{Hi6}
Hiai, F. (1997).
Standard invariants for crossed products inclusions
of factors.
{\em Pacific Journal of Mathematics},
{\bf 177}, 237--267.

\bibitem{HI}
Hiai, F. and Izumi, M. (1998).
Amenability and strong amenability for fusion algebras
with applications to subfactor theory.
{\em International Journal of Mathematics}, {\bf 9},
669--722.

\bibitem{HL}
Hislop, P. and Longo, R. (1982).
Modular structure of the local algebras
associated with the free massless scalar field theory.
{\em Communications in Mathematical Physics},
{\bf 84} (1982) 71--85.

\bibitem{Ho}
Hofman, A. J. (1972).
On limit points of spectral radii of non-negative 
symmetric  integral matrices.
{\em Lecture Notes in Mathematics}, {\bf 303}, 
Springer Verlag, 165--172.

\bibitem{Ik}
Ikeda, K. (1998).
Numerical evidence for flatness of Haagerup's connections.
{\em Journal of the Mathematical Sciences, University of Tokyo},
{\bf 5}, 257--272.

\bibitem{IT}
Ishikawa, H. and Tani, T. (preprint 2002).
Novel construction of boundary states in coset
conformal field theories.
hep-th/0207177.

\bibitem{I1}
Izumi, M. (1991).
Application of fusion rules to
classification of subfactors.
{\em Publications of the RIMS, Kyoto University},
{\bf 27}, 953--994.

\bibitem{I2}
Izumi, M. (1992).
Goldman's type theorem for index 3.
{\em Publications of the RIMS, Kyoto University},
{\bf 28}, 833--843.

\bibitem{I4}
Izumi, M. (1993).
On type III and type II principal graphs for
subfactors.
{\em Mathematica Scandinavica}, {\bf 73}, 307--319.

\bibitem{I5}
Izumi, M. (1993).
Subalgebras of infinite $C^*$-algebras with
finite Watatani indices I. Cuntz algebras.
{\em Communications in Mathematical Physics},
{\bf 155}, 157--182.

\bibitem{I3}
Izumi, M. (1994).
On flatness of the Coxeter graph $E_8$.
{\em Pacific Journal of Mathematics}, {\bf 166},
305--327.

\bibitem{I6}
Izumi, M. (1994).
Canonical extension of endomorphisms of factors.
in {\em Subfactors --- Proceedings of the Taniguchi Symposium, Katata ---},
(ed. H. Araki, et al.), World Scientific, 274--293.

\bibitem{I7}
Izumi, M. (1998).
Subalgebras of infinite $C^*$-algebras with
finite Watatani indices II: Cuntz-Krieger algebras.
{\em Duke Mathematical Journal}, {\bf 91}, 409--461.

\bibitem{I8}
Izumi, M. (1999).
Actions of compact quantum groups on operator algebras.
in {\em XIIth International Congress of Mathematical Physics},
International Press, 249--253.

\bibitem{I9}
Izumi, M. (2000).
The structure of sectors associated with Longo-Rehren
inclusions I. General theory.
{\em Communications in Mathematical Physics}, {\bf 213}, 127--179.

\bibitem{I10}
Izumi, M. (2001).
The structure of sectors associated with Longo-Rehren
inclusions II. Examples.
{\em Reviews in Mathematical Physics}, {\bf 13}, 603--674.

\bibitem{I11}
Izumi, M. (2002).
Non-commutative Markov operators arising from subfactors.
{\em Advances in Mathematics}, {\bf 169}, 1--57.

\bibitem{I12}
Izumi, M. (2003).
Canonical extension of endomorphisms of type III factors.
{\em American Journal of Mathematics}, {\bf 125}, 1--56.

\bibitem{I13}
Izumi, M. (2002).
Inclusions of simple $C^*$-algebras.
{\em Journal f\"ur die Reine und Angewandte Mathematik},
{\bf 547}, 97--138.

\bibitem{I14}
Izumi, M. (2002).
Characterization of isomorphic group-subgroup subfactors.
{\em International Mathematics Research Notices}, 1791--1803.

\bibitem{IK}
Izumi, M. and Kawahigashi, Y. (1993).
Classification of subfactors with the principal graph $D^{(1)}_n$.
{\em Journal of Functional Analysis}, {\bf 112}, 257--286.

\bibitem{IKo1}
Izumi, M. and Kosaki, H. (1996).
Finite dimensional Kac algebras arising from certain
group actions on a factor.
{\em International Mathematics Research Notices}, 357--370.

\bibitem{IKo2}
Izumi, M. and Kosaki, H. (2002).
Kac algebras arising from
composition of subfactors: General theory and classification.
{\em Memoirs of the American Mathematical Society}, {\bf 158}, 
no. 750.

\bibitem{IKo3}
Izumi, M. and Kosaki, H. (2002).
On a subfactor analogue of the second cohomology.
{\em Reviews in Mathematical Physics}, {\bf 14}, 733--757.

\bibitem{ILP}
Izumi, M., Longo, R. and Popa, S. (1998).
A Galois correspondence for compact groups of automorphisms
of von Neumann algebras with a generalization to Kac algebras.
{\em Journal of Functional Analysis}, {\bf 155}, 25--63.

\bibitem{Ji}
Jimbo, M. (1986).
A $q$-difference analogue of $U(g)$ and the Yang-Baxter
equation.
{\em Letters in Mathematical Physics},
{\bf 102}, 537--567.

\bibitem{Ji2}
Jimbo, M. (1986).
A $q$-analogue of $U(N+1)$, Hecke algebra and the
Yang-Baxter equation.
{\em Letters in Mathematical Physics},
{\bf 11}, 247--252.

\bibitem{JMO1}
Jimbo, M., T, Miwa and Okado, M. (1987).
Solvable lattice models whose states are dominant 
integral weights of $A^{(1)}_{n-1}$.
{\em Letters in Mathematical Physics}, {\bf 14}, 
123--131.

\bibitem{JMO2}
Jimbo, M., T, Miwa and Okado, M. (1988).
Solvable lattice models related to the vector 
representation of classical simple Lie algebras.
{\em Communications in Mathematical Physics},
{\bf 116}, 507--525.

\bibitem{J1}
Jones, V. F. R. (1980).
A factor anti-isomorphic to itself but without 
involutory anti-automorphisms.
{\em Mathematica Scandinavica}, {\bf 46}, 103--117.

\bibitem{J2}
Jones, V. F. R. (1980).
Actions of finite groups
on the hyperfinite type II$_1$ factor.
{\em Memoirs of the American Mathematical Society},
{\bf 237}.

\bibitem{J3}
Jones, V. F. R. (1983).
Index for subfactors.
{\em Inventiones Mathematicae}, {\bf 72}, 1--25.

\bibitem{J4}
Jones, V. F. R. (1985).
A polynomial invariant for knots via von Neumann algebras.
{\em Bulletin of the American Mathematical Society}, {\bf 12}, 103--112.

\bibitem{J5}
Jones, V. F. R. (1986).
A new knot polynomial and von Neumann algebras.
{\em Notices of the American Mathematical Society},
{\bf 33}, 219--225.

\bibitem{J6}
Jones, V. F. R. (1986).
Braid groups, Hecke algebras and type II$_1$ 
factors. (1986). in {\em Geometric Methods in
Operator Algebras}, ed. H. Araki and E. G. Effros.,
Longman, 242--273.

\bibitem{J7}
Jones, V. F. R. (1987).
Hecke algebra representations of braid groups and 
link polynomials. 
{\em Annals of Mathematics}, {\bf 126}, 335--388.

\bibitem{J8}
Jones, V. F. R. (1989).
On a certain value of the Kauffman polynomial.
{\em Communications in Mathematical Physics},  {\bf 125}, 459--467.

\bibitem{J9}
Jones, V. F. R. (1989).
On knot invariants related to some statistical mechanical models. 
{\em Pacific Journal of Mathematics}, {\bf 137}, 311--334.

\bibitem{J10}
Jones, V. F. R. (1990).
Knots, braids and statistical mechanics.
in {\em Advances in differential geometry and topology}, 149--184.

\bibitem{J11}
Jones, V. F. R. (1990).
Baxterization.
{\em International Journal of Modern Physics B}, 
{\bf 4}, 701--713.

\bibitem{J12}
Jones, V. F. R. (1990).
Notes on subfactors and statistical mechanics.
{\em International Journal of Modern Physics A}, 
{\bf 5}, 441--460.

\bibitem{J13}
Jones, V. F. R. (1991).
von Neumann algebras in mathematics and physics.
{\em Proceedings of ICM-90}, 121--138, Springer.

\bibitem{J14}
Jones, V. F. R. (1991).
Subfactors and knots.
CBMS Regional Conference Series in Mathematics, 
{\bf 80}.

\bibitem{J15}
Jones, V. F. R. (1992).
From quantum theory to knot theory and back:  a von
Neumann algebra  excursion. 
in {\em American Mathematical Society centennial publications, Vol.  II}, 
321--336.

\bibitem{J16}
Jones, V. F. R. (1994).
The Potts model and the symmetric group.
in {\em Subfactors ---
Proceedings of the Taniguchi Symposium, Katata ---},
(ed. H. Araki, et al.),
World Scientific, 259--267.

\bibitem{J17}
Jones, V. F. R. (1994).
On a family of almost commuting endomorphisms.
{\em Journal of Functional Analysis}, {\bf 119}, 84--90.

\bibitem{J18}
Jones, V. F. R. (1995).
Fusion en alg\`ebres de von Neumann et groupes de lacets
(d'apr\`es A. Wassermann).
{\em Seminaire Bourbaki}, 
{\bf 800}, 1--20.

\bibitem{J19}
Jones, V. F. R. (in press).
Planar algebras I.
{\em New Zealand Journal of Mathematics}.
math.QA/9909027.

\bibitem{J20}
Jones, V. F. R. (2000).
Ten problems.
in {\em Mathematics: frontiers and perspectives},
Amer. Math. Soc., 79--91.

\bibitem{J21}
Jones, V. F. R. (2000).
The planar algebras of a bipartite graph.
in {\em Knots in Hellas '98}, World Scientific, 94--117.

\bibitem{J22}
Jones, V. F. R. (2001).
The annular structure of subfactors.
in {\em Essays on geometry and related topics},
Monographies de L'Enseignement Mathe\'matique, {\bf 38}, 401--463.

\bibitem{J23}
Jones, V. F. R. (preprint 2003).
In and around the origin of quantum groups.
math.OA/0309199.

\bibitem{J24}
Jones, V. F. R. (preprint 2003).
Quadratic tangles in planar algebras.

\bibitem{JS}
Jones, V. F. R. and Sunder, V. S. (1997).
Introduction to subfactors.
London Math. Soc. Lecture Notes Series {\bf 234}, Cambridge
University Press.

\bibitem{JT}
Jones, V. F. R. and Takesaki, M. (1984).
Actions of compact abelian groups on semifinite 
injective factors.
{\em Acta Mathematica}, {\bf 153}, 213--258.

\bibitem{JonesWass}
Jones, V. F. R. and Wassermann, A. (in preparation).
Fermions on the circle and representations of loop groups.

\bibitem{JX}
Jones, V. F. R. and Xu, F. (2004).
Intersections of finite families of finite index subfactors.
{\em International Journal of Mathematics}, {\bf 15}, 717--733.
math.OA/0406331.

\bibitem{Kac}
Kac, V. (1990).
Infinite dimensional Lie algebras.
{\em 3\`eme \'edition, Cambridge University Press}.

\bibitem{KLX}
Kac, V., Longo, R. and Xu, F. (2004).
Solitons in affine and permutation orbifolds.
{\em Communications in Mathematical Physics}, {\bf 253}, 723--764.
math.OA/0312512.

\bibitem{KP}
Ka\v c, V. and Petersen, D. H. (1984).
Infinite dimensional Lie algebras, theta functions, 
and modular forms.
{\em Advances in Mathematics}, {\bf 53}, 125--264.

\bibitem{KR}
Ka\v c, V. and Raina, A. K. (1987).
Highest weight representations of infinite dimensional Lie algebras.
World Scientific.

\bibitem{KaR1}
Kadison, R. V. and Ringrose, J. R. (1983).
Fundamentals of the Theory of Operator Algebras 
Vol I.  Academic Press.

\bibitem{KaR2}
Kadison, R. V. and Ringrose, J. R. (1986).
Fundamentals of the Theory of Operator Algebras 
Vol II. Academic Press.

\bibitem{KPW}
Kajiwara, T., Pinzari, C., and Watatani, Y. (2004).
Jones index theory for Hilbert $C^*$-bimodules and 
its equivalence with conjugation theory.
{\em Journal of Functional Analysis}, {\bf 215}, 1--49.
math.OA/0301259  

\bibitem{Kl}
Kallman, R. R. (1971).
Groups of inner automorphisms of von Neumann algebras.
{\em Journal of Functional Analysis}, {\bf 7},  43--60.

\bibitem{Kar1}
Karowski, M. (1987).
Conformal quantum field theories and integrable theories.
{\em Proceedings of the Brasov Summer School, September
1987, ed. P. Dita et al., Academic Press}.

\bibitem{Kar}
Karowski, M. (1988).
Finite size corrections for integrable systems and conformal
properties of six-vertex models.
{\em Nuclear Physics B}, {\bf 300}, 473--499.

\bibitem{Kastler}
Kastler D. (1990).
(editor) The algebraic theory of Superselection sectors.
Introduction and recent results.
{\em World Scientific, Singapore}.

\bibitem{Kt}
Kato, A. (1987).
Classification of modular invariant partition
functions in two dimensions.
{\em Modern Physics Letters A}, {\bf 2}, 585--600.

\bibitem{Kff1}
Kauffman, L. H. (1986).
Chromatic polynomial, Potts model, Jones polynomial
{\em Lecture notes}.

\bibitem{Kff2}
Kauffman, L. (1987).
State models and the Jones polynomial.
{\em Topology}, {\bf 26}, 395--407.

\bibitem{Kff3}
Kauffman, L. (1990).
An invariant of regular isotopy.
{\em Transactions of the American Mathematical Society}, 
{\bf 318}, 417--471.

\bibitem{Kff4}
Kauffman, L. (1991).
{\em Knots and Physics}.
World Scientific, Singapore.

\bibitem{KfL}
Kauffman, L. and Lins, S. L. (1994).
{\em Temperley--Lieb recoupling theory and invariants of $3$-manifolds}.
Princeton University Press, Princeton.

\bibitem{K1}
Kawahigashi, Y. (1992).
Automorphisms commuting with a conditional expectation
onto a subfactor with finite index.
{\em Journal of Operator Theory},
{\bf 28}, 127--145.

\bibitem{K3}
Kawahigashi, Y. (1992).
Exactly solvable orbifold models and subfactors.
{\em Functional Analysis and Related Topics},
Lecture Notes in Mathematics, Springer, {\bf 1540}, 127--147.

\bibitem{K4}
Kawahigashi, Y. (1993).
Centrally trivial automorphisms and an analogue of
Connes's $\chi(M)$ for subfactors.
{\em Duke Mathematical Journal}, {\bf 71}, 93--118.

\bibitem{K7}
Kawahigashi, Y. (1994).
Paragroup and their actions on subfactors.
in {\em Subfactors ---
Proceedings of the Taniguchi Symposium, Katata ---},
(ed. H. Araki, et al.),
World Scientific, 64--84.

\bibitem{K2}
Kawahigashi, Y. (1995).
On flatness of Ocneanu's connections on the Dynkin diagrams
and classification of subfactors.
{\em Journal of Functional Analysis},
{\bf 127}, 63--107.

\bibitem{K8}
Kawahigashi, Y. (1995).
Orbifold subfactors, central sequences and the relative Jones
invariant $\kappa$.
{\em International Mathematical Research Notices},
129--140.

\bibitem{K6}
Kawahigashi, Y. (1995).
Classification of paragroup actions on subfactors.
{\em Publications of the RIMS, Kyoto University}, 
{\bf 31}, 481--517.

\bibitem{K5}
Kawahigashi, Y. (1996).
Paragroups as quantized Galois groups of subfactors.
{\em Sugaku Expositions}, {\bf 9}, 21--35.

\bibitem{K9}
Kawahigashi, Y. (1997).
Classification of approximately inner automorphisms of
subfactors.
{\em Mathematische Annalen}, {\bf 308}, 425--438.

\bibitem{K10}
Kawahigashi, Y. (1997).
Quantum doubles and orbifold subfactors.
in {\em Operator Algebras and Quantum Field Theory}
(ed. S. Doplicher, et al.), International Press, 271--283.

\bibitem{K11}
Kawahigashi, Y. (1998).
Subfactors and paragroup theory.
in {\em Operator Algebras and Operator Theory} (ed. L. Ge, et al.),
{\em Contemporary Mathematics}, {\bf 228}, 179--188.

\bibitem{K12}
Kawahigashi, Y. (1999).
Quantum Galois correspondence for subfactors.
{\em Journal of Functional Analysis}, {\bf 167}, 481--497.

\bibitem{K13}
Kawahigashi, Y. (2004).
Braiding and nets of factors on the circle.
in {\em Operator Algebras and Applications} (ed. H. Kosaki),
Advanced Studies in Pure Mathematics {\bf 38}, 219--228.

\bibitem{K14}
Kawahigashi, Y. (2001).
Braiding and extensions of endomorphisms of subfactors.
in {\em Mathematical Physics in Mathematics and Physics} (ed. R. Longo),
The Fields Institute Communications {\bf 30}, Providence, Rhode Island:
AMS Publications, 261--269.

\bibitem{K15}
Kawahigashi, Y. (2002).
Generalized Longo-Rehren subfactors and $\alpha$-induction.
{\em Communications in Mathematical Physics}, {\bf 226}, 269--287.
math.OA/0107127.

\bibitem{K16}
Kawahigashi, Y. (2003).
Conformal quantum field theory and subfactors.
{\em Acta Mathematica Sinica}, {\bf 19}, 557--566.

\bibitem{K17}
Kawahigashi, Y. (2003).
Classification of operator algebraic conformal field theories.
{\em ``Advances in Quantum Dynamics'', Contemporary Mathematics},
{\bf 335}, 183--193.
math.OA/0211141.

\bibitem{K18}
Kawahigashi, Y. (2005).
Subfactor theory and its applications
--- operator algebras and quantum field theory ---.
{\em ``Selected Papers on Differential Equations''},
Amer. Math. Soc. Transl. {\bf 215}, Amer. Math. Soc., 97--108.

\bibitem{K19}
Kawahigashi, Y. (2005). Topological quantum field theories and operator algebras.
{\em Quantum Field Theory and Noncommutative Geometry}, Lecture Notes in Physics, Springer, {\bf 662}, 241--253. math.OA/0306112.

\bibitem{K20}
Kawahigashi, Y. (2005). Classification of operator algebraic conformal field theories 
in dimensions one and two. {\em XIVth International Congress on Mathematical Physics}, 
World Scientific, 476--485. math-ph/0308029.

\bibitem{K21}
Kawahigashi, Y. (preprint 2007). Conformal field theory and operator algebras.
arXiv:0704.0097.

\bibitem{K22}
Kawahigashi, Y. (preprint 2007). Superconformal field theory and operator algebras.

\bibitem{KaL}
Kawahigashi, Y. and Longo, R. (2004).
Classification of Local Conformal Nets. Case $c<1$.
{\em Annals of Mathematics}, {\bf 160}, 493--522.
math-ph/0201015.

\bibitem{KaL2}
Kawahigashi, Y. and Longo, R. (2004).
Classification of two-dimensional local conformal nets with $c<1$
and 2-cohomology vanishing for tensor categories.
{\em Communications in Mathematical Physics}, {\bf 244}, 63--97.
math-ph/0304022.

\bibitem{KaL3}
Kawahigashi, Y. and Longo, R. (2005).Noncommutative spectral invariants and black hole entropy.
{\em Communications in Mathematical Physics}, {\bf 257}, 193--225.
math-ph/0405037.

\bibitem{KaL4}
Kawahigashi, Y. and Longo, R. (2006).
Local conformal nets arising from framed vertex operator algebras.
{\em Advances in Mathematics}, {\bf 206}, 729--751. math.OA/0407263.

\bibitem{KLM}
Kawahigashi, Y., Longo, R. and M\"uger, M. (2001).
Multi-interval subfactors and modularity of representations
in conformal field theory. {\em Communications in Mathematical Physics}, {\bf 219}, 631--669.
math.OA/9903104.

\bibitem{KLPR}
Kawahigashi, Y., Longo, R., Pennig, U. and Rehren, K.-H. (2007).
The classification of non-local chiral CFT with $c<1$.
{\em Communications in Mathematical Physics}, {\bf 271}, 375--385. math.OA/0505130.

\end{thebibliography}

%%%%%
%%%%%
\end{document}
