\documentclass[12pt]{article}
\usepackage{pmmeta}
\pmcanonicalname{QuantumAlgebraicTopologyTopics}
\pmcreated{2013-03-22 18:13:31}
\pmmodified{2013-03-22 18:13:31}
\pmowner{bci1}{20947}
\pmmodifier{bci1}{20947}
\pmtitle{quantum algebraic topology topics}
\pmrecord{63}{40811}
\pmprivacy{1}
\pmauthor{bci1}{20947}
\pmtype{Topic}
\pmcomment{trigger rebuild}
\pmclassification{msc}{81Q60}
\pmclassification{msc}{55-02}
\pmclassification{msc}{81R50}
\pmclassification{msc}{81R15}
\pmsynonym{non-Abelian Quantum Algebraic Topology}{QuantumAlgebraicTopologyTopics}
\pmsynonym{NAQAT}{QuantumAlgebraicTopologyTopics}
\pmsynonym{Quantum Operator Algebras}{QuantumAlgebraicTopologyTopics}
\pmsynonym{Supersymmetry and symmetry breaking}{QuantumAlgebraicTopologyTopics}
\pmsynonym{Algebraic Topology Foundations of Quantum theories}{QuantumAlgebraicTopologyTopics}
%\pmkeywords{algebraic topology foundations of quantum theories}
%\pmkeywords{non-Abelian algebraic theory}
%\pmkeywords{supergravity /supersymmetry theories (quantum gravity)}
\pmrelated{CAlgebra3}
\pmrelated{TopicEntryOnTheAlgebraicFoundationsOfQuantumAlgebraicTopology}
\pmrelated{CategoryOfQuantumAutomata}
\pmrelated{TheoremOnCWComplexApproximationOfQuantumStateSpacesInQAT}

% this is the default PlanetMath preamble.  as your .

\usepackage{amssymb}
\usepackage{amsmath}
\usepackage{amsfonts}

% almost certainly you want these
\usepackage{amssymb}
\usepackage{amsmath}
\usepackage{amsfonts}

% define commands here
\usepackage{amsmath, amssymb, amsfonts, amsthm, amscd,  enumerate}
\usepackage{xypic, xspace}
\usepackage[mathscr]{eucal}
\usepackage[dvips]{graphicx}
\usepackage[curve]{xy}
\setlength{\textwidth}{6.5in}
\setlength{\textheight}{9.0in}
\voffset=-.4in
\theoremstyle{plain}
\newtheorem{lemma}{Lemma}[section]
\newtheorem{proposition}{Proposition}[section]
\newtheorem{theorem}{Theorem}[section]
\newtheorem{corollary}{Corollary}[section]
\theoremstyle{definition}
\newtheorem{definition}{Definition}[section]
\newtheorem{example}{Example}[section]
\newtheorem{remark}{Remark}[section]
\newtheorem*{notation}{Notation}
\newtheorem*{claim}{Claim}
\renewcommand{\thefootnote}{\ensuremath{\fnsymbol{footnote}}}
\numberwithin{equation}{section}
\newcommand{\Ad}{{\rm Ad}}
\newcommand{\Aut}{{\rm Aut}}
\newcommand{\Cl}{{\rm Cl}}
\newcommand{\Co}{{\rm Co}}
\newcommand{\DES}{{\rm DES}}
\newcommand{\Diff}{{\rm Diff}}
\newcommand{\Dom}{{\rm Dom}}
\newcommand{\Hol}{{\rm Hol}}
\newcommand{\Mon}{{\rm Mon}}
\newcommand{\Hom}{{\rm Hom}}
\newcommand{\Ker}{{\rm Ker}}
\newcommand{\Ind}{{\rm Ind}}
\newcommand{\IM}{{\rm Im}}
\newcommand{\Is}{{\rm Is}}
\newcommand{\ID}{{\rm id}}
\newcommand{\grpL}{{\rm GL}}
\newcommand{\Iso}{{\rm Iso}}
\newcommand{\rO}{{\rm O}}
\newcommand{\Sem}{{\rm Sem}}
\newcommand{\SL}{{\rm Sl}}
\newcommand{\St}{{\rm St}}
\newcommand{\Sym}{{\rm Sym}}
\newcommand{\Symb}{{\rm Symb}}
\newcommand{\SU}{{\rm SU}}
\newcommand{\Tor}{{\rm Tor}}
\newcommand{\U}{{\rm U}}
\newcommand{\A}{\mathcal A}
\newcommand{\Ce}{\mathcal C}
\newcommand{\E}{\mathcal E}
\newcommand{\F}{\mathcal F}
%\newcommand{\grp}{\mathcal G}
\renewcommand{\H}{\mathcal H}
\renewcommand{\cL}{\mathcal L}
\newcommand{\Q}{\mathcal Q}
\newcommand{\R}{\mathcal R}
\newcommand{\cS}{\mathcal S}
\newcommand{\cU}{\mathcal U}
\newcommand{\W}{\mathcal W}
\newcommand{\bA}{\mathbb{A}}
\newcommand{\bB}{\mathbb{B}}
\newcommand{\bC}{\mathbb{C}}
\newcommand{\bD}{\mathbb{D}}
\newcommand{\bE}{\mathbb{E}}
\newcommand{\bF}{\mathbb{F}}
\newcommand{\bG}{\mathbb{G}}
\newcommand{\bK}{\mathbb{K}}
\newcommand{\bM}{\mathbb{M}}
\newcommand{\bN}{\mathbb{N}}
\newcommand{\bO}{\mathbb{O}}
\newcommand{\bP}{\mathbb{P}}
\newcommand{\bR}{\mathbb{R}}
\newcommand{\bV}{\mathbb{V}}
\newcommand{\bZ}{\mathbb{Z}}
\newcommand{\bfE}{\mathbf{E}}
\newcommand{\bfX}{\mathbf{X}}
\newcommand{\bfY}{\mathbf{Y}}
\newcommand{\bfZ}{\mathbf{Z}}
\renewcommand{\O}{\Omega}
\renewcommand{\o}{\omega}
\newcommand{\vp}{\varphi}
\newcommand{\vep}{\varepsilon}
\newcommand{\diag}{{\rm diag}}
\newcommand{\grp}{\mathcal G}
\newcommand{\dgrp}{{\mathsf{D}}}
\newcommand{\desp}{{\mathsf{D}^{\rm{es}}}}
\newcommand{\hgr}{{\mathsf{H}}}
\newcommand{\mgr}{{\mathsf{M}}}
\newcommand{\ob}{{\rm Ob}}
\newcommand{\obg}{{\rm Ob(\mathsf{G)}}}
\newcommand{\obgp}{{\rm Ob(\mathsf{G}')}}
\newcommand{\obh}{{\rm Ob(\mathsf{H})}}
\newcommand{\Osmooth}{{\Omega^{\infty}(X,*)}}
\newcommand{\grphomotop}{{\rho_2^{\square}}}
\newcommand{\grpcalp}{{\mathsf{G}(\mathcal P)}}
\newcommand{\rf}{{R_{\mathcal F}}}
\newcommand{\grplob}{{\rm glob}}
\newcommand{\loc}{{\rm loc}}
\newcommand{\TOP}{{\rm TOP}}
\newcommand{\wti}{\widetilde}
\newcommand{\what}{\widehat}
\renewcommand{\a}{\alpha}
\newcommand{\be}{\beta}
\newcommand{\de}{\delta}
\newcommand{\del}{\partial}
\newcommand{\ka}{\kappa}
\newcommand{\si}{\sigma}
\newcommand{\ta}{\tau}
\newcommand{\med}{\medbreak}
\newcommand{\medn}{\medbreak \noindent}
\newcommand{\bign}{\bigbreak \noindent}
\newcommand{\lra}{{\longrightarrow}}
\newcommand{\ra}{{\rightarrow}}
\newcommand{\rat}{{\rightarrowtail}}
\newcommand{\ovset}[1]{\overset {#1}{\ra}}
\newcommand{\ovsetl}[1]{\overset {#1}{\lra}}
\newcommand{\hr}{{\hookrightarrow}}

\begin{document}
This is a contributed topics list on quantum algebraic topology.
Online resources are also included for selected areas of QAT.

\section{Quantum Algebraic Topology (QAT)}

\emph{Quantum algebraic topology} can be described as the area of mathematical physics and physical mathematics concerned with the application of algebraic topology concepts, procedures and results to quantum theories such as quantum field theories (QFT, AQFT, locally covariant relativististic quantum theories and quantum gravity (QG)). QAT has close connections to several other fundamental fields of mathematics, such as: noncommutative geometry (NCG), the theory of categories, functors and natural transformations (TCFN), non-Abelian algebraic topology (NAAT), and higher dimensional algebra (HDA). 
 
Related references are available at the websites and documents in the following topics list.  

\subsection{QAT Topics : Online Resources}

\begin{itemize}
\item QAT: 
\PMlinkexternal{Quantum Algebraic Topology-preprint}{http://planetmath.org/?op=getobj&from=lec&id=68}.
\item Axiomatic quantum field theory (AQFT)
\item AT: 
\PMlinkexternal{Algebraic Topology papers}{http://www.bangor.ac.uk/~mas010/publicfull.htm}
\item NAAT: 
\PMlinkexternal{A first textbook on ``Non-Abelian Algebraic Topology''}{http://www.bangor.ac.uk/~mas010/nonab-a-t.html}
\item QA:  
\PMlinkname{Quantum automata and quantum supercomputers}{QuantumAutomataAndQuantumComputation2}
\item NAQAT:  
\PMlinkexternal{Non-Abelian QAT}{http://www.bangor.ac.uk/~mas010/pdffiles/BBG-158-NAC0STQG.pdf}
\item NAAT-List:
\PMlinkexternal{Category Theory, Homotopy Theory and AT Publications}{http://www.informatics.bangor.ac.uk/public/mathematics/research/preprints/07/cathom07.html}
\item SPE: 
\PMlinkexternal{Stanford Plato: Category Theory and Applications}{http://plato.stanford.edu/entries/category-theory/}
\item Abc
\end{itemize}

\begin{thebibliography}{9}
\bibitem{Abstracts1}
Abstracts1: 
\PMlinkexternal{Springer link}{http://www.springerlink.com/content/t08637r333403846/}

\bibitem{Abstracts2}
Abstracts2:  
\PMlinkexternal{Springer link}{http://www.springerlink.com/content/92r13230n3381746/}

\bibitem{ICB2k4}
CERN Preprint Archives:
\PMlinkexternal{CERN Preprints-1}{http://documents.cern.ch/cgi-bin/setlink?base=preprint&categ=ext&id=ext-2004-125}

\bibitem{ICB2k5}
CERN Articles and Preprint Archives:
\PMlinkexternal{CERN Preprints-2}{http://doc.cern.ch/archive/electronic/other/ext/ext-2004-125/Baianu_okCERNMyPreprintsJuly9k04.doc}

\end{thebibliography}


\subsection{Additional items}

\begin{enumerate}
\item QAT1
\item QAT2
\item A3
\end{enumerate}

%%%%%
%%%%%
\end{document}
