\documentclass[12pt]{article}
\usepackage{pmmeta}
\pmcanonicalname{QuantumElectrodynamics}
\pmcreated{2013-03-22 18:10:30}
\pmmodified{2013-03-22 18:10:30}
\pmowner{bci1}{20947}
\pmmodifier{bci1}{20947}
\pmtitle{quantum electrodynamics}
\pmrecord{28}{40740}
\pmprivacy{1}
\pmauthor{bci1}{20947}
\pmtype{Topic}
\pmcomment{trigger rebuild}
\pmclassification{msc}{81Q60}
\pmclassification{msc}{81Q50}
\pmclassification{msc}{55U99}
\pmclassification{msc}{81Q30}
\pmsynonym{Q.E.D}{QuantumElectrodynamics}
%\pmkeywords{quantum electrodynamics}
%\pmkeywords{Feynman diagrams}
%\pmkeywords{Schwinger theory}
%\pmkeywords{Gell-Mann and leopard spots}
\pmrelated{QuantumChromodynamicsQCD}
\pmrelated{QuantumOperatorAlgebrasInQuantumFieldTheories}
\pmrelated{QFTOrQuantumFieldTheories}
\pmrelated{QuantumSpaceTimes}
\pmrelated{RichardFeynman}
\pmrelated{Quantization}
\pmrelated{FoundationsOfQuantumFieldTheories}
\pmrelated{FoundationsOfQuantumFieldTheories}
\pmrelated{QuantumChromodynamicsQCD}
\pmdefines{electromagnetic interactions theory}

% this is the default PlanetMath preamble.  as your knowledge
% of TeX increases, you will probably want to edit this, but
% it should be fine as is for beginners.

% almost certainly you want these
\usepackage{amssymb}
\usepackage{amsmath}
\usepackage{amsfonts}


% used for TeXing text within eps files
%\usepackage{psfrag}
% need this for including graphics (\includegraphics)
%\usepackage{graphicx}
% for neatly defining theorems and propositions
%\usepackage{amsthm}
% making logically defined graphics
%%%\usepackage{xypic}

% there are many more packages, add them here as you need them

% define commands here

\begin{document}
\section{Quantum electrodynamics (\em Q.E.D)}
\PMlinkname{Q.E.D}{QEDInTheoreticalAndMathematicalPhysics} is the advanced, standard mathematical and quantum physics treatment of electromagnetic interactions through several approaches, the more advanced including the path-integral approach by Feynman, Dirac's Operator and \PMlinkname{QED}{QEDInTheoreticalAndMathematicalPhysics} Equations, thus including either Special or General Relativity formulations of electromagnetic phenomena. More recent approaches have involved spinor (Cartan and Weyl) and twistor (Penrose) representations of Quantum Hilbert spaces of quantum states and observable quantum oprators. QED results are currently at precision levels 
beyond $10^{-29}$, and thus it is one of the most precise, if not the most precise, physical theories
that however does not encompass gravity. 

\subsection{Measurements and Quantum Field Theories}
The question of measurement in quantum mechanics (QM) and
\PMlinkname{quantum field theory (QFT)}{QFTOrQuantumFieldTheories} 
has flourished for about 75 years. The intellectual stakes have been dramatically high, and the problem 
rattled the development of 20th (and 21st) century physics at the 
foundations. Up to 1955, Bohr's Copenhagen school dominated the 
terms and practice of quantum mechanics having reached (partially) 
eye--to--eye with Heisenberg on empirical grounds, although not the 
case with Einstein who was firmly opposed on grounds on 
incompleteness with respect to physical reality. Even to the
present day, the hard philosophy of this school is respected
throughout most of theoretical physics. On the other hand, post
1955, the measurement problem adopted a new lease of life when von
Neumann's beautifully formulated QM in the mathematically rigorous
context of Hilbert spaces. Measurement it was argued involved the
influence of the Schr\"odinger equation for time evolution of the
wave function $\psi$, so leading to the notion of entanglement of
states and the indeterministic reduction of the wave packet. Once
$\psi$ is determined it is possible to compute the probability of
measurable outcomes, at the same time modifying $\psi$ relative to
the probabilities of outcomes and observations eventually causes
its collapse. The well--known paradox of Schr\"odinger's cat and
the Einstein--Podolsky--Rosen (EPR) experiment are questions
mooted once dependence on reduction of the wave packet is
jettisoned, but then other interesting paradoxes have shown their
faces. Consequently, QM opened the door to other interpretations
such as `the hidden variables' and the Everett--Wheeler assigned
measurement within different worlds, theories not without their
respective shortcomings.

Arm--in--arm with the measurement problem goes a problem of 
`the right logic', for quantum mechanical/complex biological systems
and quantum gravity. It is well--known that classical Boolean
truth--valued logics are patently inadequate for quantum theory.
Logical theories founded on projections and self--adjoint
operators on Hilbert space $H$ do run in to certain problems . One
`no--go' theorem is that of Kochen--Specker (KS) which for $\dim H
\geq 3$, does not permit an evaluation (global) on a Boolean
system of `truth values'. In Butterfield and Isham (1999)--(2004)
self--adjoint operators on $H$ with purely discrete spectrum are
considered. The KS theorem is then interpreted as saying that a
particular presheaf does not admit a global section. Partial
valuations corresponding to local sections of this presheaf are
introduced, and then generalized evaluations are defined. The
latter enjoy the structure of a Heyting algebra and so comprise an
intuitionistic logic. Truth values are describable in terms of
sieve--valued maps, and the generalized evaluations are identified
as subobjects in a topos. The further relationship with interval
valuations motivates associating to the presheaf a von Neumann
algebra where the supports of states on the algebra determines
this relationship.

We turn now to another facet of quantum measurement. Note first
that QFT pure states resist description in terms of field
configurations since the former are not always physically
interpretable. Algebraic quantum field theory (AQFT) as expounded
by Roberts (2004) points to various questions raised by
considering theories of (unbounded) operator --valued
distributions and nets of von Neumann algebras. Using in part a
gauge theoretic approach, the idea is to regard two field theories
as equivalent when their associated nets of observables are
isomorphic. More specifically, AQFT considers taking (additive)
nets of field algebras over subsets of Minkowski space, which among 
other properties, enjoy Bose--Fermi commutation relations. Although 
at first glances there may be analogs with sheaf theory, theses 
analogs are severely limited. The typical net does not give rise to 
a presheaf because the relevant morphisms are in reverse. Closer 
then is to regard a net as a precosheaf, but then the additivity 
does not allow proceeding to a cosheaf structure. This may reflect 
upon some incompatibility of AQFT with those aspects of quantum 
gravity (QG) where for example sheaf--theoretic/topos approaches 
are advocated (as in e.g. Butterfield and Isham (1999)--(2004)).




%%%%%
%%%%%
\end{document}
