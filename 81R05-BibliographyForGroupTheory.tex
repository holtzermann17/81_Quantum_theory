\documentclass[12pt]{article}
\usepackage{pmmeta}
\pmcanonicalname{BibliographyForGroupTheory}
\pmcreated{2013-03-22 14:14:24}
\pmmodified{2013-03-22 14:14:24}
\pmowner{alozano}{2414}
\pmmodifier{alozano}{2414}
\pmtitle{bibliography for group theory}
\pmrecord{15}{35683}
\pmprivacy{1}
\pmauthor{alozano}{2414}
\pmtype{Bibliography}
\pmcomment{trigger rebuild}
\pmclassification{msc}{81R05}
\pmclassification{msc}{81R10}
\pmclassification{msc}{20-00}

% this is the default PlanetMath preamble.  as your knowledge
% of TeX increases, you will probably want to edit this, but
% it should be fine as is for beginners.

% almost certainly you want these
\usepackage{amssymb}
\usepackage{amsmath}
\usepackage{amsthm}
\usepackage{amsfonts}

% used for TeXing text within eps files
%\usepackage{psfrag}
% need this for including graphics (\includegraphics)
%\usepackage{graphicx}
% for neatly defining theorems and propositions
%\usepackage{amsthm}
% making logically defined graphics
%%%\usepackage{xypic}

% there are many more packages, add them here as you need them

% define commands here

\newtheorem{thm}{Theorem}
\newtheorem{defn}{Definition}
\newtheorem{prop}{Proposition}
\newtheorem{lemma}{Lemma}
\newtheorem{cor}{Corollary}

% Some sets
\newcommand{\Nats}{\mathbb{N}}
\newcommand{\Ints}{\mathbb{Z}}
\newcommand{\Reals}{\mathbb{R}}
\newcommand{\Complex}{\mathbb{C}}
\newcommand{\Rats}{\mathbb{Q}}
\begin{document}
{\bf References for Group Theory}\\
The following are excellent sources for the indicated areas in Group Theory.

{\bf Foundations, MSC 20A}
\begin{enumerate}
\item David S. Dummit, Richard M. Foote, \emph{Abstract Algebra}, John Wiley and Sons, Inc. 1999

\item I. N. Herstein, \emph{Topics in Algebra}, Xerox College Publishing, 1975.
\begin{quote}A classic undergraduate textbook introducing groups, rings, fields, and linear algebra.\end{quote}  


\item Serge Lang, \emph{Algebra}, Addison-Wesley Publishing Company, 1997. 
\begin{quote}A textbook covering a broad range of topics in algebra; some emphasis is placed on the parts of algebra most useful in algebraic geometry.  In particular, basic homological algebra and basic category theory are well covered.
\end{quote}  

\item John S. Rose, \emph{A Course on Group Theory}, Dover Publications, New York, 1994.

\item Thomas W. Hungerford, \emph{Algebra}, Springer-Verlag Graduate Texts in Mathematics 73, 1974.  
\begin{quote}
Covers all of algebra from a more classical viewpoint; when dealing with rings, the focus is on noncommutative rings.  Only two chapters on group theory are present.  A brief discussion of categories is included. 
\end{quote}
\item Joseph J. Rotman, \emph{An Introduction to the Theory of Groups}, fourth edition, Springer-Verlag Graduate Texts in Mathematics 148, 1994.  
\begin{quote}
A thorough introduction to the theory of groups.  Emphasis is on groups with no further structure; includes a chapter on computability issues and the word problem.
\end{quote}
\end{enumerate}


{\bf Applications and books on line, MSC: 81R05,81R10}

\begin{enumerate}
\item Predrag Cvitanovic,  {\em Group Theory}, 
\PMlinkexternal{on line}{http://planetmath.org/?op=getobj&from=books&id=60}
\item Michael Tinkham, {\em Group Theory and Quantum Mechanics}, 
\PMlinkexternal{on line}{http://planetmath.org/?op=getobj&from=books&id=183}
\item Morton Hamermesh, {\em Group Theory and Physical Applications},
\PMlinkexternal{on line}{http://planetmath.org/?op=getobj&from=books&id=182}
\item Emil Artin {\em Galois Theory}
\PMlinkexternal{on line}{http://planetmath.org/?op=getobj&from=books&id=25}

\end{enumerate}

%%%%%
%%%%%
\end{document}
