\documentclass[12pt]{article}
\usepackage{pmmeta}
\pmcanonicalname{LieSuperalgebra}
\pmcreated{2013-03-22 15:35:44}
\pmmodified{2013-03-22 15:35:44}
\pmowner{bci1}{20947}
\pmmodifier{bci1}{20947}
\pmtitle{Lie superalgebra}
\pmrecord{16}{37509}
\pmprivacy{1}
\pmauthor{bci1}{20947}
\pmtype{Definition}
\pmcomment{trigger rebuild}
\pmclassification{msc}{81R50}
\pmclassification{msc}{17B60}
\pmclassification{msc}{17B01}
\pmclassification{msc}{81Q60}
\pmsynonym{Lie super algebra}{LieSuperalgebra}
\pmsynonym{graded Lie algebra}{LieSuperalgebra}
%\pmkeywords{supergeometry}
%\pmkeywords{supersymmetry}
\pmrelated{CartanCalculus}
\pmrelated{Superalgebra}
\pmrelated{GradedAlgebra}
\pmrelated{LieAlgebroids}
\pmrelated{SuperfieldsSuperspace}
\pmrelated{SupersymmetryOrSupersymmetries}
\pmrelated{LieAlgebroids}
\pmrelated{JordanBanachAndJordanLieAlgebras}
\pmrelated{LieAlgebra}
\pmrelated{LieAlgebraCohomology}
\pmrelated{SuperAlgebra}
\pmrelated{CartanCalculus}
\pmrelated{QuantumGravityTheories}
\pmrelated{Fu}
\pmdefines{vector superspace}
\pmdefines{Lie superbracket}
\pmdefines{supercommutator bracket}

\endmetadata

% this is the default PlanetMath preamble.  as your knowledge
% of TeX increases, you will probably want to edit this, but
% it should be fine as is for beginners.

% almost certainly you want these
\usepackage{amssymb}
\usepackage{amsmath}
\usepackage{amsfonts}

% used for TeXing text within eps files
%\usepackage{psfrag}
% need this for including graphics (\includegraphics)
%\usepackage{graphicx}
% for neatly defining theorems and propositions
\usepackage{amsthm}
% making logically defined graphics
%%%\usepackage{xypic}

% there are many more packages, add them here as you need them

% define commands here

\theoremstyle{definition} 
\newtheorem{dfn}{Definition}

\theoremstyle{remark}
\newtheorem*{rmk}{Remark}
\newtheorem{ex}{Example} 


\newcommand{\lie}{\mathcal{L}}
\begin{document}
\begin{dfn}
  A \emph{Lie superalgebra} is a vector superspace equipped with a bilinear map
\begin{equation}
\begin{split}
[\cdot,\cdot]: V \otimes V &\rightarrow V, \\
v \otimes w &\mapsto [v, w],
\end{split}
\end{equation}
satisfying the following properties:
\begin{enumerate}
\item If $v$ and $w$ are homogeneous vectors, then $[v,w]$ is a homogeneous vector of degree $|v| + |w| \pmod 2$,
\item For any homogeneous vectors $v, w$, $[v,w] = (-1)^{|v||w| + 1} [w,v]$,
\item For any homogeneous vectors $u,v,w$, $(-1)^{|u||w|}[u, [v,w]] + (-1)^{|v||u|} [v, [w, u]] + (-1)^{|w||v|} [w,[u,v]]$ = 0.
\end{enumerate}
The map $[\cdot,\cdot]$ is called a \emph{Lie superbracket}.
\end{dfn}

\begin{ex}
A Lie algebra $V$ can be considered as a Lie superalgebra by setting $V = V_0$ and, therefore, $V_1 = \{0\}$.
\end{ex}

\begin{ex}
Any associative superalgebra $A$ has a Lie superalgebra structure where, for any homogeneous elements $a,b \in A$, the Lie superbracket is defined by the equation
\begin{equation}\label{supercomm}
[a, b] = ab - (-1)^{|a||b|}ba.
\end{equation}

 The Lie superbracket (\ref{supercomm}) is called the \emph{supercommutator bracket} on $A$.
\end{ex}



\begin{ex}
  The space of graded derivations of a supercommutative superalgebra, equipped with the supercommutator bracket, is a Lie superalgebra.
\end{ex}



\begin{dfn}
A \emph{vector superspace} is a vector space $V$ equipped with a decomposition $V = V_0 \oplus V_1$.  
\end{dfn}

Let $V = V_0 \oplus V_1$ be a vector superspace.  Then any element of $V_0$ is said to be \emph{even}, and any element of $V_1$ is said to be \emph{odd}.  By the definition of the direct sum, any element $v$ of $V$ can be uniquely written as $v = v_0 + v_1$, where $v_0 \in V_0$ and $v_1 \in V_1$.

\begin{dfn}
A vector $v \in V$ is \emph{homogeneous} of degree $i$ if $v \in V_i$ for $i = 0$ or $1$.
\end{dfn}

If $v \in V$ is homogeneous, then the degree of $v$ is denoted by $|v|$.  In other words, if $v \in V_i$, then $|v| = i$ by definition.

\begin{rmk}
The vector $0$ is homogeneous of both degree $0$ and $1$, and thus $|0|$ is not well-defined.
\end{rmk}


%%%%%
%%%%%
\end{document}
