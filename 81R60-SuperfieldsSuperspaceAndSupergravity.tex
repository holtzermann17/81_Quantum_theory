\documentclass[12pt]{article}
\usepackage{pmmeta}
\pmcanonicalname{SuperfieldsSuperspaceAndSupergravity}
\pmcreated{2013-03-22 18:17:03}
\pmmodified{2013-03-22 18:17:03}
\pmowner{bci1}{20947}
\pmmodifier{bci1}{20947}
\pmtitle{superfields, superspace and supergravity}
\pmrecord{23}{40894}
\pmprivacy{1}
\pmauthor{bci1}{20947}
\pmtype{Feature}
\pmcomment{trigger rebuild}
\pmclassification{msc}{81R60}
\pmclassification{msc}{81R50}
\pmclassification{msc}{83C47}
\pmclassification{msc}{83C75}
\pmclassification{msc}{83C45}
\pmclassification{msc}{81P05}
\pmsynonym{quantum gravity}{SuperfieldsSuperspaceAndSupergravity}
\pmsynonym{quantum space-times}{SuperfieldsSuperspaceAndSupergravity}
%\pmkeywords{superspace}
%\pmkeywords{quantum space-times}
%\pmkeywords{superfields}
%\pmkeywords{supergravity}
%\pmkeywords{supersymmetry and L-superalgebras}
\pmrelated{SupersymmetryOrSupersymmetries}
\pmrelated{NormedAlgebra}
\pmrelated{Supercategories}
\pmrelated{QuantumGravityTheories}
\pmrelated{SuperalgebroidsAndHigherDimensionalAlgebroids}
\pmrelated{AxiomaticTheoryOfSupercategories}
\pmrelated{LieSuperalgebra3}
\pmrelated{MetricSuperfields}
\pmrelated{SuperalgebroidsAndHigherDimensionalAlgebroids}
\pmdefines{superspace}
\pmdefines{superfields}
\pmdefines{supergravity}
\pmdefines{supersymmetry and L-superalgebras}

\endmetadata

% this is the default PlanetMath preamble.  as your knowledge
% of TeX increases, you will probably want to edit this, but
% it should be fine as is for beginners.

% almost certainly you want these
\usepackage{amssymb}
\usepackage{amsmath}
\usepackage{amsfonts}

% used for TeXing text within eps files
%\usepackage{psfrag}
% need this for including graphics (\includegraphics)
%\usepackage{graphicx}
% for neatly defining theorems and propositions
%\usepackage{amsthm}
% making logically defined graphics
%%%\usepackage{xypic}

% there are many more packages, add them here as you need them

% define commands here
\usepackage{amsmath, amssymb, amsfonts, amsthm, amscd, latexsym, enumerate,color}
\usepackage{xypic, xspace}
\usepackage[mathscr]{eucal}
\usepackage[dvips]{graphicx}
\usepackage[curve]{xy}

\setlength{\textwidth}{6.5in}
%\setlength{\textwidth}{16cm}
\setlength{\textheight}{9.0in}
%\setlength{\textheight}{24cm}

\hoffset=-.75in     %%ps format
%\hoffset=-1.0in     %%hp format
\voffset=-.4in
\def\blue{\textcolor{blue}}

\theoremstyle{plain}
\newtheorem{lemma}{Lemma}[section]
\newtheorem{proposition}{Proposition}[section]
\newtheorem{theorem}{Theorem}[section]
\newtheorem{corollary}{Corollary}[section]

\theoremstyle{definition}
\newtheorem{definition}{Definition}[section]
\newtheorem{example}{Example}[section]
%\theoremstyle{remark}
\newtheorem{remark}{Remark}[section]
\newtheorem*{notation}{Notation}
\newtheorem*{claim}{Claim}

\renewcommand{\thefootnote}{\ensuremath{\fnsymbol{footnote}}}
\numberwithin{equation}{section}

\newcommand{\Ad}{{\rm Ad}}
\newcommand{\Aut}{{\rm Aut}}
\newcommand{\Cl}{{\rm Cl}}
\newcommand{\Co}{{\rm Co}}
\newcommand{\DES}{{\rm DES}}
\newcommand{\Diff}{{\rm Diff}}
\newcommand{\Dom}{{\rm Dom}}
\newcommand{\Hol}{{\rm Hol}}
\newcommand{\Mon}{{\rm Mon}}
\newcommand{\Hom}{{\rm Hom}}
\newcommand{\Ker}{{\rm Ker}}
\newcommand{\Ind}{{\rm Ind}}
\newcommand{\IM}{{\rm Im}}
\newcommand{\Is}{{\rm Is}}
\newcommand{\ID}{{\rm id}}
\newcommand{\grpL}{{\rm GL}}
\newcommand{\Iso}{{\rm Iso}}
\newcommand{\rO}{{\rm O}}
\newcommand{\Sem}{{\rm Sem}}
\newcommand{\SL}{{\rm Sl}}
\newcommand{\St}{{\rm St}}
\newcommand{\Sym}{{\rm Sym}}
\newcommand{\Symb}{{\rm Symb}}
\newcommand{\SU}{{\rm SU}}
\newcommand{\Tor}{{\rm Tor}}
\newcommand{\U}{{\rm U}}

\newcommand{\A}{\mathcal A}
\newcommand{\Ce}{\mathcal C}
\newcommand{\D}{\mathcal D}
\newcommand{\E}{\mathcal E}
\newcommand{\F}{\mathcal F}
%\newcommand{\grp}{\mathcal G}
\renewcommand{\H}{\mathcal H}
\renewcommand{\cL}{\mathcal L}
\newcommand{\Q}{\mathcal Q}
\newcommand{\R}{\mathcal R}
\newcommand{\cS}{\mathcal S}
\newcommand{\cU}{\mathcal U}
\newcommand{\W}{\mathcal W}

\newcommand{\bA}{\mathbb{A}}
\newcommand{\bB}{\mathbb{B}}
\newcommand{\bC}{\mathbb{C}}
\newcommand{\bD}{\mathbb{D}}
\newcommand{\bE}{\mathbb{E}}
\newcommand{\bF}{\mathbb{F}}
\newcommand{\bG}{\mathbb{G}}
\newcommand{\bK}{\mathbb{K}}
\newcommand{\bM}{\mathbb{M}}
\newcommand{\bN}{\mathbb{N}}
\newcommand{\bO}{\mathbb{O}}
\newcommand{\bP}{\mathbb{P}}
\newcommand{\bR}{\mathbb{R}}
\newcommand{\bV}{\mathbb{V}}
\newcommand{\bZ}{\mathbb{Z}}

\newcommand{\bfE}{\mathbf{E}}
\newcommand{\bfX}{\mathbf{X}}
\newcommand{\bfY}{\mathbf{Y}}
\newcommand{\bfZ}{\mathbf{Z}}

\renewcommand{\O}{\Omega}
\renewcommand{\o}{\omega}
\newcommand{\vp}{\varphi}
\newcommand{\vep}{\varepsilon}

\newcommand{\diag}{{\rm diag}}
\newcommand{\grp}{\mathcal G}
\newcommand{\dgrp}{{\mathsf{D}}}
\newcommand{\desp}{{\mathsf{D}^{\rm{es}}}}
\newcommand{\grpeod}{{\rm Geod}}
%\newcommand{\grpeod}{{\rm geod}}
\newcommand{\hgr}{{\mathsf{H}}}
\newcommand{\mgr}{{\mathsf{M}}}
\newcommand{\ob}{{\rm Ob}}
\newcommand{\obg}{{\rm Ob(\mathsf{G)}}}
\newcommand{\obgp}{{\rm Ob(\mathsf{G}')}}
\newcommand{\obh}{{\rm Ob(\mathsf{H})}}
\newcommand{\Osmooth}{{\Omega^{\infty}(X,*)}}
\newcommand{\grphomotop}{{\rho_2^{\square}}}
\newcommand{\grpcalp}{{\mathsf{G}(\mathcal P)}}

\newcommand{\rf}{{R_{\mathcal F}}}
\newcommand{\grplob}{{\rm glob}}
\newcommand{\loc}{{\rm loc}}
\newcommand{\TOP}{{\rm TOP}}

\newcommand{\wti}{\widetilde}
\newcommand{\what}{\widehat}

\renewcommand{\a}{\alpha}
\newcommand{\be}{\beta}
\newcommand{\grpa}{\grpamma}
%\newcommand{\grpa}{\grpamma}
\newcommand{\de}{\delta}
\newcommand{\del}{\partial}
\newcommand{\ka}{\kappa}
\newcommand{\si}{\sigma}
\newcommand{\ta}{\tau}

\newcommand{\med}{\medbreak}
\newcommand{\medn}{\medbreak \noindent}
\newcommand{\bign}{\bigbreak \noindent}

\newcommand{\lra}{{\longrightarrow}}
\newcommand{\ra}{{\rightarrow}}
\newcommand{\rat}{{\rightarrowtail}}
\newcommand{\ovset}[1]{\overset {#1}{\ra}}
\newcommand{\ovsetl}[1]{\overset {#1}{\lra}}
\newcommand{\hr}{{\hookrightarrow}}

\newcommand{\<}{{\langle}}

\def\baselinestretch{1.1}


\hyphenation{prod-ucts}

%\grpeometry{textwidth= 16 cm, textheight=21 cm}

\newcommand{\sqdiagram}[9]{$$ \diagram  #1  \rto^{#2} \dto_{#4}&
#3  \dto^{#5} \\ #6    \rto_{#7}  &  #8   \enddiagram
\eqno{\mbox{#9}}$$ }

\def\C{C^{\ast}}

\newcommand{\labto}[1]{\stackrel{#1}{\longrightarrow}}

%\newenvironment{proof}{\noindent {\bf Proof} }{ \hfill $\Box$
%{\mbox{}}
\newcommand{\midsqn}[1]{\ar@{}[dr]|{#1}}
\newcommand{\quadr}[4]
{\begin{pmatrix} & #1& \\[-1.1ex] #2 & & #3\\[-1.1ex]& #4&
 \end{pmatrix}}
\def\D{\mathsf{D}}
\begin{document}
\subsection{Superspace, superfields, supergravity and Lie superalgebras.} 
\bigbreak

In general, a \emph{superfield}--or \emph{quantized gravity field}- has a highly reducible representation of the supersymmetry algebra, and the problem of specifying a supergravity theory can be defined as a search for those 
representations that allow the construction of consistent local actions, perhaps considered 
as either quantum group, or quantum groupoid, actions. Extending quantum symmetries to include
quantized gravity fields--specified as `superfields'-- is called \emph{supersymmetry} in current theories of Quantum Gravity. Graded `Lie' algebras (or Lie superalgebras) represent the quantum operator supersymmetries
by defining these simultaneously for both \emph{fermion} (spin $1/2$) and \emph{boson} (integer or 0 spin particles).

The quantized physical space with supersymmetric properties is then called a \emph{`superspace'},
(another name for \emph{`quantized space with supersymmetry'}) in Quantum Gravity. The following subsection defines
these physical concepts in precise mathematical terms.


\subsubsection{Mathematical definitions and propagation equations for superfields in superspace:
Graded Lie algebras}

Supergravity, in essence, is an extended supersymmetric theory of
both matter and gravitation ({\em viz}. Weinberg, 1995 \cite{Weinberg2000}). 
A first approach to supersymmetry relied on a curved `superspace' 
(Wess and Bagger,1983 \cite{WB83}) and is analogous to supersymmetric gauge theories (see, for
example, Sections 27.1 to 27.3 of Weinberg, 1995). Unfortunately,
a complete non--linear supergravity theory might be forbiddingly
complicated and furthermore, the constraints that need be made on
the graviton superfield appear somewhat subjective, 
(according to Weinberg, 1995). In a different approach to supergravity, 
one considers the physical components of the gravitational superfield
which can be then identified based on `flat-space' superfield methods 
(Chs. 26 and 27 of Weinberg, 1995). By implementing the {\em gravitational
weak-field approximation} one obtains several of the most important
consequences of supergravity theory, including masses for the
hypothetical `gravitino' and `gaugino particles' whose existence might be
expected from supergravity theories. Furthermore, by adding on the
higher order terms in the gravitational constant to the
supersymmetric transformation, the general coordinate
transformations form a \emph{closed algebra} and the Lagrangian that
describes the interactions of the physical fields is then {\em invariant}
under such transformations.The first quantization of such a flat-space
superfield would obviously involve its `deformation', and as a result its corresponding
\emph{supersymmetry algebra} becomes \emph{non--commutative}.

\subsubsection{Metric superfield}

Because in supergravity both spinor and tensor fields are being
considered, the gravitational fields are represented in terms of
\emph{tetrads}, $e^a_\mu(x),$ rather than in terms of Einstein's general
relativistic metric $g_{\mu \nu}(x)$. The connections between
these two distinct representations are as follows:

\begin{equation}
g_{\mu\nu}(x) = \eta_{ab}~ e^a_\mu (x)e^b_\gamma(x)~,
\end{equation}

with the general coordinates being indexed by $\mu,\nu,$ etc.,
whereas local coordinates that are being defined in a locally
inertial coordinate system are labeled with superscripts a, b,
etc.;   $ \eta_{ab}$ is the diagonal matrix with elements +1, +1,
+1 and -1. The tetrads are invariant to two distinct types of
symmetry transformations--the local Lorentz transformations:
\begin{equation}
e^a_\mu (x)\longmapsto \Lambda^a_b (x) e^b_\mu (x)~,
\end{equation}
(where $\Lambda^a_b$ is an arbitrary real matrix), and the general
coordinate transformations:
\begin{equation}
x^\mu \longmapsto (x')^\mu(x) ~.
\end{equation}
In a weak gravitational field the tetrad may be represented as:
\begin{equation}
e^a_\mu (x)=\delta^a_\mu(x)+ 2\kappa \Phi^a_\mu (x)~,
\end{equation}
where $\Phi^a_\mu(x)$ is small compared with $\delta^a_\mu(x)$ for
all $x$ values, and $\kappa= \surd 8\pi G$, where G is Newton's
gravitational constant. As it will be discussed next, the
supersymmetry algebra (SA) implies that the graviton has a
fermionic superpartner, the hypothetical \emph{`gravitino'}, with
helicities $\pm$ 3/2. Such a self-charge-conjugate massless
particle as the `gravitiono' with helicities $\pm$ 3/2 can only have
\emph{low-energy} interactions if it is represented by a Majorana
field $\psi _\mu(x)$ which is invariant under the gauge
transformations:
\begin{equation}
\psi _\mu(x)\longmapsto \psi _\mu(x)+\delta _\mu \psi(x) ~,
\end{equation}

with $\psi(x)$ being an arbitrary Majorana field as defined by
Grisaru and Pendleton (1977). The tetrad field $\Phi _{\mu
\nu}(x)$ and the graviton field $\psi _\mu(x)$ are then
incorporated into a term $H_\mu (x,\theta)$ defined as the
\emph{metric superfield}. The relationships between $\Phi _{\mu _
\nu}(x)$ and $\psi _\mu(x)$, on the one hand, and the components
of the metric superfield $H_\mu (x,\theta)$, on the other hand,
can be derived from the transformations of the whole metric
superfield:
\begin{equation}
H_\mu (x,\theta)\longmapsto H_\mu (x,\theta)+ \Delta _\mu
(x,\theta)~,
\end{equation}
by making the simplifying-- and physically realistic-- assumption
of a weak gravitational field (further details can be found, for
example, in Ch.31 of vol.3. of Weinberg, 1995). The interactions
of the entire superfield $H_\mu (x)$ with matter would be then
described by considering how a weak gravitational field,
$h_{\mu_\nu}$ interacts with an energy-momentum tensor $T^{\mu
\nu}$ represented as a linear combination of components of a real
vector superfield $\Theta^\mu$.  Such interaction terms would,
therefore, have the form:
\begin{equation}
 I_{\mathcal M}= 2\kappa \int dx^4 [H_\mu \Theta^\mu]_D ~,
\end{equation}
($\mathcal M$ denotes `matter') integrated over a four-dimensional
(Minkowski) spacetime with the metric defined by the superfield
$H_\mu (x,\theta)$. The term $\Theta^\mu$, as defined above, is
physically a \emph{supercurrent} and satisfies the conservation
conditions:
\begin{equation}
\gamma^\mu \mathbf{D} \Theta _\mu = \mathbf{D} ~,
\end{equation}
where $\mathbf{D}$ is the four-component super-derivative and $X$
denotes a real chiral scalar superfield. This leads immediately to
the calculation of the interactions of matter with a weak
gravitational field as:
\begin{equation}
I_{\mathcal M} = \kappa \int d^4 x T^{\mu \nu}(x)h_{\mu \nu}(x) ~,
\end{equation}
It is interesting to note that the gravitational actions for the
superfield that are invariant under the generalized gauge
transformations $H_\mu \longmapsto H _\mu  + \Delta _\mu$ lead to
solutions of the Einstein field equations for a homogeneous,
non-zero vacuum energy density $\rho _V$ that correspond to either
a de Sitter space for $\rho _V>0$, or an anti-de Sitter space for
$\rho _V <0$. Such spaces can be represented in terms of the
hypersurface equation
\begin{equation}
x^2_5 \pm \eta _{\mu,\nu} x^\mu x^\nu = R^2 ~,
\end{equation}
in a {\em quasi-Euclidean five-dimensional space} with the metric
specified as:
\begin{equation}
ds^2 = \eta _{\mu,\nu} x^\mu x^\nu \pm dx^2_5 ~,
\end{equation}
with '$+$' for de Sitter space and '$-$' for anti-de Sitter space,
respectively.

\med
The spacetime symmetry groups, or extended symmetry groupoids, as the case may
be-- are different from the `classical' Poincar\'e symmetry group
of translations and Lorentz transformations. Such spacetime
symmetry groups, in the simplest case, are therefore the
$\rO(4,1)$ group for the \emph{de Sitter space} and the $\rO(3,2)$ group
for the \emph{anti--de Sitter space}. A detailed calculation indicates
that the transition from ordinary flat space to a bubble of
anti-de Sitter space is \emph{not} favored energetically and,
therefore, the ordinary (de Sitter) flat space is stable (viz.
Coleman and De Luccia, 1980), even though quantum fluctuations
might occur to an anti--de Sitter bubble within the limits
permitted by the Heisenberg uncertainty principle.

\subsection {Supersymmetry algebras and Lie (graded) superalgebras.}

It is well known that \emph{continuous symmetry transformations}
can be represented in terms of a \emph{Lie algebra} of linearly
independent \emph{symmetry generators} $t_j$ that satisfy the
commutation relations:

\begin{equation}
[t_j,t_k] = \iota \Sigma_l C_{jk} t_l ~,
\end{equation}

Supersymmetry is similarly expressed in terms of the symmetry
generators $t_j$ of a \textit{graded (`Lie') algebra} which is in
fact defined as a \textit{superalgebra}) by satisfying relations of the
general form:
\begin{equation}
t_j t_k - (-1)^{\eta _j \eta _k} t_k  t_j = \iota \Sigma_l C_{jk}
^l t_l ~.
\end{equation}
The generators for which $\eta _j =1$ are fermionic whereas those
for which $\eta _j =0$ are bosonic. The coefficients $C^l_{jk}$
are structure constants satisfying the following conditions:
\begin{equation}
C _{jk} ^l = -(-1)^{\eta _j \eta _k} C _{jk} ^l ~.
\end{equation}
If the generators $ _j$ are quantum Hermitian operators, then the
structure constants satisfy the reality conditions $C_{jk}^* = -
C_{jk}$~.  Clearly, such a graded algebraic structure is a superalgebra
and not a proper Lie algebra; thus graded Lie algebras are often called
\textit{`Lie superalgebras'}.

\med
The standard computational approach in QM utilizes the S-matrix
approach, and therefore, one needs to consider the general,
\emph{graded} `Lie algebra' of \emph{supersymmetry generators} that
commute with the S-matrix. If one denotes the fermionic generators
by $Q$, then $U^{-1}(\Lambda)Q U(\Lambda)$ will also be of the
same type when $U(\Lambda)$ is the quantum operator corresponding
to arbitrary, homogeneous Lorentz transformations $\Lambda^{\mu
_\nu}$~. Such a group of generators provide therefore a
representation of the homogeneous Lorentz group of transformations
$ \mathbb{L}$~. The irreducible representation of the homogeneous
Lorentz group of transformations provides therefore a
classification of such individual generators.

\subsubsection{Graded `Lie Algebras'/Superalgebras.}

A set of quantum operators $Q^{AB}_{jk}$ form an $\mathbf A,
\mathbf B$ representation of the group $\mathbf L$ defined above
which satisfy the commutation relations:

\begin{equation}
[\mathbf{A},Q^{AB}_{jk}] = -[\Sigma _j' J^A _{j j'}, Q^{AB}_{j'k}]
~,
\end{equation}
and

\begin{equation}
[\mathbf{B},Q^{AB}_{jk}] = -[\Sigma _{j'} J^A _{k k'},
Q^{AB}_{jk'}] ~,
\end{equation}
with the generators $\mathbf{A}$ and $\mathbf{B}$ defined by
$\mathbf{A}\equiv (1/2)(\mathbf{J} \pm i\mathbf{K})$ and
$\mathbf{B} \equiv (1/2)(\mathbf{J }- i\mathbf{K})$, with
$\mathbf{J}$ and $\mathbf{K}$ being the Hermitian generators of
rotations and `boosts', respectively.

\med
In the case of the two-component Weyl-spinors $Q _{jr}$ the
Haag--Lopuszanski--Sohnius (HLS) theorem applies, and thus the
fermions form a \emph{supersymmetry algebra} defined by the
anti-commutation relations:
\begin{equation}
\begin{aligned}
~[Q _{jr}, Q _{ks}^*] &= 2\delta _{rs} \sigma^\mu _{jk} P _\mu ~,
\\ [Q _{jr}, Q _{ks}] &= e _{jk} Z _{rs} ~,
\end{aligned}
\end{equation}
where $P _\mu$ is the 4--momentum operator, $Z_{rs} = -Z _{s r}$
are the bosonic symmetry generators, and $\sigma _\mu$ and
$\mathbf{e}$ are the usual $2 \times 2$ Pauli matrices.
Furthermore, the fermionic generators commute with both energy and
momentum operators:
\begin{equation}
[P _\mu,Q _{jr}] = [P _\mu, Q^* _{jr}] = 0 ~.
\end{equation}
The bosonic symmetry generators $Z _{ks}$ and $Z^* _{ks}$
represent the set of \emph{central charges} of the supersymmetric
algebra:
\begin{equation}
~[Z _{rs}, Z^* _{tn}] = [Z^* _{rs}, Q _{jt}]=  [Z^* _{rs}, Q^*
_{jt}]= [Z^* _{rs}, Z^* _{tn}]=0 ~.
\end{equation}
From another direction, the Poincar\'e symmetry mechanism of
special relativity can be extended to new algebraic systems
(Tanas\u a, 2006). In Moultaka et al. (2005) in view of such
extensions, consider invariant-free Lagrangians and bosonic
multiplets  constituting a symmetry that interplays with (Abelian)
$\U(1)$--gauge symmetry that may possibly be described in
categorical terms, in particular, within the notion of a
\emph{cubical site} (Grandis and Mauri, 2003).

We shall proceed to introduce in the next section generalizations
of the concepts of Lie algebras and graded Lie algebras to the
corresponding Lie \emph{algebroids} that may also be regarded as
C*--convolution representations of \emph{quantum gravity
groupoids} and superfield (or supergravity) supersymmetries. This
is therefore a novel approach to the proper representation of the
\emph{non-commutative geometry of quantum spacetimes}--that are
\emph{curved} (or `deformed') by the presence of \emph{intense}
gravitational fields--in the framework of \emph{non-Abelian,
graded Lie algebroids}. Their correspondingly \emph{deformed
quantum gravity groupoids} (QGG) should, therefore, adequately
represent supersymmetries modified by the presence of such intense
gravitational fields on the Planck scale. Quantum fluctuations
that give rise to quantum `foams' at the Planck scale may be then
represented by \emph{quantum homomorphisms} of such QGGs. If the
corresponding graded Lie algebroids are also \emph{integrable},
then one can reasonably expect to recover in the limit of $\hbar
\rightarrow 0$ the Riemannian geometry of General Relativity and
the \emph{globally hyperbolic spacetime} of Einstein's classical
gravitation theory (GR), as a result of such an integration to the
\emph{quantum gravity fundamental groupoid} (QGFG). The following
subsection will define the precise mathematical concepts
underlying our novel quantum supergravity and extended
supersymmetry notions.

\begin{thebibliography}{9}
\bibitem{Weinberg2000}
S. Weinberg.: \emph{The Quantum Theory of Fields}. Cambridge, New York and Madrid: 
Cambridge University Press, Vols. 1 to 3, (1995--2000).

\bibitem{Weinstein}
A. Weinstein : Groupoids: unifying internal and external symmetry,
\emph{Notices of the Amer. Math. Soc.} \textbf{43} (7): 744-752 (1996).

\bibitem{WB83}
J. Wess and J. Bagger: \emph{Supersymmetry and Supergravity},
Princeton University Press, (1983).
\end{thebibliography}
%%%%%
%%%%%
\end{document}
