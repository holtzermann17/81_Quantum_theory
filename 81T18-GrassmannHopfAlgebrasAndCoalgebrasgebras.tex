\documentclass[12pt]{article}
\usepackage{pmmeta}
\pmcanonicalname{GrassmannHopfAlgebrasAndCoalgebrasgebras}
\pmcreated{2013-03-22 18:10:55}
\pmmodified{2013-03-22 18:10:55}
\pmowner{bci1}{20947}
\pmmodifier{bci1}{20947}
\pmtitle{Grassmann-Hopf algebras and coalgebras\gebras}
\pmrecord{52}{40754}
\pmprivacy{1}
\pmauthor{bci1}{20947}
\pmtype{Topic}
\pmcomment{trigger rebuild}
\pmclassification{msc}{81T18}
\pmclassification{msc}{81T13}
\pmclassification{msc}{55Q25}
\pmclassification{msc}{81T10}
\pmclassification{msc}{16W30}
\pmclassification{msc}{81T05}
\pmclassification{msc}{57T05}
\pmclassification{msc}{15A75}
\pmsynonym{tangled-dual Grassmann-Hopf co-algebra}{GrassmannHopfAlgebrasAndCoalgebrasgebras}
%\pmkeywords{observable operator algebras encountered in QFT}
%\pmkeywords{Grassman-Hopf algebras}
%\pmkeywords{tangled-dual Grassman-Hopf co-algebras}
%\pmkeywords{quantum operator algebras}
%\pmkeywords{advanced QAT or quantum algebraic topology}
\pmrelated{QED}
\pmrelated{WeakHopfCAlgebra2}
\pmrelated{WeakHopfCAlgebra}
\pmrelated{DualOfACoalgebraIsAnAlgebra}
\pmrelated{CAlgebra3}
\pmrelated{TopicEntryOnTheAlgebraicFoundationsOfQuantumAlgebraicTopology}
\pmrelated{AlgebraicFoundationsOfQuantumAlgebraicTopology}
\pmrelated{QuantumAlgebraicTopologyOfCWComplexRepresentationsNewQATResult}
\pmdefines{Grassmann-Hopf algebra}
\pmdefines{dual Grassmann-Hopf co-algebra and gebra or tangled algebra}
\pmdefines{observable operator algebra}
\pmdefines{Grassman-Hopf algebroid}

\endmetadata

% this is the default PlanetMath preamble.  as your knowledge
% of TeX increases, you will probably want to edit this, but
% it should be fine as is for beginners.

% almost certainly you want these
\usepackage{amssymb}
\usepackage{amsmath}
\usepackage{amsfonts}

% used for TeXing text within eps files
%\usepackage{psfrag}
% need this for including graphics (\includegraphics)
%\usepackage{graphicx}
% for neatly defining theorems and propositions
%\usepackage{amsthm}
% making logically defined graphics
%%%\usepackage{xypic}

% there are many more packages, add them here as you need them

% define commands here
\usepackage{amsmath, amssymb, amsfonts, amsthm, amscd, latexsym, enumerate}
\usepackage{xypic, xspace}
\usepackage[mathscr]{eucal}
\usepackage[dvips]{graphicx}
\usepackage[curve]{xy}

\setlength{\textwidth}{6.5in}
%\setlength{\textwidth}{16cm}
\setlength{\textheight}{9.0in}
%\setlength{\textheight}{24cm}

\hoffset=-.75in     %%ps format
%\hoffset=-1.0in     %%hp format
\voffset=-.4in


\theoremstyle{plain}
\newtheorem{lemma}{Lemma}[section]
\newtheorem{proposition}{Proposition}[section]
\newtheorem{theorem}{Theorem}[section]
\newtheorem{corollary}{Corollary}[section]

\theoremstyle{definition}
\newtheorem{definition}{Definition}[section]
\newtheorem{example}{Example}[section]
%\theoremstyle{remark}
\newtheorem{remark}{Remark}[section]
\newtheorem*{notation}{Notation}
\newtheorem*{claim}{Claim}

\renewcommand{\thefootnote}{\ensuremath{\fnsymbol{footnote}}}
\numberwithin{equation}{section}

\newcommand{\Ad}{{\rm Ad}}
\newcommand{\Aut}{{\rm Aut}}
\newcommand{\Cl}{{\rm Cl}}
\newcommand{\Co}{{\rm Co}}
\newcommand{\DES}{{\rm DES}}
\newcommand{\Diff}{{\rm Diff}}
\newcommand{\Dom}{{\rm Dom}}
\newcommand{\Hol}{{\rm Hol}}
\newcommand{\Mon}{{\rm Mon}}
\newcommand{\Hom}{{\rm Hom}}
\newcommand{\Ker}{{\rm Ker}}
\newcommand{\Ind}{{\rm Ind}}
\newcommand{\IM}{{\rm Im}}
\newcommand{\Is}{{\rm Is}}
\newcommand{\ID}{{\rm id}}
\newcommand{\grpL}{{\rm GL}}
\newcommand{\Iso}{{\rm Iso}}
\newcommand{\rO}{{\rm O}}
\newcommand{\Sem}{{\rm Sem}}
\newcommand{\SL}{{\rm Sl}}
\newcommand{\St}{{\rm St}}
\newcommand{\Sym}{{\rm Sym}}
\newcommand{\Symb}{{\rm Symb}}
\newcommand{\SU}{{\rm SU}}
\newcommand{\Tor}{{\rm Tor}}
\newcommand{\U}{{\rm U}}

\newcommand{\A}{\mathcal A}
\newcommand{\Ce}{\mathcal C}
\newcommand{\D}{\mathcal D}
\newcommand{\E}{\mathcal E}
\newcommand{\F}{\mathcal F}
%\newcommand{\grp}{\mathcal G}
\renewcommand{\H}{\mathcal H}
\renewcommand{\cL}{\mathcal L}
\newcommand{\Q}{\mathcal Q}
\newcommand{\R}{\mathcal R}
\newcommand{\cS}{\mathcal S}
\newcommand{\cU}{\mathcal U}
\newcommand{\W}{\mathcal W}

\newcommand{\bA}{\mathbb{A}}
\newcommand{\bB}{\mathbb{B}}
\newcommand{\bC}{\mathbb{C}}
\newcommand{\bD}{\mathbb{D}}
\newcommand{\bE}{\mathbb{E}}
\newcommand{\bF}{\mathbb{F}}
\newcommand{\bG}{\mathbb{G}}
\newcommand{\bK}{\mathbb{K}}
\newcommand{\bM}{\mathbb{M}}
\newcommand{\bN}{\mathbb{N}}
\newcommand{\bO}{\mathbb{O}}
\newcommand{\bP}{\mathbb{P}}
\newcommand{\bR}{\mathbb{R}}
\newcommand{\bV}{\mathbb{V}}
\newcommand{\bZ}{\mathbb{Z}}

\newcommand{\bfE}{\mathbf{E}}
\newcommand{\bfX}{\mathbf{X}}
\newcommand{\bfY}{\mathbf{Y}}
\newcommand{\bfZ}{\mathbf{Z}}

\renewcommand{\O}{\Omega}
\renewcommand{\o}{\omega}
\newcommand{\vp}{\varphi}
\newcommand{\vep}{\varepsilon}

\newcommand{\diag}{{\rm diag}}
\newcommand{\grp}{{\mathsf{G}}}
\newcommand{\dgrp}{{\mathsf{D}}}
\newcommand{\desp}{{\mathsf{D}^{\rm{es}}}}
\newcommand{\grpeod}{{\rm Geod}}
%\newcommand{\grpeod}{{\rm geod}}
\newcommand{\hgr}{{\mathsf{H}}}
\newcommand{\mgr}{{\mathsf{M}}}
\newcommand{\ob}{{\rm Ob}}
\newcommand{\obg}{{\rm Ob(\mathsf{G)}}}
\newcommand{\obgp}{{\rm Ob(\mathsf{G}')}}
\newcommand{\obh}{{\rm Ob(\mathsf{H})}}
\newcommand{\Osmooth}{{\Omega^{\infty}(X,*)}}
\newcommand{\grphomotop}{{\rho_2^{\square}}}
\newcommand{\grpcalp}{{\mathsf{G}(\mathcal P)}}

\newcommand{\rf}{{R_{\mathcal F}}}
\newcommand{\grplob}{{\rm glob}}
\newcommand{\loc}{{\rm loc}}
\newcommand{\TOP}{{\rm TOP}}

\newcommand{\wti}{\widetilde}
\newcommand{\what}{\widehat}

\renewcommand{\a}{\alpha}
\newcommand{\be}{\beta}
\newcommand{\grpa}{\grpamma}
%\newcommand{\grpa}{\grpamma}
\newcommand{\de}{\delta}
\newcommand{\del}{\partial}
\newcommand{\ka}{\kappa}
\newcommand{\si}{\sigma}
\newcommand{\ta}{\tau}

\newcommand{\med}{\medbreak}
\newcommand{\medn}{\medbreak \noindent}
\newcommand{\bign}{\bigbreak \noindent}

\newcommand{\lra}{{\longrightarrow}}
\newcommand{\ra}{{\rightarrow}}
\newcommand{\rat}{{\rightarrowtail}}
\newcommand{\ovset}[1]{\overset {#1}{\ra}}
\newcommand{\ovsetl}[1]{\overset {#1}{\lra}}
\newcommand{\hr}{{\hookrightarrow}}

\newcommand{\<}{{\langle}}

%\newcommand{\>}{{\rangle}}

%\usepackage{geometry, amsmath,amssymb,latexsym,enumerate}
%%%\usepackage{xypic}

\def\baselinestretch{1.1}


\hyphenation{prod-ucts}

%\grpeometry{textwidth= 16 cm, textheight=21 cm}

\newcommand{\sqdiagram}[9]{$$ \diagram  #1  \rto^{#2} \dto_{#4}&
#3  \dto^{#5} \\ #6    \rto_{#7}  &  #8   \enddiagram
\eqno{\mbox{#9}}$$ }

\def\C{C^{\ast}}

\newcommand{\labto}[1]{\stackrel{#1}{\longrightarrow}}

%\newenvironment{proof}{\noindent {\bf Proof} }{ \hfill $\Box$
%{\mbox{}}

\newcommand{\quadr}[4]
{\begin{pmatrix} & #1& \\[-1.1ex] #2 & & #3\\[-1.1ex]& #4&
 \end{pmatrix}}
\def\D{\mathsf{D}}
\begin{document}
\subsection{Definitions of Grassmann-Hopf Al/gebras, Their Dual \\
Co-Algebras, and Grassmann--Hopf Al/gebroids}

 Let $V$ be a (complex) vector space, $\dim_{\mathcal C} V = n$, and let $\{e_0, e_1, \ldots, \}$ with identity $e_0 \equiv 1$, be the generators of a Grassmann (exterior) algebra

\begin{equation}
\Lambda^*V = \Lambda^0 V \oplus \Lambda^1 V \oplus \Lambda^2 V
\oplus  \cdots
\end{equation}
subject to the relation $e_i e_j + e_j e_i = 0$~. Following Fauser
(2004) we append this algebra with a Hopf structure to obtain a
`co--gebra' based on the interchange (or \textsl{`tangled duality'}): \\

$$\text{(\textit{objects/points}, \textit{morphisms})} \mapsto \text{(\textsl{morphisms}, \textsl{objects/points.})}$$

 This leads to a \textsl{tangle duality} between an associative (unital algebra) 
$\A=(A,m)$, and an associative (unital) `co--gebra' $\mathcal{C}=(C,\Delta)$ :

\begin{itemize}
\item[i] the binary product $A \otimes A \ovsetl{m} A$, and
\item[ii] the coproduct $C \ovsetl{\Delta} C \otimes C$ \end{itemize}, 
where the Sweedler notation (Sweedler, 1996), with respect to an
arbitrary basis is adopted: $$
\begin{aligned}
\Delta (x) &= \sum_r a_r \otimes b_r = \sum_{(x)} x_{(1)} \otimes
x_{(2)} = x _{(1)} \otimes x_{(2)} \\ \Delta (x^i) &= \sum_i
\Delta^{jk}_i = \sum_{(r)} a^j_{(r)} \otimes b^k_{(r)} = x _{(1)}
\otimes x_{(2)}
\end{aligned}
$$

Here the $\Delta^{jk}_i$ are called `section coefficients'. We have then a generalization of associativity to coassociativity:
\begin{equation}
\begin{CD}
C  @> \Delta >> C \otimes C
\\ @VV \Delta V   @VV \ID \otimes \Delta V  \\ C \otimes C
 @> \Delta \otimes \ID >> C \otimes C \otimes C
\end{CD}
\end{equation}
inducing a tangled duality between an associative (unital algebra
$\mathcal A = (A,m)$, and an associative (unital) `co--gebra'
$\mathcal C = (C, \Delta)$~. The idea is to take this structure
and combine the Grassmann algebra $(\Lambda^*V, \wedge)$ with the
`co-gebra' $(\Lambda^*V, \Delta_{\wedge})$ (the `tangled dual')
along with the Hopf algebra compatibility rules: 1) the product
and the unit are `co--gebra' morphisms, and 2) the coproduct and
counit are algebra morphisms.


Next we consider the following ingredients:

\begin{itemize}
\item[(1)]
the graded switch $\hat{\tau} (A \otimes B) = (-1)^{\del
A \del B} B \otimes A$
\item[(2)]
the counit $\varepsilon$ (an algebra morphism) satisfying
$(\varepsilon \otimes \ID) \Delta = \ID = (\ID \otimes
\varepsilon) \Delta$


\item[(3)] the antipode $S$~.
\end{itemize}

The \textit{Grassmann-Hopf algebra} $\widehat{H}$ thus consists of--is defined by-- the
\textit{septet} $\widehat{H}=(\Lambda^*V, \wedge, \ID, \varepsilon, \hat{\tau},S)~$.

Its generalization to a \textit{Grassmann-Hopf algebroid} is
straightforward by considering a groupoid $\grp$, and then
defining a $H^{\wedge}- \textit{Algebroid}$ as a
\textit{quadruple} $(GH, \Delta, \vep, S)$ by modifying the Hopf
algebroid definition so that 
$\widehat{H} = (\Lambda^*V, \wedge, \ID, \varepsilon, \hat{\tau},S)$ satisfies the standard
Grassmann-Hopf algebra axioms stated above. We may also say that
$(HG, \Delta, \vep, S)$ is a \emph{weak C*-Grassmann-Hopf
algebroid} when $H^{\wedge}$ is a unital C*-algebra (with $\mathbf 1$).  
We thus set $\mathbb F = \mathbb C~$. Note however
that the tangled-duals of Grassman-Hopf algebroids retain both the
intuitive interactions and the dynamic diagram advantages of their
physical, extended symmetry representations exhibited by the
Grassman-Hopf al/gebras and co-gebras over those of either weak
C*- Hopf algebroids or weak Hopf C*- algebras.

\begin{thebibliography}{9}

\bibitem{AS}
E. M. Alfsen and F. W. Schultz: \emph{Geometry of State Spaces of Operator Algebras}, Birkh\"auser, Boston--Basel--Berlin (2003).

\bibitem{ICB71}
I. Baianu : Categories, Functors and Automata Theory: A Novel Approach to Quantum Automata through Algebraic--Topological Quantum Computations., \textit{Proceed. 4th Intl. Congress LMPS}, (August-Sept. 1971).

\bibitem{BGB07}
I. C. Baianu, J. F. Glazebrook and R. Brown.: A Non--Abelian, Categorical Ontology of Spacetimes and Quantum Gravity., \emph{Axiomathes} \textbf{17},(3-4): 353-408(2007).

\bibitem{BBGGk8}
I.C.Baianu, R. Brown J.F. Glazebrook, and G. Georgescu, {\em Towards Quantum Non--Abelian Algebraic Topology}, (2008).

\bibitem{BSS}
F.A. Bais, B. J. Schroers and J. K. Slingerland: Broken quantum symmetry and confinement phases in planar physics, \emph{Phys. Rev. Lett.} \textbf{89} No. 18 (1--4): 181--201 (2002).

\bibitem{BJW}
J.W. Barrett.: Geometrical measurements in three-dimensional quantum gravity.
Proceedings of the Tenth Oporto Meeting on Geometry, Topology and Physics (2001).
\textit{Intl. J. Modern Phys.} \textbf{A 18} , October, suppl., 97--113 (2003)

\bibitem{Chaician}
M. Chaician and A. Demichev: \emph{Introduction to Quantum Groups}, World Scientific (1996).

\bibitem{Coleman}
Coleman and De Luccia: Gravitational effects on and of vacuum decay., \emph{Phys. Rev. D} \textbf{21}: 3305 (1980).

\bibitem{CF}
L. Crane and I.B. Frenkel. Four-dimensional topological quantum field theory, Hopf categories, and the canonical bases. Topology and physics. \textit{J. Math. Phys}. \textbf{35} (no. 10): 5136--5154 (1994).

\bibitem{DT96}
W. Drechsler and P. A. Tuckey:  On quantum and parallel transport in a Hilbert bundle over spacetime., Classical and Quantum Gravity, \textbf{13}:611-632 (1996).
doi: 10.1088/0264--9381/13/4/004

\bibitem{Drinfeld}
V. G. Drinfel'd: Quantum groups, In \emph{Proc. Int. Congress of
Mathematicians, Berkeley, 1986}, (ed. A. Gleason), Berkeley, 798-820 (1987).

\bibitem{Ellis}
G. J. Ellis: Higher dimensional crossed modules of algebras,
\emph{J. of Pure Appl. Algebra} \textbf{52}: 277-282 (1988), .

\bibitem{Etingof1}
P.. I. Etingof and A. N. Varchenko, Solutions of the Quantum Dynamical Yang-Baxter Equation and Dynamical Quantum Groups, \emph{Comm.Math.Phys.}, \textbf{196}: 591-640 (1998).

\bibitem{Etingof2}
P. I. Etingof and A. N. Varchenko: Exchange dynamical quantum
groups, \emph{Commun. Math. Phys.} \textbf{205} (1): 19-52 (1999)

\bibitem{Etingof3}
P. I. Etingof and O. Schiffmann: Lectures on the dynamical Yang--Baxter equations, in \emph{Quantum Groups and Lie Theory (Durham, 1999)}, pp. 89-129, Cambridge University Press, Cambridge, 2001.

\bibitem{Fauser2002}
B. Fauser: \emph{A treatise on quantum Clifford Algebras}. Konstanz,
Habilitationsschrift. \\ arXiv.math.QA/0202059 (2002).

\bibitem{Fauser2004}
B. Fauser: Grade Free product Formulae from Grassmann--Hopf Gebras.
Ch. 18 in R. Ablamowicz, Ed., \emph{Clifford Algebras: Applications to Mathematics, Physics and Engineering}, Birkh\"{a}user: Boston, Basel and Berlin, (2004).

\bibitem{Fell}
J. M. G. Fell.: The Dual Spaces of  C*--Algebras., \emph{Transactions of the American
Mathematical Society}, \textbf{94}: 365--403 (1960).

\bibitem{FernCastro}
F.M. Fernandez and E. A. Castro.:  \textit{(Lie) Algebraic Methods in Quantum Chemistry and Physics.}, Boca Raton: CRC Press, Inc  (1996).

\bibitem{Feynman}
 R. P. Feynman: Space--Time Approach to Non--Relativistic Quantum Mechanics, {\em Reviews 
of Modern Physics}, 20: 367--387 (1948). [It is also reprinted in (Schwinger 1958).]

\bibitem{frohlich:nonab}
A.~Fr{\"o}hlich: Non-Abelian Homological Algebra. {I}.{D}erived functors and satellites.\/, 
\emph{Proc. London Math. Soc.}, \textbf{11}(3): 239--252 (1961).

\bibitem{GR02}
R. Gilmore: \textit{Lie Groups, Lie Algebras and Some of Their Applications.},
Dover Publs., Inc.: Mineola and New York, 2005.

\bibitem{Hahn1}
P. Hahn: Haar measure for measure groupoids., \textit{Trans. Amer. Math. Soc}. \textbf{242}: 1--33(1978).

\bibitem{Hahn2}
P. Hahn: The regular representations of measure groupoids., \textit{Trans. Amer. Math. Soc}. \textbf{242}:34--72(1978).

\bibitem{HeynLifsctz}
R. Heynman and S. Lifschitz. 1958. \emph{Lie Groups and Lie Algebras}., New York and London: Nelson Press.

\bibitem{HLS2k8}
C. Heunen, N. P. Landsman, B. Spitters.: A topos for algebraic quantum theory, (2008)   \\ 
arXiv:0709.4364v2 [quant--ph]

\end{thebibliography}
%%%%%
%%%%%
\end{document}
