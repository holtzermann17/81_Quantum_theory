\documentclass[12pt]{article}
\usepackage{pmmeta}
\pmcanonicalname{AlgebroidStructuresAndExtendedSymmetries}
\pmcreated{2013-03-22 18:13:55}
\pmmodified{2013-03-22 18:13:55}
\pmowner{bci1}{20947}
\pmmodifier{bci1}{20947}
\pmtitle{algebroid structures and extended symmetries}
\pmrecord{50}{40819}
\pmprivacy{1}
\pmauthor{bci1}{20947}
\pmtype{Topic}
\pmcomment{trigger rebuild}
\pmclassification{msc}{81T25}
\pmclassification{msc}{81T18}
\pmclassification{msc}{81T13}
\pmclassification{msc}{81T10}
\pmclassification{msc}{81T05}
\pmclassification{msc}{81R50}
\pmclassification{msc}{55U35}
\pmsynonym{extensions of quantum operator algebras}{AlgebroidStructuresAndExtendedSymmetries}
%\pmkeywords{algebroids}
%\pmkeywords{QFT}
%\pmkeywords{symmetry sectors}
%\pmkeywords{groupoids}
%\pmkeywords{continuous function with compact support}
%\pmkeywords{convolution product}
%\pmkeywords{extensions of  quantum operator algebras}
%\pmkeywords{extended algebroid symmetries}
%\pmkeywords{double algebras}
%\pmkeywords{double algebroids}
\pmrelated{HamiltonianAlgebroids}
\pmrelated{QFTOrQuantumFieldTheories}
\pmrelated{LieAlgebroids}
\pmrelated{RCategory}
\pmrelated{RAlgebroid}
\pmrelated{AxiomsOfMetacategoriesAndSupercategories}
\pmrelated{MonoidalCategory}
\pmrelated{Groupoids}
\pmrelated{ETAS}
\pmdefines{algebroid structure}
\pmdefines{convolution product}
\pmdefines{pre-algebroid}
\pmdefines{algebroid extended symmetries}
\pmdefines{set of functions with finite support}

\endmetadata

% this is the default PlanetMath preamble.  as your 

% almost certainly you want these
\usepackage{amssymb}
\usepackage{amsmath}
\usepackage{amsfonts}

% define commands here
\usepackage{amsmath, amssymb, amsfonts, amsthm, amscd, latexsym, enumerate}
%%\usepackage{xypic}
\usepackage[mathscr]{eucal}
\theoremstyle{plain}
\newtheorem{lemma}{Lemma}[section]
\newtheorem{proposition}{Proposition}[section]
\newtheorem{theorem}{Theorem}[section]
\newtheorem{corollary}{Corollary}[section]
\theoremstyle{definition}
\newtheorem{definition}{Definition}[section]
\newtheorem{example}{Example}[section]
%\theoremstyle{remark}
\newtheorem{remark}{Remark}[section]
\newtheorem*{notation}{Notation}
\newtheorem*{claim}{Claim}
\renewcommand{\thefootnote}{\ensuremath{\fnsymbol{footnote}}}
\numberwithin{equation}{section}
\newcommand{\Ad}{{\rm Ad}}
\newcommand{\Aut}{{\rm Aut}}
\newcommand{\Cl}{{\rm Cl}}
\newcommand{\Co}{{\rm Co}}
\newcommand{\DES}{{\rm DES}}
\newcommand{\Diff}{{\rm Diff}}
\newcommand{\Dom}{{\rm Dom}}
\newcommand{\Hol}{{\rm Hol}}
\newcommand{\Mon}{{\rm Mon}}
\newcommand{\Hom}{{\rm Hom}}
\newcommand{\Ker}{{\rm Ker}}
\newcommand{\Ind}{{\rm Ind}}
\newcommand{\IM}{{\rm Im}}
\newcommand{\Is}{{\rm Is}}
\newcommand{\ID}{{\rm id}}
\newcommand{\GL}{{\rm GL}}
\newcommand{\Iso}{{\rm Iso}}
\newcommand{\rO}{{\rm O}}
\newcommand{\Sem}{{\rm Sem}}
\newcommand{\St}{{\rm St}}
\newcommand{\Sym}{{\rm Sym}}
\newcommand{\SU}{{\rm SU}}
\newcommand{\Tor}{{\rm Tor}}
\newcommand{\U}{{\rm U}}
\newcommand{\A}{\mathcal A}
\newcommand{\Ce}{\mathcal C}
\newcommand{\D}{\mathcal D}
\newcommand{\E}{\mathcal E}
\newcommand{\F}{\mathcal F}
\newcommand{\G}{\mathcal G}
\renewcommand{\H}{\mathcal H}
\renewcommand{\cL}{\mathcal L}
\newcommand{\Q}{\mathcal Q}
\newcommand{\R}{\mathcal R}
\newcommand{\cS}{\mathcal S}
\newcommand{\cU}{\mathcal U}
\newcommand{\W}{\mathcal W}
\newcommand{\bA}{\mathbb{A}}
\newcommand{\bB}{\mathbb{B}}
\newcommand{\bC}{\mathbb{C}}
\newcommand{\bD}{\mathbb{D}}
\newcommand{\bE}{\mathbb{E}}
\newcommand{\bF}{\mathbb{F}}
\newcommand{\bG}{\mathbb{G}}
\newcommand{\bK}{\mathbb{K}}
\newcommand{\bM}{\mathbb{M}}
\newcommand{\bN}{\mathbb{N}}
\newcommand{\bO}{\mathbb{O}}
\newcommand{\bP}{\mathbb{P}}
\newcommand{\bR}{\mathbb{R}}
\newcommand{\bV}{\mathbb{V}}
\newcommand{\bZ}{\mathbb{Z}}
\newcommand{\bfE}{\mathbf{E}}
\newcommand{\bfX}{\mathbf{X}}
\newcommand{\bfY}{\mathbf{Y}}
\newcommand{\bfZ}{\mathbf{Z}}
\renewcommand{\O}{\Omega}
\renewcommand{\o}{\omega}
\newcommand{\vp}{\varphi}
\newcommand{\vep}{\varepsilon}
\newcommand{\diag}{{\rm diag}}
\newcommand{\grp}{{\mathsf{G}}}
\newcommand{\dgrp}{{\mathsf{D}}}
\newcommand{\desp}{{\mathsf{D}^{\rm{es}}}}
\newcommand{\Geod}{{\rm Geod}}
\newcommand{\geod}{{\rm geod}}
\newcommand{\hgr}{{\mathsf{H}}}
\newcommand{\mgr}{{\mathsf{M}}}
\newcommand{\ob}{{\rm Ob}}
\newcommand{\obg}{{\rm Ob(\mathsf{G)}}}
\newcommand{\obgp}{{\rm Ob(\mathsf{G}')}}
\newcommand{\obh}{{\rm Ob(\mathsf{H})}}
\newcommand{\Osmooth}{{\Omega^{\infty}(X,*)}}
\newcommand{\ghomotop}{{\rho_2^{\square}}}
\newcommand{\gcalp}{{\mathsf{G}(\mathcal P)}}
\newcommand{\rf}{{R_{\mathcal F}}}
\newcommand{\glob}{{\rm glob}}
\newcommand{\loc}{{\rm loc}}
\newcommand{\TOP}{{\rm TOP}}
\newcommand{\wti}{\widetilde}
\newcommand{\what}{\widehat}
\renewcommand{\a}{\alpha}
\newcommand{\be}{\beta}
\newcommand{\ga}{\gamma}
\newcommand{\Ga}{\Gamma}
\newcommand{\de}{\delta}
\newcommand{\del}{\partial}
\newcommand{\ka}{\kappa}
\newcommand{\si}{\sigma}
\newcommand{\ta}{\tau}
\newcommand{\lra}{{\longrightarrow}}
\newcommand{\ra}{{\rightarrow}}
\newcommand{\rat}{{\rightarrowtail}}
\newcommand{\ovset}[1]{\overset {#1}{\ra}}
\newcommand{\ovsetl}[1]{\overset {#1}{\lra}}
\newcommand{\hr}{{\hookrightarrow}}
\newcommand{\<}{{\langle}}
%\usepackage{geometry, amsmath,amssymb,latexsym,enumerate}
%%%\usepackage{xypic}
\def\baselinestretch{1.1}
\hyphenation{prod-ucts}

%\geometry{textwidth= 16 cm, textheight=21 cm}

\newcommand{\sqdiagram}[9]{$$ \diagram  #1  \rto^{#2} \dto_{#4}&
#3  \dto^{#5} \\ #6    \rto_{#7}  &  #8   \enddiagram
\eqno{\mbox{#9}}$$ }

\def\C{C^{\ast}}

\newcommand{\labto}[1]{\stackrel{#1}{\longrightarrow}}

%\newenvironment{proof}{\noindent {\bf Proof} }{ \hfill $\Box$
%{\mbox{}}
\newcommand{\quadr}[4]{\begin{pmatrix} & #1& \\[-1.1ex] #2 & & #3\\[-1.1ex]& #4&
 \end{pmatrix}}
\def\D{\mathsf{D}}
\begin{document}
\subsection{Algebroid Structures and Algebroid Extended Symmetries}
 
\begin{definition}
An \emph{algebroid structure} $A$ will be specifically defined to mean
either a ring, or more generally, any of the specifically defined algebras, but \emph{with several
objects} instead of a single object, in the sense specified by Mitchell
(1965). Thus, an {\em algebroid}  has been defined (Mosa, 1986a; Brown and Mosa 1986b, 2008)
as follows.  An \textit{$R$-algebroid } $A$ on a set of ``objects" $A_0$
is a directed graph over $A_0$ such that for each $x,y \in A_0,\;
A(x,y)$ has an $R$-module structure and there is an $R$-bilinear
function $$ \circ : A(x,y) \times A(y,z) \to  A(x,z)$$ $(a , b)
\mapsto  a\circ b$ called ``composition" and satisfying the
associativity condition, and the existence of identities. 
\end{definition}


\begin{definition}
A {\em pre-algebroid} has the same structure as an algebroid and the same
axioms except for the fact that the existence of identities  $1_x \in A(x,x)$
is not assumed. For example, if $A_0$ has exactly one object, then
an $R$-algebroid $A$ over $A_0$ is just an $R$-algebra. An ideal
in $A$ is then an example of a pre-algebroid. 
\end{definition}

%%\cite{M1,M2,A}
 Let $R$ be a commutative ring. An \textit{$R$-category }$\A$ is a category equipped with an $R$-module structure on each \textit{hom} set such that the composition is $R$-bilinear. More precisely, let us assume for instance that we are given a commutative ring $R$ with identity. Then a small $R$-category--or equivalently an \emph{$R$-algebroid}-- will be defined as a category enriched in the monoidal category of $R$-modules, with respect to the
monoidal structure of tensor product. This means simply that for all objects $b,c$ of $\A$, the set $\A(b,c)$ is given the structure of an $R$-module, and composition $\A(b,c) \times \A(c,d) \lra
\A(b,d)$ is $R$--bilinear, or is a morphism of $R$-modules $\A(b,c) \otimes_R \A(c,d) \lra \A(b,d)$.

 If $\mathsf{G}$ is a \PMlinkname{groupoid}{Groupoids} (or, more generally, a category)
then we can construct an \emph{$R$-algebroid} $R\mathsf{G}$ as
follows. The object set of $R\mathsf{G}$ is the same as that of
$\mathsf{G}$ and $R\mathsf{G}(b,c)$ is the free $R$-module on the
set $\mathsf{G}(b,c)$, with composition given by the usual
bilinear rule, extending the composition of $\mathsf{G}$.

 Alternatively, one can define $\bar{R}\mathsf{G}(b,c)$ to be the
set of functions $\mathsf{G}(b,c)\lra R$ with finite support, and
then we define the \emph{convolution product} as follows:

\begin{equation}
(f*g)(z)= \sum \{(fx)(gy)\mid z=x\circ y \} ~.
\end{equation}

As it is very well known, only the second construction is natural
for the topological case, when one needs to replace `function' by
\PMlinkname{`continuous function with compact support'}{SmoothFunctionsWithCompactSupport} (or \emph{locally
compact support} for the \PMlinkname{QFT}{QFTOrQuantumFieldTheories} extended 
\PMlinkexternal{symmetry sectors}{http://planetmath.org/?op=getobj&from=books&id=153}), and in
this case $R \cong \mathbb{C}$~. The point made here is
that to carry out the usual construction and end up with only an algebra
rather than an algebroid, is a procedure analogous to replacing a
\PMlinkname{groupoid}{Groupoids} $\mathsf{G}$ by a semigroup $G'=G\cup \{0\}$ in which the
compositions not defined in $G$ are defined to be $0$ in $G'$. We
argue that this construction removes the main advantage of
\PMlinkname{groupoids}{Groupoids}, namely the spatial component given by the set of
objects.

\textbf{Remarks:}
One can also define categories of algebroids, $R$-algebroids, double algebroids , and so on.
A `category' of $R$-categories is however a \PMlinkname{super-category}{Supercategory} $\mathbb{S}$, or it can also be viewed as a specific example of a \PMlinkname{metacategory}{AxiomsOfMetacategoriesAndSupercategories} (or 
$R$-supercategory, in the more general case of multiple operations--categorical `composition laws'-- being defined within the same structure, for the same class, $C$).  


\begin{thebibliography}{9}
\bibitem{BBG2009}
I. C. Baianu , James F. Glazebrook, and Ronald Brown. 2009. Algebraic Topology Foundations of Supersymmetry and Symmetry Breaking in Quantum Field Theory and Quantum Gravity: A Review. \emph{SIGMA} 5 (2009), 051, 70 pages.      $arXiv:0904.3644$, $doi:10.3842/SIGMA.2009.051$, 
\PMlinkexternal{Symmetry, Integrability and Geometry: Methods and Applications (SIGMA)}{http://www.emis.de/journals/SIGMA/2009/051/}
\end{thebibliography}

%%%%%
%%%%%
\end{document}
