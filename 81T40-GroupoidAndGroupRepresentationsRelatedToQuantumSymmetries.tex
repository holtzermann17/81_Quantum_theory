\documentclass[12pt]{article}
\usepackage{pmmeta}
\pmcanonicalname{GroupoidAndGroupRepresentationsRelatedToQuantumSymmetries}
\pmcreated{2013-03-22 18:10:59}
\pmmodified{2013-03-22 18:10:59}
\pmowner{bci1}{20947}
\pmmodifier{bci1}{20947}
\pmtitle{groupoid and group representations related to quantum symmetries}
\pmrecord{80}{40755}
\pmprivacy{1}
\pmauthor{bci1}{20947}
\pmtype{Topic}
\pmcomment{trigger rebuild}
\pmclassification{msc}{81T40}
\pmclassification{msc}{81T25}
\pmclassification{msc}{81T18}
\pmclassification{msc}{81T13}
\pmclassification{msc}{20N02}
\pmclassification{msc}{81T10}
\pmclassification{msc}{81T05}
\pmclassification{msc}{22A22}
\pmclassification{msc}{20L05}
\pmclassification{msc}{18B40}
\pmsynonym{groupoid representations}{GroupoidAndGroupRepresentationsRelatedToQuantumSymmetries}
%\pmkeywords{quantum groups}
%\pmkeywords{Hopf algebras}
%\pmkeywords{groupoid and group representations}
%\pmkeywords{Haar measure}
%\pmkeywords{quantum symmetries}
%\pmkeywords{measured groupoids}
%\pmkeywords{Haar measure systems}
%\pmkeywords{locally compact (quantum) groups--not Hopf algebras}
%\pmkeywords{locally compact groupoids}
%\pmkeywords{quantum groupoids}
\pmrelated{Groupoid}
\pmrelated{Groupoids}
\pmrelated{GroupoidCategory}
\pmrelated{HopfAlgebra}
\pmrelated{GroupRepresentation}
\pmrelated{QuantumGroups}
\pmrelated{WeakHopfCAlgebra2}
\pmrelated{WeakHopfCAlgebra}
\pmrelated{CAlgebra3}
\pmrelated{QuantumGroupoids2}
\pmrelated{RepresentationsOfLocallyCompactGroupoids}
\pmrelated{LocallyCompactGroupoids}
\pmrelated{GroupoidRepresentationTheorem}
\pmrelated{2CCa}
\pmdefines{groupoid}
\pmdefines{locally compact groupoid}
\pmdefines{topological groupoid}
\pmdefines{groupoid representation}
\pmdefines{topological groupoid}
\pmdefines{Haar systems}
\pmdefines{groupoid representations}
\pmdefines{quantum group}

% this is the default PlanetMath preamble.  
\usepackage{amssymb}
\usepackage{amsmath}
\usepackage{amsfonts}

% define commands here
\usepackage{amsmath, amssymb, amsfonts, amsthm, amscd, latexsym, enumerate}
\usepackage{xypic, xspace}
\usepackage[mathscr]{eucal}
\usepackage[dvips]{graphicx}
\usepackage[curve]{xy}

\setlength{\textwidth}{6.5in}
%\setlength{\textwidth}{16cm}
\setlength{\textheight}{9.0in}
%\setlength{\textheight}{24cm}

\hoffset=-.75in     %%ps format
%\hoffset=-1.0in     %%hp format
\voffset=-.4in

\theoremstyle{plain}
\newtheorem{lemma}{Lemma}[section]
\newtheorem{proposition}{Proposition}[section]
\newtheorem{theorem}{Theorem}[section]
\newtheorem{corollary}{Corollary}[section]

\theoremstyle{definition}
\newtheorem{definition}{Definition}[section]
\newtheorem{example}{Example}[section]
%\theoremstyle{remark}
\newtheorem{remark}{Remark}[section]
\newtheorem*{notation}{Notation}
\newtheorem*{claim}{Claim}

\renewcommand{\thefootnote}{\ensuremath{\fnsymbol{footnote}}}
\numberwithin{equation}{section}

\newcommand{\Ad}{{\rm Ad}}
\newcommand{\Aut}{{\rm Aut}}
\newcommand{\Cl}{{\rm Cl}}
\newcommand{\Co}{{\rm Co}}
\newcommand{\DES}{{\rm DES}}
\newcommand{\Diff}{{\rm Diff}}
\newcommand{\Dom}{{\rm Dom}}
\newcommand{\Hol}{{\rm Hol}}
\newcommand{\Mon}{{\rm Mon}}
\newcommand{\Hom}{{\rm Hom}}
\newcommand{\Ker}{{\rm Ker}}
\newcommand{\Ind}{{\rm Ind}}
\newcommand{\IM}{{\rm Im}}
\newcommand{\Is}{{\rm Is}}
\newcommand{\ID}{{\rm id}}
\newcommand{\grpL}{{\rm GL}}
\newcommand{\Iso}{{\rm Iso}}
\newcommand{\rO}{{\rm O}}
\newcommand{\Sem}{{\rm Sem}}
\newcommand{\SL}{{\rm Sl}}
\newcommand{\St}{{\rm St}}
\newcommand{\Sym}{{\rm Sym}}
\newcommand{\Symb}{{\rm Symb}}
\newcommand{\SU}{{\rm SU}}
\newcommand{\Tor}{{\rm Tor}}
\newcommand{\U}{{\rm U}}

\newcommand{\A}{\mathcal A}
\newcommand{\Ce}{\mathcal C}
\newcommand{\D}{\mathcal D}
\newcommand{\E}{\mathcal E}
\newcommand{\F}{\mathcal F}
%\newcommand{\grp}{\mathcal G}
\renewcommand{\H}{\mathcal H}
\renewcommand{\cL}{\mathcal L}
\newcommand{\Q}{\mathcal Q}
\newcommand{\R}{\mathcal R}
\newcommand{\cS}{\mathcal S}
\newcommand{\cU}{\mathcal U}
\newcommand{\W}{\mathcal W}

\newcommand{\bA}{\mathbb{A}}
\newcommand{\bB}{\mathbb{B}}
\newcommand{\bC}{\mathbb{C}}
\newcommand{\bD}{\mathbb{D}}
\newcommand{\bE}{\mathbb{E}}
\newcommand{\bF}{\mathbb{F}}
\newcommand{\bG}{\mathbb{G}}
\newcommand{\bK}{\mathbb{K}}
\newcommand{\bM}{\mathbb{M}}
\newcommand{\bN}{\mathbb{N}}
\newcommand{\bO}{\mathbb{O}}
\newcommand{\bP}{\mathbb{P}}
\newcommand{\bR}{\mathbb{R}}
\newcommand{\bV}{\mathbb{V}}
\newcommand{\bZ}{\mathbb{Z}}

\newcommand{\bfE}{\mathbf{E}}
\newcommand{\bfX}{\mathbf{X}}
\newcommand{\bfY}{\mathbf{Y}}
\newcommand{\bfZ}{\mathbf{Z}}

\renewcommand{\O}{\Omega}
\renewcommand{\o}{\omega}
\newcommand{\vp}{\varphi}
\newcommand{\vep}{\varepsilon}

\newcommand{\diag}{{\rm diag}}
\newcommand{\grp}{{\mathsf{G}}}
\newcommand{\dgrp}{{\mathsf{D}}}
\newcommand{\desp}{{\mathsf{D}^{\rm{es}}}}
\newcommand{\grpeod}{{\rm Geod}}
%\newcommand{\grpeod}{{\rm geod}}
\newcommand{\hgr}{{\mathsf{H}}}
\newcommand{\mgr}{{\mathsf{M}}}
\newcommand{\ob}{{\rm Ob}}
\newcommand{\obg}{{\rm Ob(\mathsf{G)}}}
\newcommand{\obgp}{{\rm Ob(\mathsf{G}')}}
\newcommand{\obh}{{\rm Ob(\mathsf{H})}}
\newcommand{\Osmooth}{{\Omega^{\infty}(X,*)}}
\newcommand{\grphomotop}{{\rho_2^{\square}}}
\newcommand{\grpcalp}{{\mathsf{G}(\mathcal P)}}

\newcommand{\rf}{{R_{\mathcal F}}}
\newcommand{\grplob}{{\rm glob}}
\newcommand{\loc}{{\rm loc}}
\newcommand{\TOP}{{\rm TOP}}

\newcommand{\wti}{\widetilde}
\newcommand{\what}{\widehat}

\renewcommand{\a}{\alpha}
\newcommand{\be}{\beta}
\newcommand{\grpa}{\grpamma}
%\newcommand{\grpa}{\grpamma}
\newcommand{\de}{\delta}
\newcommand{\del}{\partial}
\newcommand{\ka}{\kappa}
\newcommand{\si}{\sigma}
\newcommand{\ta}{\tau}

\newcommand{\med}{\medbreak}
\newcommand{\medn}{\medbreak \noindent}
\newcommand{\bign}{\bigbreak \noindent}

\newcommand{\lra}{{\longrightarrow}}
\newcommand{\ra}{{\rightarrow}}
\newcommand{\rat}{{\rightarrowtail}}
\newcommand{\ovset}[1]{\overset {#1}{\ra}}
\newcommand{\ovsetl}[1]{\overset {#1}{\lra}}
\newcommand{\hr}{{\hookrightarrow}}

\newcommand{\<}{{\langle}}

%\newcommand{\>}{{\rangle}}

%\usepackage{geometry, amsmath,amssymb,latexsym,enumerate}
%%%\usepackage{xypic}

\def\baselinestretch{1.1}


\hyphenation{prod-ucts}

%\grpeometry{textwidth= 16 cm, textheight=21 cm}

\newcommand{\sqdiagram}[9]{$$ \diagram  #1  \rto^{#2} \dto_{#4}&
#3  \dto^{#5} \\ #6    \rto_{#7}  &  #8   \enddiagram
\eqno{\mbox{#9}}$$ }

\def\C{C^{\ast}}

\newcommand{\labto}[1]{\stackrel{#1}{\longrightarrow}}

%\newenvironment{proof}{\noindent {\bf Proof} }{ \hfill $\Box$
%{\mbox{}}

\newcommand{\quadr}[4]
{\begin{pmatrix} & #1& \\[-1.1ex] #2 & & #3\\[-1.1ex]& #4&
 \end{pmatrix}}
\def\D{\mathsf{D}}

\begin{document}
\section{Groupoid representations}

Whereas \PMlinkname{group representations}{GroupRepresentation} of quantum unitary operators are
extensively employed in standard quantum mechanics, the applications of 
\PMlinkname{groupoid representations}{RepresentationsOfLocallyCompactGroupoids} 
are still under development. For example, a description of stochastic quantum
mechanics in curved spacetime (Drechsler and Tuckey, 1996)
involving a Hilbert bundle is possible in terms of
\textit{groupoid representations} which can indeed be defined on
such a Hilbert bundle $(X*\H,\pi)$, but cannot be expressed as
the simpler group representations on a Hilbert space $\H$. On the
other hand, as in the case of group representations, unitary
groupoid representations induce associated C*-algebra
representations. In the next subsection we recall some of the
basic results concerning groupoid representations and their
associated groupoid *-algebra representations. For further
details and recent results in the mathematical theory of groupoid
representations one has also available the succint monograph by
Buneci (2003) and references cited therein (\textit{www.utgjiu.ro/math/mbuneci/preprint.html}).

Let us consider first the relationships between these mainly algebraic concepts and their extended
quantum symmetries, also including relevant computation examples;
then let us consider several further extensions of symmetry
and algebraic topology in the context of local quantum physics/algebraic quantum field theory,
symmetry breaking, quantum chromodynamics and the  development of novel supersymmetry theories of quantum gravity.
In this respect one can also take spacetime `inhomogeneity' as a
criterion for the comparisons between physical, partial or local,
symmetries: on the one hand, the example of paracrystals
reveals thermodynamic disorder (entropy) within its own spacetime
framework, whereas in spacetime itself, whatever the selected
model, the inhomogeneity arises through (super) gravitational
effects. More specifically, in the former case one has the
technique of the generalized Fourier--Stieltjes transform (along
with convolution and Haar measure), and in view of the latter, we
may compare the resulting `broken'/paracrystal--type symmetry with
that of the supersymmetry predictions for weak gravitational
fields (e.g., `ghost' particles) along with the broken
supersymmetry in the presence of intense gravitational fields.
Another significant extension of quantum symmetries may result
from the superoperator algebra/algebroids of Prigogine's quantum
\textit{superoperators} which are defined only for irreversible,
infinite-dimensional systems (Prigogine, 1980).


\subsection{Definition of extended quantum groupoid and algebroid symmetries}

 Quantum groups~ $\rightarrow$ Representations ~ $\rightarrow$ Weak Hopf algebras ~$\rightarrow$ ~Quantum groupoids and algebroids

   Our intention here is to view the latter scheme in terms of
\emph{weak Hopf C*--algebroid}-- and/or other-- extended
symmetries, which we propose to do, for example, by incorporating
the concepts of \emph{rigged Hilbert spaces} and \emph{sectional
functions for a small category}. We note, however, that an
alternative approach to quantum `groupoids' has already been
reported (Maltsiniotis, 1992), (perhaps also related to
noncommutative geometry); this was later expressed in terms of
deformation-quantization:  the Hopf algebroid deformation of the
universal enveloping algebras of Lie algebroids (Xu, 1997) as the
classical limit of a quantum `groupoid'; this also parallels the
introduction of quantum `groups' as the deformation-quantization
of Lie bialgebras. Furthermore, such a Hopf algebroid approach
(Lu, 1996) leads to categories of Hopf algebroid modules (Xu,
1997) which are monoidal, whereas the links between Hopf
algebroids and monoidal bicategories were investigated by Day and
Street (1997).

As defined under the following heading on groupoids, let 
$(\grp_{lc},\tau)$ be a \emph{locally compact groupoid} endowed with a (left) Haar system,
and let $A= C^*(\grp_{lc},\tau)$ be the convolution
$C^*$--algebra (we append $A$ with $\mathbf 1$ if necessary, so
that $A$ is unital). Then consider such a \textit{groupoid
representation} \\ $\Lambda :(\grp_{lc}, \tau) \lra \{\mathcal
H_x, \sigma_x \}_{x \in X}$  that respects a compatible measure
$\sigma_x$ on $\mathcal H_x$ (cf Buneci, 2003). On taking a state
$\rho$ on $A$, we assume a parametrization
\begin{equation} (\mathcal H_x, \sigma_x) := (\mathcal H_{\rho},
\sigma)_{x \in X}~.
\end{equation}

  Furthermore, each $\mathcal H_x$ is considered as a \emph{rigged Hilbert
space} Bohm and Gadella (1989), that is, one also has the following nested inclusions:
\begin{equation}
\Phi_x \subset (\mathcal H_x, \sigma_x) \subset
\Phi^{\times}_x~,
\end{equation}

in the usual manner, where $\Phi_x$ is a dense subspace of
$\mathcal H_x$ with the appropriate locally convex topology, and
$\Phi_x^{\times}$ is the space of continuous antilinear
functionals of $\Phi$~. For each $x \in X$, we require $\Phi_x$ to
be invariant under $\Lambda$ and $\IM~ \Lambda \vert \Phi_x$ is a
continuous representation of $\grp_{lc}$ on $\Phi_x$~. With these
conditions, representations of (proper) quantum groupoids that are
derived for weak C*--Hopf algebras (or algebroids) modeled on
rigged Hilbert spaces could be suitable generalizations in the
framework of a Hamiltonian generated semigroup of time evolution
of a quantum system via integration of Schr\"odinger's equation
$\iota \hslash \frac{\del \psi}{\del t} = H \psi$ as studied in
the case of Lie groups (Wickramasekara and Bohm, 2006). The
adoption of the rigged Hilbert spaces is also based on how the
latter are recognized as reconciling the Dirac and von Neumann
approaches to quantum theories (Bohm and Gadella, 1989).

  Next, let $\grp$ be a \emph{locally compact Hausdorff groupoid} and $X$ a
locally compact Hausdorff space. ($\grp$ will be called a \emph{locally compact groupoid,
or lc- groupoid} for short). In order to achieve a small C*--category 
we follow a suggestion of A. Seda (private communication) by using a 
general principle in the context of Banach bundles (Seda, 1976, 982)). 
Let $q= (q_1, q_2) : \grp \lra X \times X$ be a continuous, open and surjective map. 
For each $z = (x,y) \in X \times X$, consider the fibre 
$\grp_z = \grp (x,y) = q^{-1}(z)$, and set $\A_z = C_0(\grp_z) = C_0(\grp(x,y))$ equipped
with a uniform norm $\Vert ~ \Vert_z$~. Then we set $\A =
\bigcup_z \A_z$~. We form a Banach bundle $p : \A \lra X \times X$
as follows. Firstly, the projection is defined via the typical
fibre $p^{-1}(z) = \A_z = \A_{(x,y)}$~. Let $C_c(\grp)$ denote the
continuous complex valued functions on $\grp$ with compact
support. We obtain a sectional function $\wti{\psi} : X \times X
\lra \A$ defined via restriction as $\wti{\psi}(z) = \psi \vert
\grp_z = \psi \vert \grp (x,y)$~. Commencing from the vector space
$\gamma = \{ \wti{\psi} : \psi \in C_c(\grp) \}$, the set $\{
\wti{\psi}(z) : \wti{\psi} \in \gamma \}$ is dense in $\A_z$~. For
each $\wti{\psi} \in \gamma$, the function $\Vert \wti{\psi} (z)
\Vert_z$ is continuous on $X$, and each $\wti{\psi}$ is a
continuous section of $p : \A \lra X \times X$~. These facts
follow from Seda (1982, Theorem 1). Furthermore, under the convolution
product $f*g$, \textit{the space $C_c(\grp)$ forms an associative algebra
over $\bC$} (cf. Seda, 1982, Theorem 3).

\subsection{Groupoids}

Recall that a groupoid $\grp$ is, loosely speaking, a small
category with inverses over its set of objects $X = Ob(\grp)$~. One
often writes $\grp^y_x$ for the set of morphisms in $\grp$ from
$x$ to $y$~. \emph{A topological groupoid} consists of a space
$\grp$, a distinguished subspace $\grp^{(0)} = \obg \subset \grp$,
called {\it the space of objects} of $\grp$, together with maps
\begin{equation}
r,s~:~ \xymatrix{ \grp \ar@<1ex>[r]^r \ar[r]_s & \grp^{(0)} }
\end{equation}

called the {\it range} and {\it source maps} respectively,
together with a law of composition
\begin{equation}
\circ~:~ \grp^{(2)}: = \grp \times_{\grp^{(0)}} \grp = \{
~(\gamma_1, \gamma_2) \in \grp \times \grp ~:~ s(\gamma_1) =
r(\gamma_2)~ \}~ \lra ~\grp~,
\end{equation}

such that the following hold~:~
\begin{enumerate}
\item[(1)]
$s(\gamma_1 \circ \gamma_2) = r(\gamma_2)~,~ r(\gamma_1 \circ
\gamma_2) = r(\gamma_1)$~, for all $(\gamma_1, \gamma_2) \in
\grp^{(2)}$~.

\item[(2)]
$s(x) = r(x) = x$~, for all $x \in \grp^{(0)}$~.

\item[(3)]
$\gamma \circ s(\gamma) = \gamma~,~ r(\gamma) \circ \gamma =
\gamma$~, for all $\gamma \in \grp$~.

\item[(4)]
$(\gamma_1 \circ \gamma_2) \circ \gamma_3 = \gamma_1 \circ
(\gamma_2 \circ \gamma_3)$~.

\item[(5)]
Each $\gamma$ has a two--sided inverse $\gamma^{-1}$ with $\gamma
\gamma^{-1} = r(\gamma)~,~ \gamma^{-1} \gamma = s (\gamma)$~.

Furthermore, only for topological groupoids the inverse map needs be continuous.
It is usual to call $\grp^{(0)} = Ob(\grp)$ {\it the set of objects}
of $\grp$~. For $u \in Ob(\grp)$, the set of arrows $u \lra u$ forms a
group $\grp_u$, called the \emph{isotropy group of $\grp$ at $u$}.
\end{enumerate}

  Thus, as is well kown, a topological groupoid is just a groupoid internal to the category of topological spaces and continuous maps. The notion of internal groupoid has proved significant in a number of fields, since groupoids generalise bundles of groups, group actions, and equivalence relations. For a further study of groupoids we refer the reader to Brown (2006).


 Several examples of groupoids are:
\begin{itemize}
\item (a) locally compact groups, transformation groups, and any group in general (e.g. [59]
\item (b) equivalence relations
\item (c) tangent bundles
\item (d) the tangent groupoid (e.g. [4])
\item (e) holonomy groupoids for foliations (e.g. [4])
\item (f) Poisson groupoids (e.g. [81])
\item (g) graph groupoids (e.g. [47, 64]).
\end{itemize}

 As a simple example of a groupoid, consider (b) above. Thus, let \textit{R} be an \textit{equivalence relation} on a set X. Then \textit{R} is a groupoid under the following operations:
$(x, y)(y, z) = (x, z), (x, y)^{-1} = (y, x)$. Here, $\grp^0 = X $, (the diagonal of $X \times X$ ) and $r((x, y)) = x,  s((x, y)) = y$.

 So $ R^2$ = $\left\{((x, y), (y, z)) : (x, y), (y, z) \in R \right\} $.
When $R = X \times X $,  \textit{R} is called a \textit{trivial} groupoid. A special case of a trivial groupoid is
$R = R_n = \left\{ 1, 2, . . . , n \right\}$  $\times $ $\left\{ 1, 2, . . . , n \right\} $. (So every \textit{i} is equivalent to every \textit{j}). Identify $(i,j) \in R_n$ with the matrix unit $e_{ij}$. Then the groupoid $R_n$ is just matrix multiplication except that we only multiply $e_{ij}, e_{kl}$ when $k = j$, and $(e_{ij} )^{-1} = e_{ji}$. We do not really lose anything by restricting the multiplication, since the pairs $e_{ij}, {e_{kl}}$ excluded from groupoid multiplication just give the 0 product in normal algebra anyway.

\begin{definition}
 For a groupoid $\grp_{lc}$ to be a \emph{locally compact groupoid} means that $\grp_{lc}$ is required to be \emph{a (second countable) locally compact Hausdorff space}, and the product and also inversion maps are required to be continuous. Each $\grp_{lc}^u$ as well as the unit space $\grp_{lc}^0$ is closed in $\grp_{lc}$. 
\end{definition}

\begin{remark}
 What replaces the left Haar measure on $\grp_{lc}$ is a system of measures $\lambda^u$ ($u \in \grp_{lc}^0$), where $\lambda^u$ is a positive regular Borel measure on $\grp_{lc}^u$ with dense support. In addition, the $\lambda^u$ 's are required to vary continuously (when integrated against $f \in C_c(\grp_{lc}))$ and to form an invariant family in the sense that for each x, the map $y \mapsto xy$ is a measure preserving homeomorphism from $\grp_{lc}^s(x)$ onto $\grp_{lc}^r(x)$. Such a system
$\left\{ \lambda^u \right\}$ is called a \textit{left Haar system} for the locally compact groupoid $\grp_{lc}$.
\end{remark}


 This is defined more precisely next.

\subsection{Haar systems for locally compact topological groupoids}

 Let 
\begin{equation}
\xymatrix{ \grp \ar@<1ex>[r]^r \ar[r]_s & \grp^{(0)}}=X
\end{equation}
be a locally compact, locally trivial topological groupoid with
its transposition into transitive (connected) components. Recall
that for $x \in X$, the \emph{costar of $x$} denoted
$\rm{CO}^*(x)$ is defined as the closed set $\bigcup\{ \grp(y,x) :
y \in \grp \}$, whereby
\begin{equation}
\grp(x_0, y_0) \hookrightarrow \rm{CO}^*(x) \lra X~,
\end{equation}
is a principal $\grp(x_0, y_0)$--bundle relative to
fixed base points $(x_0, y_0)$~. Assuming all relevant sets are
locally compact, then following Seda (1976), a \emph{(left) Haar
system on $\grp$} denoted $(\grp, \tau)$ (for later purposes), is
defined to comprise of i) a measure $\kappa$ on $\grp$, ii) a
measure $\mu$ on $X$ and iii) a measure $\mu_x$ on $\rm{CO}^*(x)$
such that for every Baire set $E$ of $\grp$, the following hold on
setting $E_x = E \cap \rm{CO}^*(x)$~:
\begin{itemize}
\item[(1)] $x \mapsto \mu_x(E_x)$ is measurable.


\item[(2)] $\kappa(E) = \int_x \mu_x(E_x)~d\mu_x$ ~.


\item[(3)] $\mu_z(t E_x) = \mu_x(E_x)$, for all $t \in \grp(x,z)$ and $x, z
\in \grp$~.
\end{itemize}


The presence of a left Haar system on $\grp_{lc}$ has important
topological implications: it requires that the range map $r :
\grp_{lc} \rightarrow \grp_{lc}^0$ is open. For such a $\grp_{lc}$
with a left Haar system, the vector space $C_c(\grp_{lc})$ is a
\textit{convolution} \textit{*--algebra}, where for $f, g \in
C_c(\grp_{lc})$: \\
\med
$f * g(x) = \int f(t)g(t^{-1} x) d \lambda^{r(x)} (t)$,  with
f*(x) $ = \overline{f(x^{-1})}$.
\med
One has $C^*(\grp_{lc})$ to be the \textit{enveloping C*--algebra}
of $C_c(\grp_{lc})$ (and also representations are required to be
continuous in the inductive limit topology). Equivalently, it is
the completion of $\pi_{univ}(C_c(\grp_{lc}))$ where $\pi_{univ}$
is the \textit{universal representation} of $\grp_{lc}$. For
example, if $ \grp_{lc} = R_n$, then $C^*(\grp_{lc})$ is just the
finite dimensional algebra $C_c(\grp_{lc}) = M_n$, the span of the
$e_{ij}'$s.


There exists (cf. \cite{BRM2k3}) a \textit{measurable Hilbert bundle}
$(\grp_{lc}^0, \H, \mu)$ with $\H  = \left\{ \H^u_{u \in
\grp_{lc}^0} \right\}$ and a G-representation L on $\H$.  Then,
for every pair $\xi, \eta$ of square integrable sections of $\H$,
it is required that the function $x \mapsto (L(x)\xi (s(x)), \eta
(r(x)))$ be $\nu$--measurable. The representation $\Phi$ of
$C_c(\grp_{lc})$ is then given by:\\ $\left\langle \Phi(f) \xi
\vert,\eta \right\rangle = \int f(x)(L(x) \xi (s(x)), \eta (r(x)))
d \nu_0(x)$.


The triple $(\mu, \H, L)$ is called a \textit{measurable
$\grp_{lc}$--Hilbert bundle}.



\begin{thebibliography}{9}

\bibitem{AS}
E. M. Alfsen and F. W. Schultz: \emph{Geometry of State Spaces of Operator Algebras}, Birkh\"auser, Boston--Basel--Berlin (2003).

\bibitem{ICB71}
I. C. Baianu : Categories, Functors and Automata Theory: A Novel Approach to Quantum Automata through Algebraic--Topological Quantum Computations., \textit{Proceed. 4th Intl. Congress LMPS}, (August-Sept. 1971).

\bibitem{BGB07}
I. C. Baianu, J. F. Glazebrook and R. Brown.: A Non--Abelian, Categorical Ontology of Spacetimes and Quantum Gravity., \emph{Axiomathes} \textbf{17},(3-4): 353-408(2007).

\bibitem{BBGGk8}
I.C.Baianu, R. Brown J.F. Glazebrook, and G. Georgescu, Towards Quantum Non--Abelian Algebraic Topology. \textit{in preparation}, (2008).

\bibitem{BSS}
F.A. Bais, B. J. Schroers and J. K. Slingerland: Broken quantum symmetry and confinement phases in planar physics, \emph{Phys. Rev. Lett.} \textbf{89} No. 18 (1--4): 181--201 (2002).

\bibitem{BJW}
J.W. Barrett.: Geometrical measurements in three-dimensional quantum gravity.
Proceedings of the Tenth Oporto Meeting on Geometry, Topology and Physics (2001).
\textit{Intl. J. Modern Phys.} \textbf{A 18} , October, suppl., 97--113 (2003).

\bibitem{BRM2k3}
M. R. Buneci.: \emph{Groupoid Representations}, (orig. title ``Reprezentari de Grupoizi''),
Ed. Mirton: Timishoara (2003). 

\bibitem{Chaician}
M. Chaician and A. Demichev: \emph{Introduction to Quantum Groups}, World Scientific (1996).

\bibitem{Coleman}
Coleman and De Luccia: Gravitational effects on and of vacuum decay., \emph{Phys. Rev. D} \textbf{21}: 3305 (1980).

\bibitem{CF}
L. Crane and I.B. Frenkel. Four-dimensional topological quantum field theory, Hopf categories, and the canonical bases. Topology and physics. \textit{J. Math. Phys}. \textbf{35} (no. 10): 5136--5154 (1994).

\bibitem{DT96}
W. Drechsler and P. A. Tuckey:  On quantum and parallel transport in a Hilbert bundle over spacetime., \emph{Classical and Quantum Gravity}, \textbf{13}:611-632 (1996).
doi: 10.1088/0264--9381/13/4/004

\bibitem{Drinfeld}
V. G. Drinfel'd: Quantum groups, In \emph{Proc. Intl. Congress of
Mathematicians, Berkeley 1986}, (ed. A. Gleason), Berkeley, 798-820 (1987).

\bibitem{Ellis}
G. J. Ellis: Higher dimensional crossed modules of algebras,
\emph{J. of Pure Appl. Algebra} \textbf{52} (1988), 277-282.

\bibitem{Etingof1}
P.. I. Etingof and A. N. Varchenko, Solutions of the Quantum Dynamical Yang-Baxter Equation and Dynamical Quantum Groups, \emph{Comm.Math.Phys.}, \textbf{196}: 591-640 (1998).

\bibitem{Etingof2}
P. I. Etingof and A. N. Varchenko: Exchange dynamical quantum
groups, \emph{Commun. Math. Phys.} \textbf{205} (1): 19-52 (1999)

\bibitem{Etingof3}
P. I. Etingof and O. Schiffmann: Lectures on the dynamical Yang--Baxter equations, in \emph{Quantum Groups and Lie Theory (Durham, 1999)}, pp. 89-129, Cambridge University Press, Cambridge, 2001.

\bibitem{Fauser2002}
B. Fauser: \emph{A treatise on quantum Clifford Algebras}. Konstanz,
Habilitationsschrift. \\ arXiv.math.QA/0202059 (2002).

\bibitem{Fauser2004}
B. Fauser: Grade Free product Formulae from Grassman--Hopf Gebras.
Ch. 18 in R. Ablamowicz, Ed., \emph{Clifford Algebras: Applications to Mathematics, Physics and Engineering}, Birkh\"{a}user: Boston, Basel and Berlin, (2004).

\bibitem{Fell}
J. M. G. Fell.: The Dual Spaces of  C*--Algebras., \emph{Transactions of the American
Mathematical Society}, \textbf{94}: 365--403 (1960).

\bibitem{FernCastro}
F.M. Fernandez and E. A. Castro.:  \emph{(Lie) Algebraic Methods in Quantum Chemistry and Physics.}, Boca Raton: CRC Press, Inc  (1996).

\bibitem{Feynman}
 R. P. Feynman: Space--Time Approach to Non--Relativistic Quantum Mechanics, {\em Reviews 
of Modern Physics}, 20: 367--387 (1948). [It is also reprinted in (Schwinger 1958).]

\bibitem{frohlich:nonab}
A.~Fr{\"o}hlich: Non--Abelian Homological Algebra. {I}.
{D}erived functors and satellites.\/, \emph{Proc. London Math. Soc.}, \textbf{11}(3): 239--252 (1961).

\bibitem{GR02}
R. Gilmore: \emph{Lie Groups, Lie Algebras and Some of Their Applications.},
Dover Publs., Inc.: Mineola and New York, 2005.

\bibitem{Hahn1}
P. Hahn: Haar measure for measure groupoids., \textit{Trans. Amer. Math. Soc}. \textbf{242}: 1--33(1978).

\bibitem{Hahn2}
P. Hahn: The regular representations of measure groupoids., \textit{Trans. Amer. Math. Soc}. \textbf{242}:34--72(1978).

\bibitem{HeynLifsctz}
R. Heynman and S. Lifschitz. 1958. \emph{Lie Groups and Lie Algebras}., New York and London: Nelson Press.

\bibitem{HLS2k8}
C. Heunen, N. P. Landsman, B. Spitters.: A topos for algebraic quantum theory, (2008)   \\ 
arXiv:0709.4364v2 [quant--ph]

\end{thebibliography}

%%%%%
%%%%%
\end{document}
