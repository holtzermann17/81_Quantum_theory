\documentclass[12pt]{article}
\usepackage{pmmeta}
\pmcanonicalname{FermatNumbers}
\pmcreated{2013-03-22 11:42:46}
\pmmodified{2013-03-22 11:42:46}
\pmowner{drini}{3}
\pmmodifier{drini}{3}
\pmtitle{Fermat numbers}
\pmrecord{30}{30068}
\pmprivacy{1}
\pmauthor{drini}{3}
\pmtype{Definition}
\pmcomment{trigger rebuild}
\pmclassification{msc}{81T45}
\pmclassification{msc}{81T13}
\pmclassification{msc}{11A51}
\pmclassification{msc}{20L05}
\pmclassification{msc}{46L87}
\pmclassification{msc}{43A35}
\pmclassification{msc}{43A25}
\pmclassification{msc}{22D25}
\pmclassification{msc}{55U40}
\pmclassification{msc}{18B40}
\pmclassification{msc}{46L05}
\pmclassification{msc}{22A22}
\pmclassification{msc}{81R50}
\pmclassification{msc}{55U35}
%\pmkeywords{primality}
%\pmkeywords{primes}
%\pmkeywords{Fermat}
%\pmkeywords{Gauss}
%\pmkeywords{Mersenne}
\pmdefines{Fermat prime}

\endmetadata

\usepackage{amsmath}
\begin{document}
The $n$-th \emph{Fermat number} is defined as:
\[
  F_n=2^{2^n}+1.
\]

Fermat incorrectly conjectured that all these numbers were primes,
although he had no proof. 
The first 5 Fermat numbers: $3, 5, 17,257,65537$ (corresponding to $n=0,1,2,3,4$) are all primes (so called Fermat primes)
Euler was the first to point out the falsity of Fermat's conjecture 
by proving that $641$ is a divisor of $F_5$. (In fact, $F_5=641\times6700417$).
Moreover, no other Fermat number is known to be prime for $n>4$, so now it is conjectured that those are all prime Fermat numbers. It is also unknown whether there are infinitely many composite Fermat numbers or not.

One of the famous achievements of Gauss was to prove that the regular polygon of $m$ sides can be constructed with ruler and compass if and only if $m$ can be written as
$$m=2^k F_{r_1}F_{r_2}\cdots F_{r_t}$$
where $k\ge 0$ and the other factors are distinct primes of the form $F_n$ (of course, $t$ may be $0$ here, i.e. $m=2^k$ is allowed).

There are many interesting properties involving Fermat numbers. For instance:
\[
F_m = F_0F_1\cdots F_{m-1}+2
\]
for any $m\geq 1$, which implies that $F_m-2$ is divisible by all smaller Fermat numbers.

The previous formula holds because
\[
F_m -2 = (2^{2^m}+1)-2 = 2^{2^m}-1 = (2^{2^{m-1}}-1)(2^{2^{m-1}}+1) =  (2^{2^{m-1}}-1) F_{m-1}
\]
and expanding recursively the left factor in the last expression gives the desired result.

\bigskip
\textbf{References.}\\
Kr\'\i zek, Luca, Somer. \emph{17 Lectures on Fermat Numbers.} CMS Books in Mathematics.
%%%%%
%%%%%
%%%%%
%%%%%
%%%%%
%%%%%
%%%%%
%%%%%
%%%%%
%%%%%
%%%%%
%%%%%
%%%%%
%%%%%
\end{document}
