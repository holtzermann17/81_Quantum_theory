\documentclass[12pt]{article}
\usepackage{pmmeta}
\pmcanonicalname{TwodimensionalFourierTransforms}
\pmcreated{2013-03-22 18:18:56}
\pmmodified{2013-03-22 18:18:56}
\pmowner{bci1}{20947}
\pmmodifier{bci1}{20947}
\pmtitle{two-dimensional Fourier transforms}
\pmrecord{57}{40940}
\pmprivacy{1}
\pmauthor{bci1}{20947}
\pmtype{Topic}
\pmcomment{trigger rebuild}
\pmclassification{msc}{81V80}
\pmclassification{msc}{81V55}
\pmclassification{msc}{42B10}
\pmsynonym{2D-FT imaging}{TwodimensionalFourierTransforms}
%\pmkeywords{two-dimensional Fourier transforms}
%\pmkeywords{2D-FT NMR and NIR Spectroscopies}
\pmrelated{FourierStieltjesTransform}
\pmrelated{FourierTransform}
\pmrelated{TableOfGeneralizedFourierAndMeasuredGroupoidTransforms}
\pmrelated{SpinNetworksAndSpinFoams}
\pmrelated{FourierSineAndCosineSeries}
\pmrelated{DiscreteFourierTransform}
\pmdefines{2D-FT}
\pmdefines{2D-FT NMRI}
\pmdefines{MRI}
\pmdefines{NMRI}
\pmdefines{2D-FT NMR}

% this is the default PlanetMath preamble.  as your knowledge
% of TeX increases, you will probably want to edit this, but
% it should be fine as is for beginners.

% almost certainly you want these
\usepackage{amssymb}
\usepackage{amsmath}
\usepackage{amsfonts}

% used for TeXing text within eps files
%\usepackage{psfrag}
% need this for including graphics (\includegraphics)
%\usepackage{graphicx}
% for neatly defining theorems and propositions
%\usepackage{amsthm}
% making logically defined graphics
%%%\usepackage{xypic}

% there are many more packages, add them here as you need them

% define commands here
\usepackage{amsmath, amssymb, amsfonts, amsthm, amscd, latexsym}
%%\usepackage{xypic}
\usepackage[mathscr]{eucal}

\setlength{\textwidth}{6.5in}
%\setlength{\textwidth}{16cm}
\setlength{\textheight}{9.0in}
%\setlength{\textheight}{24cm}

\hoffset=-.75in     %%ps format
%\hoffset=-1.0in     %%hp format
\voffset=-.4in

\theoremstyle{plain}
\newtheorem{lemma}{Lemma}[section]
\newtheorem{proposition}{Proposition}[section]
\newtheorem{theorem}{Theorem}[section]
\newtheorem{corollary}{Corollary}[section]

\theoremstyle{definition}
\newtheorem{definition}{Definition}[section]
\newtheorem{example}{Example}[section]
%\theoremstyle{remark}
\newtheorem{remark}{Remark}[section]
\newtheorem*{notation}{Notation}
\newtheorem*{claim}{Claim}

\renewcommand{\thefootnote}{\ensuremath{\fnsymbol{footnote%%@
}}}
\numberwithin{equation}{section}

\newcommand{\Ad}{{\rm Ad}}
\newcommand{\Aut}{{\rm Aut}}
\newcommand{\Cl}{{\rm Cl}}
\newcommand{\Co}{{\rm Co}}
\newcommand{\DES}{{\rm DES}}
\newcommand{\Diff}{{\rm Diff}}
\newcommand{\Dom}{{\rm Dom}}
\newcommand{\Hol}{{\rm Hol}}
\newcommand{\Mon}{{\rm Mon}}
\newcommand{\Hom}{{\rm Hom}}
\newcommand{\Ker}{{\rm Ker}}
\newcommand{\Ind}{{\rm Ind}}
\newcommand{\IM}{{\rm Im}}
\newcommand{\Is}{{\rm Is}}
\newcommand{\ID}{{\rm id}}
\newcommand{\GL}{{\rm GL}}
\newcommand{\Iso}{{\rm Iso}}
\newcommand{\Sem}{{\rm Sem}}
\newcommand{\St}{{\rm St}}
\newcommand{\Sym}{{\rm Sym}}
\newcommand{\SU}{{\rm SU}}
\newcommand{\Tor}{{\rm Tor}}
\newcommand{\U}{{\rm U}}

\newcommand{\A}{\mathcal A}
\newcommand{\Ce}{\mathcal C}
\newcommand{\D}{\mathcal D}
\newcommand{\E}{\mathcal E}
\newcommand{\F}{\mathcal F}
\newcommand{\G}{\mathcal G}
\newcommand{\Q}{\mathcal Q}
\newcommand{\R}{\mathcal R}
\newcommand{\cS}{\mathcal S}
\newcommand{\cU}{\mathcal U}
\newcommand{\W}{\mathcal W}

\newcommand{\bA}{\mathbb{A}}
\newcommand{\bB}{\mathbb{B}}
\newcommand{\bC}{\mathbb{C}}
\newcommand{\bD}{\mathbb{D}}
\newcommand{\bE}{\mathbb{E}}
\newcommand{\bF}{\mathbb{F}}
\newcommand{\bG}{\mathbb{G}}
\newcommand{\bK}{\mathbb{K}}
\newcommand{\bM}{\mathbb{M}}
\newcommand{\bN}{\mathbb{N}}
\newcommand{\bO}{\mathbb{O}}
\newcommand{\bP}{\mathbb{P}}
\newcommand{\bR}{\mathbb{R}}
\newcommand{\bV}{\mathbb{V}}
\newcommand{\bZ}{\mathbb{Z}}

\newcommand{\bfE}{\mathbf{E}}
\newcommand{\bfX}{\mathbf{X}}
\newcommand{\bfY}{\mathbf{Y}}
\newcommand{\bfZ}{\mathbf{Z}}

\renewcommand{\O}{\Omega}
\renewcommand{\o}{\omega}
\newcommand{\vp}{\varphi}
\newcommand{\vep}{\varepsilon}

\newcommand{\diag}{{\rm diag}}
\newcommand{\grp}{{\mathbb G}}
\newcommand{\dgrp}{{\mathbb D}}
\newcommand{\desp}{{\mathbb D^{\rm{es}}}}
\newcommand{\Geod}{{\rm Geod}}
\newcommand{\geod}{{\rm geod}}
\newcommand{\hgr}{{\mathbb H}}
\newcommand{\mgr}{{\mathbb M}}
\newcommand{\ob}{{\rm Ob}}
\newcommand{\obg}{{\rm Ob(\mathbb G)}}
\newcommand{\obgp}{{\rm Ob(\mathbb G')}}
\newcommand{\obh}{{\rm Ob(\mathbb H)}}
\newcommand{\Osmooth}{{\Omega^{\infty}(X,*)}}
\newcommand{\ghomotop}{{\rho_2^{\square}}}
\newcommand{\gcalp}{{\mathbb G(\mathcal P)}}

\newcommand{\rf}{{R_{\mathcal F}}}
\newcommand{\glob}{{\rm glob}}
\newcommand{\loc}{{\rm loc}}
\newcommand{\TOP}{{\rm TOP}}

\newcommand{\wti}{\widetilde}
\newcommand{\what}{\widehat}

\renewcommand{\a}{\alpha}
\newcommand{\be}{\beta}
\newcommand{\ga}{\gamma}
\newcommand{\Ga}{\Gamma}
\newcommand{\de}{\delta}
\newcommand{\del}{\partial}
\newcommand{\ka}{\kappa}
\newcommand{\si}{\sigma}
\newcommand{\ta}{\tau}
\newcommand{\med}{\medbreak}
\newcommand{\medn}{\medbreak \noindent}
\newcommand{\bign}{\bigbreak \noindent}
\newcommand{\lra}{{\longrightarrow}}
\newcommand{\ra}{{\rightarrow}}
\newcommand{\rat}{{\rightarrowtail}}
\newcommand{\oset}[1]{\overset {#1}{\ra}}
\newcommand{\osetl}[1]{\overset {#1}{\lra}}
\newcommand{\hr}{{\hookrightarrow}}

\begin{document}
\section{Introduction}

 A two-dimensional Fourier transform (2D-FT) is computed numerically, or carried out, in two stages that are both
involving `standard', one-dimensional Fourier transforms. However, the second stage
Fourier transform is \emph{not the inverse} Fourier transform (which would result in the original 
function that was transformed at the first stage), but a Fourier transform in a second variable--
which is `shifted' in value-- relative to that involved in the result of the first Fourier transform.
Such 2D-FT analysis is a very powerful method for three-dimensional reconstruction
of polymer and biopolymer structures by two-dimensional Nuclear Magnetic Resonance (2D-NMR, \cite{KurtWutrich86})
of solutions for molecular weights ($M_w$) of the dissolved polymers up to about 50,000 $M_w$. 
For larger biopolymers or polymers, more complex methods have been developed to obtain the desired
resolution needed for the 3D-reconstruction of higher molecular structures, e.g. for $900,000 M_w$,
methods that can also be utilized \emph{in vivo}. The 2D-FT method is also widely utilized in optical spectroscopy, such as \emph{2D-FT NIR hyperspectral imaging}, or in \emph{MRI imaging} for research and clinical, diagnostic applications in Medicine. 

\section{Basic definition}
  A more precise mathematical definition of the `double' Fourier transform involved is specified next. 

\begin{definition}
  A 2D-FT, or two-dimensional Fourier transform, is a standard Fourier transformation of a function
of two variables, $f(x_1, x_2)$, carried first in the first variable $x_1$, followed by the Fourier transform
in the second variable $x_2$ of the resulting function $F(s_1, x_2)$. (For further specific details and example
for 2D-FT Imaging v. URLs provided in the following recent Bibliography).
\end{definition}

\subsection{Examples}
 
 A 2D Fourier transformation and phase correction is applied to a set of 2D NMR (FID) signals $s(t_1, t_2)$ yielding a real 2D-FT NMR `spectrum' (collection of 1D FT-NMR spectra) represented by a matrix $S$ whose elements are 
$$S(\nu_1,\nu_2) = \textbf{Re} \int \int cos(\nu_1 t_1)exp^{(-i\nu_2 t_2)} s(t_1, t_2)dt_1 dt_2$$
where $\nu_1$ and $\nu_2$ denote the discrete indirect double-quantum and single-quantum(detection) axes, respectively,
in the 2D NMR experiments. Next, the \emph{covariance matrix} is calculated in the frequency domain according
to the following equation 
$$ C(\nu_2', \nu_2) =  S^T S = \sum_{\nu^1}[S(\nu_1,\nu_2')S(\nu_1,\nu_2)],$$

 with $\nu_2, \nu_2'$ taking all possible single-quantum frequency values and  with the summation carried out over all discrete, double quantum
frequencies $\nu_1$.


 \PMlinkexternal{2D-FT STEM Images (obtained at Cornell University) of electron distributions in a high-temperature
cuprate superconductor `paracrystal'}{http://74.125.95.132/search?q=cache:x6OQWq_GVoYJ:www.physorg.com/multimedia/pix1815/+http://www.physorg.com/multimedia/pix1815/&hl=en&ct=clnk&cd=1&gl=uk} reveal both the domains (or `location') and the local symmetry of the ``pseudo-gap'' in the electron-pair correlation band responsible for the high--temperature superconductivity effect (a definite
possibility for the next Nobel (?) iff the mathematical physics treatment is also developed to include also
such results). 

  So far there have been three Nobel prizes awarded for 2D-FT NMR/MRI during 1992-2003, and an additional,
earlier Nobel prize for 2D-FT of X-ray data (`CAT scans'); recently the advanced possibilities
of 2D-FT techniques in 
\PMlinkexternal{Chemistry}{http://nobelprize.org/nobel_prizes/chemistry/laureates/1991/ernst-lecture.pdf}, Physiology and Medicine received very significant recognition.

\begin{thebibliography}{9}

\bibitem{KurtWutrich86}
Kurt W\''{u}trich:  1986, \emph{NMR of Proteins and Nucleic Acids.}, J. Wiley and Sons:
New York, Chichester, Brisbane, Toronto, Singapore. 
\PMlinkexternal{(Nobel Laureate in 2002 for 2D-FT NMR Studies of Structure and Function of Biological Macromolecules)}{http://nobelprize.org/nobel_prizes/chemistry/laureates/2002/wutrich-lecture.pdf};
\PMlinkexternal{2D-FT NMR Instrument Image Example: a JPG color image of a 2D-FT NMR Imaging `monster' Instrument}{http://upload.wikimedia.org/wikipedia/en/b/bf/HWB-NMRv900.jpg} 

\bibitem{RICHARDRERNST1992}
Richard R. Ernst. 1992. Nuclear Magnetic Resonance Fourier Transform (2D-FT) Spectroscopy.
\PMlinkexternal{Nobel Lecture}{http://nobelprize.org/nobel_prizes/chemistry/laureates/1991/ernst-lecture.pdf}, on December 9, 1992.

\bibitem{PM2k3}
Peter Mansfield. 2003. \PMlinkexternal{Nobel Laureate in Physiology and Medicine for (2D and 3D) MRI.}{http://www.parteqinnovations.com/pdf-doc/fandr-Gaz1006.pdf}

\bibitem{MRI-2DFT}
D. Benett. 2007. \emph{PhD Thesis}. Worcester Polytechnic Institute. ({\em lots of 2D-FT images of mathematical, brain scans}.)
\PMlinkexternal{PDF of 2D-FT Imaging Applications to MRI in Medical Research}{http://www.wpi.edu/Pubs/ETD/Available/etd-081707-080430/unrestricted/dbennett.pdf}. 

\bibitem{PL2k3}
Paul Lauterbur. 2003. 
\PMlinkexternal{Nobel Laureate in Physiology and Medicine for (2D and 3D) MRI.}{http://nobelprize.org/nobel_prizes/medicine/laureates/2003/}

\bibitem{JeanJeneer1971}
Jean Jeener. 1971. Two-dimensional Fourier Transform NMR, presented at an Amp\`ere International Summer School, Basko Polje, \emph{unpublished}. A verbatim quote follows from Richard R. Ernst's Nobel Laureate Lecture
delivered on December 2nd, 1992, ``A new approach to measure two-dimensional (2D) spectra has been
proposed by Jean Jeener at an Ampere Summer School in Basko Polje, Yugoslavia, 1971 (\cite{JeanJeneer1971}). He suggested a 2D Fourier transform experiment consisting of two $\pi/2$ pulses with a variable time $t_1$ between the pulses and the time variable $t_2$ measuring the time elapsed after the second pulse as shown in Fig. 6 that expands the principles of Fig. 1. Measuring the response $s(t_1,t_2)$ of the two-pulse sequence and Fourier-transformation with
respect to both time variables produces a two-dimensional spectrum $S(O_1,O_2)$ of the desired form (62,63). This two-pulse experiment by Jean Jeener is the forefather of a whole class of $2D$ experiments (8,63) 
that can also easily be expanded to multidimensional spectroscopy.''

\end{thebibliography}

%%%%%
%%%%%
\end{document}
